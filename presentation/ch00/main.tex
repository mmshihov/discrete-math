%include part: see main.beamer.tex and main.article.tex
%include common packages and settings
\usepackage{etex} %эта магическая херь избавляет от переполнения регистров TeX а!!!

\mode<article>{\usepackage{fullpage}}
\mode<presentation>{
    \usetheme{Madrid} %%Boadilla,Madrid,AnnArbor,CambridgeUS,Malmoe,Singapore,Berlin
    \useoutertheme{shadow}
} 

\usepackage[utf8]{inputenc}
\usepackage[russian]{babel}
\usepackage{indentfirst}
\usepackage{graphicx}

\usepackage{amsmath}
\usepackage{amsfonts}
\usepackage{amsthm}
\usepackage{algorithm}
\usepackage{algorithmic}

\usepackage[all]{xy}

\date{Лекция по дисциплине <<дискретная математика>>\\(\today)}
\author[М.~М.~Шихов]{Михаил Шихов \\ \texttt{\underline{m.m.shihov@gmail.com}}}

%для рисования графов пакетом xy-pic
\entrymodifiers={++[o][F-]}

%для псевдокода алгоритмов (algorithm,algorithmic)
\renewcommand{\algorithmicrequire}{\textbf{Вход:}}
\renewcommand{\algorithmicensure}{\textbf{Выход:}}
\renewcommand{\algorithmiccomment}[1]{// #1}
\floatname{algorithm}{Псевдокод}



\title[Системы счисления]{Позиционные системы счисления}


\begin{document}

%титул и содержание статьи
\mode<article>{\maketitle\tableofcontents}

%титул и содержание презентации
\frame<presentation>{\titlepage}
\begin{frame}<presentation>
    \frametitle{Содержание}
    \tableofcontents
\end{frame}


\section{Символьное представление чисел}


\subsection{Римская система счисления}

\begin{frame}
    \frametitle{Римская система счисления}
    
    \begin{block}{}
        \begin{center}
            \begin{tabular}{lll}
                \hline\hline
                    1       &I   &лат. unus\\
                    5       &V   &лат. quinque\\
                    10      &X   &лат. decem\\
                    50      &L   &лат. quinquaginta\\
                    100     &C   &лат. centum\\
                    500     &D   &лат. quingenti\\
                    1000    &M   &лат. mille\\
                \hline\hline
            \end{tabular}
        \end{center}
    \end{block}
    \begin{itemize}
        \item II --- 2;
        \item IV --- 4;
        \item VI --- 6;
        \item XCIX --- 99;
        \item MMMCMXCIX --- 3999\ldots и это \alert{предел}!
    \end{itemize}
\end{frame}


\subsection{Позиционная система счисления}


\begin{frame}
    \frametitle{Натуральное число}
    \framesubtitle{В позиционной системе счисления с основанием $K$}

    Представление числа в $K$-ичной системе счисления:
    \[
        (a_{n-1}\cdots a_{1}a_{0})_K,
    \]
    
    Число, соответствующее представлению \footnote{Замкнутая запись суммы:\[\sum_{i=1}^{n}x_i = x_1 + x_2 + \ldots + x_n\]}:
    \[
        \sum_{i=0}^{n-1}a_i\cdot K^{i}.
    \]
\end{frame}

\begin{frame}
    \frametitle{Вклад разряда}

    Каким бы большим не было натуральное число, рано или поздно все цифры в разрядах старше некоторого $(n-1)$-го будут нулевыми:
    \[
        (\cdots 0000000a_{n-1}\cdots a_{1}a_{0})_K,
    \]
    где $a_{n-1}\neq 0$. Поэтому бесконечный ряд нулей слева в записи числа опускают.
\end{frame}

\begin{frame}
    \frametitle{Вклад $n$-го разряда}
    
    Вклад $n$-го разряда при $a_n\neq 0$ больше вклада младших\footnote{$100>99$, $1000>999$,\ldots}:
    \[
        a_n\cdot K^n \geq 1+\sum_{i=0}^{n-1}a_i\cdot K^i.
    \]

    Рано или поздно для сколь угодно большого числа:
    \[
        (\cdots 0000000a_{n-1}\cdots a_{1}a_{0})_K,
    \]
    где $a_{n-1}\neq 0$.
\end{frame}

\begin{frame}
    \begin{example}[Число в десятичной СС]
        Записи $(78642)_{10}$ (большинство запишет просто \textbf{78642}) соответствует число
        \[
            7\cdot 10^{4}+
            8\cdot 10^{3}+
            6\cdot 10^{2}+
            4\cdot 10^{1}+
            2\cdot 10^{0}=78642.
        \]
    \end{example}
\end{frame}

\begin{frame}
    \begin{example}[Число в двоичной СС]
        Записи $(10101)_{2}$  соответствует число
        \[
            \begin{array}[c]{c}
                1\cdot 2^{4}+
                0\cdot 2^{3}+
                1\cdot 2^{2}+
                0\cdot 2^{1}+
                1\cdot 2^{0} = \\
                =1\cdot 16+
                0\cdot 8+
                1\cdot 4+
                0\cdot 2+
                1\cdot 1 = \\
                = 21.
            \end{array}
        \]
    \end{example}
\end{frame}

\begin{frame}
    \begin{example}[Число в троичной СС]
        Записи $(10221)_{3}$  соответствует число
        \[
            \begin{array}[c]{c}
                1\cdot 3^{4}+
                0\cdot 3^{3}+
                2\cdot 3^{2}+
                2\cdot 3^{1}+
                1\cdot 3^{0} = \\
                =1\cdot 81+
                0\cdot 27+
                2\cdot 9+
                2\cdot 3+
                1\cdot 1 = \\
                = 106.
            \end{array}
        \]
    \end{example}
\end{frame}

\begin{frame}
    \begin{example}[Символ цифры]
        В ручной записи \emph{числа} каждой \emph{цифре} соответствует \emph{символ} определенного начертания. Поэтому, если оговорено, например, что $K=3$ и цифре
        $\alpha$ соответствует ноль, $\beta$ --- один, $\gamma$ --- два, то записи 
        \[
            (\beta\alpha\gamma\gamma\beta)_3
        \]
        соответствует число $106$. 
    \end{example}
\end{frame}


\begin{frame}
    \frametitle{Вещественное число}
    \framesubtitle{Дробная часть $Y$, $0\leq Y < 1$}
    
    В позиционной системе счисления с основанием $K$ представляется так:
    \[
        Y\equiv(.a_{-1}a_{-2}\cdots a_{-m}\cdots)_K,
    \] где $m>0$ --- натуральное число.

    Дробная на основе своего представления формируется так:
    \[
        Y=
            \sum_{i=-m}^{-1}a_{i}\cdot K^{i}=
            \sum_{i=1}^{m}\frac{a_{-i}}{K^{i}}.
    \]
\end{frame}

\begin{frame}
    \frametitle{Иррациональные числа}
    
    Для записи иррациональных чисел, например таких, как число 
    \[\pi=3.141592653589793238462643\cdots\] 
    понадобится \emph{бесконечное} количество цифр для представления дробной части в позиционной системе счисления с \emph{любым} целым основанием.
\end{frame}

\begin{frame}
    \frametitle{Потеря точности}
    
    \begin{example}
        $0.5 = (.1)_2$.
        
        $0.3 = (.0[1001])_2 \approx (.010011001\cdots)_2$.
    \end{example}
    
    Число, представимое в одной позиционной СС точно, в ПСС с другим основанием может быть представлено только периодической дробью, а на практике лишь приближённо\ldots
\end{frame}


\begin{frame}
    \frametitle{Вклад $m$-го разряда}
    \framesubtitle{$a_{-m}\neq 0$}
    
    Вклад $m$-го разряда дробной части при $a_{-m}\neq 0$ больше вклада младших\footnote{$0.1>0.099999\cdots$}:
    \[
        a_{-m}\cdot K^{-m}>\sum_{i=m+1}^{\infty}\frac{a_{-i}}{k^i}.
    \]
\end{frame}

\begin{frame}
    \begin{example}[Дробная часть в двоичной СС]
        Записи дробной части $(.10111)_{2}$  соответствует число $Y$:
        \[
            \begin{array}[c]{c}
                Y=
                1\cdot 2^{-1}+
                0\cdot 2^{-2}+
                1\cdot 2^{-3}+
                1\cdot 2^{-4}+
                1\cdot 2^{-5} = \\
                =
                1\cdot 0.5+
                0\cdot 0.25+
                1\cdot 0.125+
                1\cdot 0.0625+
                1\cdot 0.03125 = \\
                = 0.71875
            \end{array}
        \]
    \end{example}
\end{frame}

\begin{frame}
    \frametitle{Представление вещественного числа}
    
    Число
    \[
        X=
        \sum_{i=-m}^{-1}a_i\cdot K^{i} + \sum_{j=0}^{n-1}a_j\cdot K^{j}=
        \sum_{i=-m}^{n-1}a_i\cdot K^{i}.
    \]
    можно представить как
    \[
        X\equiv(a_{n-1}\cdots a_{1}a_{0}\fbox{.}a_{-1}a_{-2}\cdots a_{-m+1}a_{-m})_K.
    \]
    Записи отрицательного числа будет предшествовать знак минус:
    \[
        X\equiv(-a_{n-1}\cdots a_{2}a_{1}a_{0}.a_{-1}a_{-2}\cdots a_{-m+1}a_{-m})_K.
    \]
\end{frame}

\begin{frame}
    \frametitle{Обсуждение}
    
    Как представить число в физической среде?
\end{frame}


\section{Перевод чисел из одной системы счисления в другую}

\begin{frame}
    \frametitle{Перевод чисел}
    \framesubtitle{Условия}
    
    Необходимо число, представленное в $L$-ичной СС, представить в $K$-ичной СС:
    \begin{itemize}
        \item $A\equiv(\pm\cdots a_1a_0.a_{-1}a_{-2}\cdots)_L$;
        \item $A\to B$;
        \item $B\equiv(\pm\cdots b_1b_0.b_{-1}b_{-2}\cdots)_K$;
        \item Вычислитель считает в $M$-ичной СС.
    \end{itemize}
\end{frame}

\begin{frame}
    \frametitle{Перевод числа в систему счисления вычислителя}
    \framesubtitle{Люди привыкли считать в десятичной}
    
    Исходное число
    \[
        A\equiv(\pm a_{n}\cdots a_{0}.a_{-1}\cdots a_{-m})_L
    \]
    достаточно пересчитать в СС вычислителя
    \[
        A=\pm\sum_{i=-m}^{n}a_i\cdot L^{i},
    \]
    представив\footnote{Не представляет сложности} в ней же $L$ и $a_i$.
\end{frame}

\begin{frame}
    \frametitle{Число в $16$-ичной СС}
    \framesubtitle{Данный слайд создан вычислителем, которому удобно считать в десятичной системе}
    
    \begin{block}{}
        Пусть дано число $Z\equiv(-7AFC.4)_{16}$.

         В шестнадцатиричной системе счисления цифры обозначены следующим образом: цифрам от нуля до девяти соответствуют цифры десятичной системы, а далее используются латинские буквы от $A$ до $F$ в алфавитном порядке, которым соответствуют числа от 10 до 15 соответственно:
        \[
            \begin{array}[c]{c}
                Z=-(
                7\cdot 16^{3}+
                A\cdot 16^{2}+
                F\cdot 16^{1}+
                C\cdot 16^{0}+
                4\cdot 16^{-1})=\\
                =-(
                7\cdot 16^{3}+
                10\cdot 16^{2}+
                15\cdot 16^{1}+
                12\cdot 16^{0}+
                4\cdot 16^{-1})=\\
                =
                -(7\cdot 4096+
                10\cdot 256+
                15\cdot 16+
                12\cdot 1+
                4\cdot 0.0625)=\\
                = -31484.25
            \end{array}
        \]    
    \end{block}
\end{frame}


\subsection{Перевод целой части}

\begin{frame}
    \frametitle{Перевод в $K$-ичную СС}
    \framesubtitle{Целая часть}
    
    Допустим, что целая часть ($X$) уже представлена в $K$-ичной системе: $X\equiv(b_n\cdots b_0)_K$. $X$ делится нацело на основание $K$:
    \[
        X=K\cdot\left(\sum_{i=1}^{n}b_i\cdot K^{i-1}\right)+b_{0},
    \]
    получая в остатке $b_{0}\in[0,K-1]$.
    
    \[
        X=K\cdot X^{(1)} + b_{0}.
    \]
\end{frame}

\begin{frame}
    \frametitle{Перевод в $K$-ичную СС}
    \framesubtitle{Целая часть}

    \[
        \begin{array}[c]{l}
            \displaystyle
            X^{(1)}=K\cdot\left(\sum_{i=2}^{n}b_i\cdot K^{i-2}\right)+b_{1} = K\cdot X^{(2)}+b_{1},\\
            \displaystyle
            X^{(2)}=K\cdot X^{(3)}+b_{2},\\
            \displaystyle
            X^{(3)}=K\cdot X^{(4)}+b_{3},\\
            \displaystyle
            \cdots \\
            \displaystyle
            X^{(n)}=K\cdot 0 + b_{n}.
        \end{array}
    \]
\end{frame}


\subsection{Перевод дробной части}

\begin{frame}
    \frametitle{Перевод в $K$-ичную СС}
    \framesubtitle{Дробная часть $Y$, $0\leq Y < 1$}
    
    Представим, что дробная часть ($Y$) числа уже представлена в $K$-ичной системе: $(.b_{-1}\cdots b_{-m})_K$. Умножая дробную часть на $K$.
    \[
        Y\cdot K = 
            K\cdot \left(\sum_{i=-m}^{-1}b_i\cdot K^{i} \right)=
                b_{-1} + \sum_{i=-m}^{-2}b_i\cdot K^{i+1},
    \]
    находим $b_{-1}=\lfloor Y\cdot K\rfloor$ и справедливо $b_{-1}\in[0,K-1]$.
    
    \[Y\cdot K = b_{-1} + Y^{(1)}.\]
\end{frame}

\begin{frame}
    \frametitle{Перевод в $K$-ичную СС}
    \framesubtitle{Дробная часть}
    
    \[
        \begin{array}[c]{l}
            \displaystyle
            Y^{(1)}\cdot K = \left(\sum_{i=-m}^{-2}b_i\cdot K^{i+1} \right)\cdot K=
                b_{-2} + Y^{(2)},\\
            \displaystyle
            Y^{(2)}\cdot K = b_{-3} + Y^{(3)},\\
            \displaystyle
            Y^{(3)}\cdot K = b_{-4} + Y^{(4)},\\
            \cdots \\
            \displaystyle
            Y^{(m-1)}\cdot K = \left(\sum_{i=-m}^{-m}b_i\cdot K^{i+m-1} \right)\cdot K=
                b_{-m} + 0.
        \end{array}
    \]
\end{frame}

\begin{frame}
    \frametitle{Перевод чисел}
    \framesubtitle{Точность представления дробной части $(Y)_L\to(Y)_K$.}
    
    \[(.a_{-1}\cdots a_{-m_L})_L\approx(.b_{-1}\cdots b_{-m_K})_K.\]
    
    Сколько разрядов $m_K$ необходимо?
    \[
        \begin{array}{ccc}
            \Delta_{K}      &\leq& \Delta_{L},\\
            K^{-m_K}        &\leq& L^{-m_L},\\
            \log_K K^{-m_K} &\leq& \log_K L^{-m_L},\\
            -m_K            &\leq& -m_L\cdot\log_K L,\\
        \end{array}
    \]
    
    \alert{$m_k \geq m_L\cdot\log_K L$}.
\end{frame}

\begin{frame}
    \frametitle{$X\equiv-31484.25$ в $3$-ичную СС}
    \framesubtitle{Целая часть}
    
    \[
        \begin{array}[c]{ll}
            31484=10494\cdot 3 + 2, &\Rightarrow b_{0}=2, \\
            10494=3498\cdot 3 + 0,  &\Rightarrow b_{1}=0, \\
            3498=1166\cdot 3 + 0,   &\Rightarrow b_{2}=0, \\
            1166=388\cdot 3 + 2,    &\Rightarrow b_{3}=2, \\
            388=129\cdot 3 + 1,     &\Rightarrow b_{4}=1, \\
            129=43\cdot 3 + 0,      &\Rightarrow b_{5}=0, \\
            43=14\cdot 3 + 1,       &\Rightarrow b_{6}=1, \\
            14=4\cdot 3 + 2,        &\Rightarrow b_{7}=2, \\
            4=1\cdot 3 + 1,         &\Rightarrow b_{8}=1, \\
            1=0\cdot 3 + 1,         &\Rightarrow b_{9}=1
        \end{array}
    \]
    $31484\equiv(1121012002)_3$.
\end{frame}

\begin{frame}
    \frametitle{$X\equiv-31484.25$ в $3$-ичную СС}
    \framesubtitle{Дробная часть}
    
    \[
        \begin{array}[c]{ll}
            0.25\cdot 3=0.75,   &\Rightarrow b_{-1}=0, \\
            0.75\cdot 3=2.25,   &\Rightarrow b_{-2}=2, \\
            0.25\cdot 3=0.75,   &\Rightarrow b_{-3}=0, \\
            0.75\cdot 3=2.25,   &\Rightarrow b_{-4}=2, \\
            \cdots
        \end{array}
    \]
    
    $0.25\equiv (0.[02])_3$.
\end{frame}    

\begin{frame}
    \frametitle{$X\equiv(-7AFC.4)_{16}\equiv-31484.25$ в $3$-ичную СС}

    \[X\equiv(-7AFC.4)_{16}\equiv -31484.25 \equiv (-1121012002.[02])_3\]
\end{frame}    


\section{Двоичная система счисления и пр.}


\subsection{Двоичная система счисления}

\begin{frame}
    \frametitle{Двоичная система счисления}
    
    \begin{itemize}
        \item Система счисления с основанием $2$. 
        \item Цифры всего две: $0$ и $1$. 
        \item Разряд $\equiv$ бит\footnote{bit --- \alert{b}inary dig\alert{it}. Двоичная цифра.}.
    \end{itemize}
    \[
        \begin{tabular}{ll|ll}
            \hline\hline
            10\text{-я CC} 
                & 2\text{-я CC}
                        & 10\text{-я CC} 
                            & 2\text{-я CC}\\
            \hline\hline
            0   &0000   &8  &1000\\
            1   &0001   &9  &1001\\
            2   &0010   &10 &1010\\
            3   &0011   &11 &1011\\
            4   &0100   &12 &1100\\
            5   &0101   &13 &1101\\
            6   &0110   &14 &1110\\
            7   &0111   &15 &1111\\
            \hline
        \end{tabular}
    \]
\end{frame}

Перевод в двоичную систему счисления или из неё не имеет никаких особенностей и полностью соответствует приведённому выше для любой позиционной системы.

\begin{frame}
    \frametitle{Двоичная система счисления}
    \framesubtitle{Таблица сложения}
    
    \[
        \begin{array}[c]{c|c|c|}
            + & 0 & 1 \\
            \hline
            0 & \xleftarrow{0}0 & \xleftarrow{0}1\\
            \hline
            1 & \xleftarrow{0}1 & \xleftarrow{1}0 \\
            \hline
        \end{array}
    \]
    
    Например, $1+1=2 \equiv (10)_2 \equiv \xleftarrow{1}0$.
\end{frame}

\begin{frame}
    \frametitle{Двоичная система счисления}
    \framesubtitle{Сложение чисел}
    
    \begin{example}[Задача]
        Сложить двоичные числа:
        $A\equiv(101.1101)_2$ и
        $B\equiv(11.010111)_2$.
    \end{example}
    \begin{proof}[Решение]
    \[
        {\entrymodifiers={}
            {\xymatrix@=1pc{
                A&\equiv
                    & &1&0&1&.&1&1&0&1&0&0\\
                B&\equiv
                    & &0&1&1&.&0&1&0&1&1&1\\
                \xleftarrow{c}
                 &  &_1&_1&_1&_1& &_1& &_1& & & \\
                A+B&\equiv
                    &1
                      &0\ar[ul]
                        &0\ar[ul]
                          &1\ar[ul]
                            &.
                              &0\ar[ull]
                                &0\ar[ul]
                                  &1
                                    &0\ar[ul]
                                      &1
                                        &1
            }}
        }
    \]
    \end{proof}
\end{frame}

\begin{frame}
    \frametitle{Двоичная система счисления}
    \framesubtitle{Закономерности}
    
    \begin{itemize}
        \item В $n$ двоичных разрядах можно представить 
        \[2^n\] 
        чисел. $[0,2^n-1] \equiv [(\underbrace{0\cdots 0}_n)_2,(\underbrace{1\cdots 1}_n)_2]$
        \item Чтобы пронумеровать $m$ объектов, требуется 
        \[\lceil\log_2m\rceil\]
        разрядное $2$-ичное число.
    \end{itemize}
\end{frame}

    
\subsection{Восьмиричная и шестнадцатиричная СС}

\begin{frame}
    \frametitle{Вспомогательные системы счисления}
    \framesubtitle{$8,16$ СС}
    
    Системы счисления, основание которых есть степень двух: 
    \begin{columns}
        \column{.3\textwidth}
        \begin{itemize}
            \item $8=2^3$;
            \item $16=2^4$.
        \end{itemize}
    \end{columns}
\end{frame}

\begin{frame}
    \frametitle{Восьмиричная СС}
    \framesubtitle{
        \(
        X=(\pm\cdots\fbox{$a_8a_7a_6$}\fbox{$a_5a_4a_3$}\fbox{$a_2a_1a_0$}.
        \fbox{$a_{-1}a_{-2}a_{-3}$}\fbox{$a_{-4}a_{-5}a_{-6}$}\fbox{$a_{-7}a_{-8}a_{-9}$}\cdots)_2
        \)
    }
    
    \[
        X=
            \begin{array}[c]{c}
                \ldots+\\+
                a_{8}\cdot 2^{8} +
                a_{7}\cdot 2^{7} +
                a_{6}\cdot 2^{6} +\\+

                a_{5}\cdot 2^{5} +
                a_{4}\cdot 2^{4} +
                a_{3}\cdot 2^{3} +\\+

                a_{2}\cdot 2^{2} +
                a_{1}\cdot 2^{1} +
                a_{0}\cdot 2^{0} +\\+

                \frac {a_{-1}}{2^{1}} +
                \frac {a_{-2}}{2^{2}} +
                \frac {a_{-3}}{2^{3}} +\\+

                \frac {a_{-4}}{2^{4}} +
                \frac {a_{-5}}{2^{5}} +
                \frac {a_{-6}}{2^{6}} +\\+

                \frac {a_{-7}}{2^{7}} +
                \frac {a_{-8}}{2^{8}} +
                \frac {a_{-9}}{2^{9}} +\\

                +\ldots
        \end{array}
    \]
\end{frame}

\begin{frame}
    \frametitle{Восьмиричная СС}
    \framesubtitle{
        \(
        X=(\pm\cdots\fbox{$a_8a_7a_6$}\fbox{$a_5a_4a_3$}\fbox{$a_2a_1a_0$}.
        \fbox{$a_{-1}a_{-2}a_{-3}$}\fbox{$a_{-4}a_{-5}a_{-6}$}\fbox{$a_{-7}a_{-8}a_{-9}$}\cdots)_2
        \)
    }

    \[
        X=
        \begin{array}[c]{c}
            \ldots+\\+
            \left(
            a_{8}\cdot 2^{2} +
            a_{7}\cdot 2^{1} +
            a_{6}\cdot 2^{0}\right)\cdot 8^{2} +\\+

            \left(
            a_{5}\cdot 2^{2} +
            a_{4}\cdot 2^{1} +
            a_{3}\cdot 2^{0}\right)\cdot 8^{1} +\\+

            \left(
            a_{2}\cdot 2^{2} +
            a_{1}\cdot 2^{1} +
            a_{0}\cdot 2^{0}\right)\cdot 8^{0} +\\+

            \frac{\displaystyle
            (a_{-1}\cdot 2^{2} +
            a_{-2}\cdot 2^{1} +
            a_{-3}\cdot 2^{0})
            }{8^{1}} +\\+

            \frac{\displaystyle
            (a_{-4}\cdot 2^{2} +
            a_{-5}\cdot 2^{1} +
            a_{-6}\cdot 2^{0})
            }{8^{2}} +\\+

            \frac{\displaystyle
            (a_{-7}\cdot 2^{1} +
            a_{-8}\cdot 2^{2} +
            a_{-9}\cdot 2^{3})
            }{8^{3}} +\\

            +\ldots
        \end{array}
    \]
\end{frame}

\begin{frame}
    \frametitle{Восьмиричная СС}
    \framesubtitle{
        \(
        X=(\pm\cdots\fbox{$a_8a_7a_6$}\fbox{$a_5a_4a_3$}\fbox{$a_2a_1a_0$}.
        \fbox{$a_{-1}a_{-2}a_{-3}$}\fbox{$a_{-4}a_{-5}a_{-6}$}\fbox{$a_{-7}a_{-8}a_{-9}$}\cdots)_2
        \)
    }
    
    Получили запись числа:
    \[
        X=\sum_{i=-m'}^{n'}b_{i}\cdot 8^{i},
    \]
    где 
    \[
        b_{i}=\sum_{j=0}^{2}a_{3\cdot i + j}\cdot 2^{j}
    \]
    и $b_i\in [0,2^{3}-1]$, те $b_i\in [0,7]$.
    
    $b_i$ --- \alert{восмиричная} цифра.
\end{frame}


\begin{frame}[fragile]
    \frametitle{Восьмиричные числа в языках программирования}
    
    \begin{itemize}
        \item \verb"С++,java,C#" и т.д.: если справа перед числом записан ноль, то число в восьмиричной системе. \verb"015720" - восьмиричное число (равное 7120). \verb"0189" - ошибка: недопустимы цифры \verb"8" и \verb"9". Без ведущего нуля число считается десятичным;
        
        \item В ассемблере\footnote{masm} после цифр восьмиричного числа пишется латинская буква <<\verb"o">> (octal). \verb"15720o". Ну, а \verb"189o"\ldots
    \end{itemize}
\end{frame}

\begin{frame}
    \frametitle{Шестнадцатиричная система}

    \[
        X=(\pm\cdots\fbox{$a_7a_6a_5a_4$}\fbox{$a_3a_2a_1a_0$}.
        \fbox{$a_{-1}a_{-2}a_{-3}a_{-4}$}\fbox{$a_{-5}a_{-6}a_{-7}a_{-8}$}\cdots)_2
    \]

    \[
        \begin{tabular}{lll|lll}
            \hline\hline
            16\text{-я CC} 
                &10\text{-я CC} 
                        & 2\text{-я CC}
                            & 16\text{-я CC} 
                                & 10\text{-я CC} 
                                    & 2\text{-я CC}\\
            \hline\hline
            0   &0  &0000   &8  &8  &1000\\
            1   &1  &0001   &9  &9  &1001\\
            2   &2  &0010   &A  &10 &1010\\
            3   &3  &0011   &B  &11 &1011\\
            4   &4  &0100   &C  &12 &1100\\
            5   &5  &0101   &D  &13 &1101\\
            6   &6  &0110   &E  &14 &1110\\
            7   &7  &0111   &F  &15 &1111\\
            \hline
        \end{tabular}
    \]
\end{frame}

\begin{frame}[fragile]
    \frametitle{Шестнадцатиричные числа в языках программирования}
    
    \begin{itemize}
        \item \verb"С++,java,C#" и т.д.: если слева от цифр цисла есть префикс <<\verb"0x">>, то число в шестнадцатеричной системе. \verb"0xAF" - шестнадцатеричное число (равное 175). \verb"0x1h" - ошибка: недопустима цифра \verb"h".
        
        \item pascal: если слева от цифр цисла есть префикс <<\verb"$">>, то число в шестнадцатеричной системе. \verb"$AF" - шестнадцатеричное число (равное 175). \verb"$1h" - ошибка: недопустима цифра \verb"h".
        
        \item В некоторых ассемблерах после цифр шестнадцатиричного числа пишется латинская буква <<\verb"h">> (hexadecimal): \verb"1beh", \verb"0afh", \verb"0AFh", \verb"0AFH".
    \end{itemize}
\end{frame}

\begin{frame}
    \frametitle{Вспомогательные системы счисления}
    \framesubtitle{$8,16$ СС}

    \begin{example}[Задача]
        Дано двоичное число $(1110011.0101101)_2$. Перевести его в системы счисления с основанием $8$ и $16$.
    \end{example}
    \begin{proof}<2>[Решение]
        \[
            \begin{array}[c]{c}
                (
                \underbrace{001}_{1}
                \underbrace{110}_{6}
                \underbrace{011}_{3}.
                \underbrace{010}_{2}
                \underbrace{110}_{6}
                \underbrace{100}_{4})_2\equiv(163.264)_8\equiv\\
                \equiv(
                \underbrace{0111}_{7}
                \underbrace{0011}_{3}.
                \underbrace{0101}_{5}
                \underbrace{1010}_{A}
                )_2\equiv(73.5A)_{16}
            \end{array}
        \]
    \end{proof}
\end{frame}

\begin{frame}
    \frametitle{Вспомогательные ($8,16$) системы счисления}
    
    \begin{example}[Задача]
        Дано восьмиричное число $(673245.471)_8$. Перевести его в систему счисления c основанием $16$.
    \end{example}
    \begin{proof}<2>[Решение]
        Переводим в двоичную систему, а из двоичной в шестнадцатеричную:
        \[
            \begin{array}[c]{c}
                (110\ 111\ 011\ 010\ 100\ 101.100\ 111\ 001)_2=\\
                =(0011\ 0111\ 0110\ 1010\ 0101.1001\ 1100\ 1000)_2=\\
                =(376A5.9C8)_{16}
            \end{array}
        \]
    \end{proof}
\end{frame}

\begin{frame}
    \frametitle{Вспомогательные ($8,16$) системы счисления}

    \begin{example}[Задача]
        Дано число $65045.875$. Перевести его в $2$ CC.
    \end{example}
    \begin{proof}<2>[Решение]
        Переводим в шестнадцатеричную систему целую часть:
        \[
            \begin{array}[c]{l}
                65045=4065\cdot 16 + 5,\Rightarrow a_{0}=5, \\
                4065=254\cdot 16 + 1,\Rightarrow a_{1}=1, \\
                254=15\cdot 16 + 14,\Rightarrow a_{2}=E, \\
                15=0\cdot 16 + 15,\Rightarrow a_{3}=F.
            \end{array}
        \]

        Дробную часть:
        \[
            0.875\cdot 16=14.0,\Rightarrow a_{-1}=E.
        \]

        В двоичной системе: $(FE15.E)_{16}=(1111111000010101.1110)_2$.
    \end{proof}
\end{frame}

\begin{frame}
    \frametitle{Удобство представления двоичных чисел в 8 и 16 СС}
    
    \begin{example}[Компактность уменьшает количество ошибок]
        \[
            \begin{array}[c]{c}
                (11111110101000000001011111001101)_2=\\
                =(37650013715)_8=\\
                =(FEA017CD)_{16}
            \end{array}
        \]
        
        В менее короткой записи числа ошибиться сложнее.
    \end{example}
\end{frame}


\subsection{Биты, байты, тетрады,\ldots}

\begin{frame}
    \frametitle{Информатика и 2 СС}
    
    \begin{itemize}
        \item Бит (bit --- binary digit).
        \item Байт (byte).
        \item Октет.
        \item Ниббл, тетрада.
        \item \alert{Килобайт}?
    \end{itemize}
\end{frame}

\begin{frame}
    \frametitle{Война префиксов закончена 19 марта 2005 года!}
    \framesubtitle{IEEE 1541. 1000 байт --- 1 kB (килобайт), 1024 байт --- 1KiB (кибибайт)}
    
    Префиксы для формирования крупных единиц измерения информации
    \[
    \begin{tabular}[c]{ll|ll}
        \hline\hline
        Множитель          & СИ/SI                  & Множитель        & IEEE 1541 \\
        \hline\hline
        $10^3  = 1000^1$   & \emph{kilo} (k) кило   & $2^{10} =1024^1$ & \emph{kibi} (Ki) киби\\
        $10^6  = 1000^2$   & \emph{mega} (M) мега   & $2^{20} =1024^2$ & \emph{mebi} (Mi) меби \\
        $10^9  = 1000^3$   & \emph{giga} (G) гига   & $2^{30} =1024^3$ & \emph{gibi} (Gi) гиби\\
        $10^{12} = 1000^4$ & \emph{tera} (T) тера   & $2^{40} =1024^4$ & \emph{tebi} (Ti) теби\\
        $10^{15} = 1000^5$ & \emph{peta} (P) пета   & $2^{50} =1024^5$ & \emph{pebi} (Pi) пеби\\
        $10^{18} = 1000^6$ & \emph{exa} (E) экса    & $2^{60} =1024^6$ & \emph{exbi} (Ei) эксби\\
        $10^{21} = 1000^7$ & \emph{zetta} (Z) зетта & $2^{70} =1024^7$ & \emph{zebi} (Zi) зеби\\
        $10^{24} = 1000^8$ & \emph{yotta} (Y) йотта & $2^{80} =1024^8$ & \emph{yobi} (Yi) йоби\\
        \hline
    \end{tabular}
    \]
\end{frame}

\begin{frame}
    \frametitle{Обсуждение}
    
    Как представляются числа в современных компьютерах?
\end{frame}


\appendix

\begin{frame}
    \frametitle{В заключение}
    
    Подробнее о ситемах счисления см. \cite{bib:gorbatov:fodm,bib:sudoplatov:discrmath}.
\end{frame}


\begin{frame}[allowframebreaks]{Библиография}
    \bibliographystyle{gost780u}
    \bibliography{./../../bibliobase}
\end{frame}

\end{document}
%include part: see main.beamer.tex and main.article.tex
%include part: see main.beamer.tex and main.article.tex
%include common packages and settings
\usepackage{etex} %эта магическая херь избавляет от переполнения регистров TeX а!!!

\mode<article>{\usepackage{fullpage}}
\mode<presentation>{
    \usetheme{Madrid} %%Boadilla,Madrid,AnnArbor,CambridgeUS,Malmoe,Singapore,Berlin
    \useoutertheme{shadow}
} 

\usepackage[utf8]{inputenc}
\usepackage[russian]{babel}
\usepackage{indentfirst}
\usepackage{graphicx}

\usepackage{amsmath}
\usepackage{amsfonts}
\usepackage{amsthm}
\usepackage{algorithm}
\usepackage{algorithmic}

\usepackage[all]{xy}

\date{Лекция по дисциплине <<дискретная математика>>\\(\today)}
\author[М.~М.~Шихов]{Михаил Шихов \\ \texttt{\underline{m.m.shihov@gmail.com}}}

%для рисования графов пакетом xy-pic
\entrymodifiers={++[o][F-]}

%для псевдокода алгоритмов (algorithm,algorithmic)
\renewcommand{\algorithmicrequire}{\textbf{Вход:}}
\renewcommand{\algorithmicensure}{\textbf{Выход:}}
\renewcommand{\algorithmiccomment}[1]{// #1}
\floatname{algorithm}{Псевдокод}



\title[$P\subseteq A^2$]{Бинарные отношения на множестве $A$}


\begin{document}

%титул и содержание статьи
\mode<article>{\maketitle\tableofcontents}

%титул и содержание презентации
\frame<presentation>{\titlepage}
\begin{frame}<presentation>
    \frametitle{Содержание}
    \tableofcontents
\end{frame}


\section{Свойства бинарных отношений на множестве $A$}


\subsection{Свойства отношений} 

\begin{frame}
    \frametitle{Бинарное отношение}
    
    \begin{definition}
        \alert{Бинарным отношением} $P$ на множестве $A$ называется 
        \[P\subseteq A^2,\]
        где $A^2=A\times A$.
    \end{definition}
\end{frame}

\begin{frame}
    \frametitle{Отношение $P$ на $A$ называется}
    
    \begin{definition}
        \alert{Рефлексивным ($\rho$)}, если для всех $x\in A$ справедливо $(x,x)\in P$. Т.е. $I_A\subseteq P$.
    \end{definition}    
    \[  
        \begin{matrix}
            \begin{array}{c|cccc}
                 &1&2&3&4 \\ \hline
                1&\alert{1}&1&0&0 \\
                2&0&\alert{1}&1&0 \\
                3&0&0&\alert{1}&1 \\
                4&1&0&0&\alert{1}
            \end{array}
            &\raisebox{2\height}{
                {\xymatrix{
                    1 \ar@{->}@(l,u)[] \ar@{:>}[r]
                        &2 \ar@{->}@(u,r)[] \ar@{:>}[d]
                            \\
                    4 \ar@{->}@(d,l)[] \ar@{:>}[u]
                        &3 \ar@{->}@(r,d)[] \ar@{:>}[l]
                }}
            }
        \end{matrix}
    \]
\end{frame}
    
\begin{frame}
    \frametitle{Отношение $P$ на $A$ называется}
    
    \begin{definition}
        \alert{Антарефлексивным}, если для всех $x\in A$ справедливо $(x,x)\not\in P$.
    \end{definition}
    \[  
        \begin{matrix}
            \begin{array}{c|cccc}
                 &1&2&3&4 \\ \hline
                1&\alert{0}&1&0&0 \\
                2&0&\alert{0}&1&0 \\
                3&0&0&\alert{0}&1 \\
                4&1&0&0&\alert{0}
            \end{array}
            &\raisebox{2\height}{
                {\xymatrix{
                    1 \ar@{.>}@(l,u)[]|{\alert{0}} \ar@{:>}[r]
                        &2 \ar@{.>}@(u,r)[]|{\alert{0}} \ar@{:>}[d]
                            \\
                    4 \ar@{.>}@(d,l)[]|{\alert{0}} \ar@{:>}[u]
                        &3 \ar@{.>}@(r,d)[]|{\alert{0}} \ar@{:>}[l]
                }}
            }
        \end{matrix}
    \]
\end{frame}

\begin{frame}
    \frametitle{Отношение $P$ на $A$ называется}
    
    \begin{definition}
        \alert{Симметричным ($\sigma$)}, если для любых $x,y\in A$ из $(x,y)\in P$ следует $(y,x)\in P$. То есть $P^{-1}=P$.
    \end{definition}
    \[  
        \begin{matrix}
            \begin{array}{c|cccc}
                 &1&2&3&4 \\ \hline
                1&1&1&0&1 \\
                2&1&0&1&0 \\
                3&0&1&0&1 \\
                4&1&0&1&0
            \end{array}
            &\raisebox{2\height}{
                {\xymatrix{
                    1 \ar@{->}@(l,u)[] \ar@{->}@/^/[r] \ar@{->}@/^/[d]
                        &2 \ar@{->}@/^/[d] \ar@{->}@/^/[l]
                            \\
                    4 \ar@{->}@/^/[u] \ar@{->}@/^/[r]
                        &3 \ar@{->}@/^/[l] \ar@{->}@/^/[u]
                }}
            }
        \end{matrix}
    \]
\end{frame}    

\begin{frame}
    \frametitle{Отношение $P$ на $A$ называется}
    \begin{definition}
        \alert{Антисимметричным}, если для любых $x,y\in A$ из $(x,y)\in P\land(y,x)\in P$ следует $x=y$. То есть $P\cap P^{-1}\subseteq I_A$.
    \end{definition}
    \[  
        \begin{matrix}
            \begin{array}{c|cccc}
                 &1&2&3&4 \\ \hline
                1&1&1&0&0 \\
                2&0&0&1&0 \\
                3&0&0&1&1 \\
                4&1&0&0&0
            \end{array}
            &\raisebox{2\height}{
                {\xymatrix{
                    1 \ar@{->}@(l,u)[] \ar@{->}[r] 
                        &2 \ar@{->}[d]
                            \\
                    4 \ar@{->}[u]
                        &3 \ar@{->}[l] \ar@{->}@(r,d)[]
                }}
            }
        \end{matrix}
    \]    
\end{frame}


\begin{frame}
    \frametitle{Отношение $P$ на $A$ называется}
    
    \begin{definition}
        \alert{Транзитивным ($\eta$)}, если для любых $x,y\in A$ из $(x,y)\in P\land (y,z)\in P$ следует $(x,z)\in P$. То есть $P\cdot P\subseteq P$.
    \end{definition}
    \[  
        \begin{matrix}
            \begin{array}{c|cccc}
                 &1&2&3&4 \\ \hline
                1&0&1&0&0 \\
                2&0&0&0&0 \\
                3&0&1&0&0 \\
                4&1&1&1&0
            \end{array}
            &\raisebox{2\height}{
                {\xymatrix{
                    1 \ar@{->}[r] 
                        &2 
                            \\
                    4 \ar@{->}[u] \ar@{->}[r] \ar@{->}[ur]
                        &3 \ar@{->}[u]
                }}
            }
        \end{matrix}
    \]    
\end{frame}


\begin{frame}
    \frametitle{Отношение $P$ на $A$ называется}
    
    \begin{definition}
        \alert{Полным} или \alert{линейным}, если для любых $x,y\in A$ из $x\neq y$ следует $(x,y)\in P\lor (y,x)\in P$. То есть $I_A\cup P\cup P^{-1}=A^2$.
    \end{definition}
    \[  
        \begin{matrix}
            \begin{array}{c|cccc}
                 &1&2&3&4 \\ \hline
                1&0&1&0&0 \\
                2&0&1&1&0 \\
                3&1&0&0&1 \\
                4&1&1&0&0
            \end{array}
            &\raisebox{2\height}{
                {\xymatrix{
                    1 \ar@{->}[r] 
                        &2 \ar@{->}[d] \ar@{->}@(u,r)[]
                            \\
                    4 \ar@{->}[u] \ar@{->}[ur] 
                        &3 \ar@{->}[l] \ar@{->}[ul] 
                }}
            }
        \end{matrix}
    \]    
\end{frame}


\subsection{Замыкание относительно свойства} 

\begin{frame}
    \frametitle{Замыкание}
    
    Если $P\subseteq A^2$ не обладает тем или иным свойством $S$, то его можно дополнить упорядоченными парами из $A^2$ до отношения $P^*$, которое обладает нужным свойством. 

    \begin{definition}
        Отношение $P^*$ называется \alert{замыканием} отношения $P$ \alert{относительно свойства} $S$, если оно:
        \begin{enumerate}
            \item обладает нужным свойством $S$, т.е. справедливо $S(P^*)$;
            \item содержит $P$, т.е. $P^*\supset P$;
            \item является подмножеством любого другого отношения $P'$, содержащего $P$ и обладающего свойством $S$, т.е. справедливо $S(P')\land (P'\supset P)\Rightarrow P^*\subset P'$.
        \end{enumerate}
    \end{definition}
\end{frame}

\begin{frame}
    \frametitle{Примеры замыканий на практике}
    
    \begin{itemize}
        \item если задан граф коммуникационной сети (отражающий отношение между парой узлов <<существует линия связи>>), то найденное замыкание относительно транзитивности даст ответ на вопрос: \uncover<2>{существует ли возможность передать сообщение из одного узла в другой.}
        
        \item если на группе людей задано бинарное отношение <<родитель>>, то замыкание относительно транзитивности даст отношение\uncover<2>{<<предок>>.}
    \end{itemize}
\end{frame}

\begin{frame}
    \frametitle{Нахождение замыканий}
    
    \begin{example}
        На множестве $\{1,2,3,4\}$ задано бинарное отношение
        \[
            P=\{(1,2),(2,4),(3,2),(4,1),(4,3),(4,4)\}.
        \]
        Найти замыкания относительно рефлексивности ($\rho$), симметричности ($\sigma$) и транзитивности ($\eta$).
    \end{example}
    \uncover<2>{
        \[  
            \begin{matrix}
                \begin{array}{c|cccc}
                     &1&2&3&4 \\ \hline
                    1&0&1&0&0 \\
                    2&0&0&0&1 \\
                    3&0&1&0&0 \\
                    4&1&0&1&1
                \end{array}
                &
                \raisebox{2\height}{
                    {\xymatrix{
                        1 \ar@{->}[r] 
                            &2 \ar@{->}[dl]
                                \\
                        4 \ar@{->}[r] \ar@{->}@(d,l)[] \ar@{->}[u]
                            &3 \ar@{->}[u]
                    }}
                }
            \end{matrix}
        \]    
    }
\end{frame}

\begin{frame}
    \frametitle{Нахождение замыканий}
    
    \[  
        \begin{matrix}
            \begin{array}{c|cccc}
                 &1&2&3&4 \\ \hline
                1&0&1&0&0 \\
                2&0&0&0&1 \\
                3&0&1&0&0 \\
                4&1&0&1&1
            \end{array}
            &
            \raisebox{2\height}{
                {\xymatrix{
                    1 \ar@{->}[r] 
                        &2 \ar@{->}[dl]
                            \\
                    4 \ar@{->}[r] \ar@{->}@(d,l)[] \ar@{->}[u]
                        &3 \ar@{->}[u]
                }}
            }
        \end{matrix}
    \]
    \only<1>{
        Относительно рефлексивности ($\rho$):
        \[  
            \begin{matrix}
                \begin{array}{c|cccc}
                     &1&2&3&4 \\ \hline
                    1&\alert{1}&1&0&0 \\
                    2&0&\alert{1}&0&1 \\
                    3&0&1&\alert{1}&0 \\
                    4&1&0&1&1
                \end{array}
                &
                \raisebox{2\height}{
                    {\xymatrix{
                        1 \ar@{->}[r] \ar@{.>}@(l,u)[]
                            &2 \ar@{->}[dl] \ar@{.>}@(u,r)[]
                                \\
                        4 \ar@{->}[r] \ar@{->}[u] \ar@{->}@(d,l)[] 
                            &3 \ar@{->}[u] \ar@{.>}@(r,d)[]
                    }}
                }
            \end{matrix}
        \]
    }
    \only<2>{
        Относительно симметричности ($\sigma$):
        \[  
            \begin{matrix}
                \begin{array}{c|cccc}
                     &1&2&3&4 \\ \hline
                    1&0&1&0&\alert{1} \\
                    2&\alert{1}&0&\alert{1}&1 \\
                    3&0&1&0&\alert{1} \\
                    4&1&\alert{1}&1&1
                \end{array}
                &
                \raisebox{2\height}{
                    {\xymatrix{
                        1 \ar@{->}[r] \ar@{<.}@/^/[r]
                            &2 \ar@{->}[dl] \ar@{<.}@/^/[dl]
                                \\
                        4 \ar@{->}[r] \ar@{<.}@/_/[r] 
                          \ar@{->}[u] \ar@{<.}@/^/[u] \ar@{->}@(d,l)[]
                            &3 \ar@{->}[u] \ar@{<.}@/_/[u] 
                    }}
                }
            \end{matrix}
        \]
    }
    \only<3>{
        Относительно транзитивности ($\eta$):
        \[  
            \begin{matrix}
                \begin{array}{c|cccc}
                     &1&2&3&4 \\ \hline
                    1&\alert{1}&1&\alert{1}&\alert{1} \\
                    2&\alert{1}&\alert{1}&\alert{1}&1 \\
                    3&\alert{1}&1&\alert{1}&\alert{1} \\
                    4&1&\alert{1}&1&1
                \end{array}
                &
                \raisebox{2\height}{
                    {\xymatrix{
                        1 \ar@{->}[r] \ar@{<.}@/^/[r] \ar@{.>}@(l,u)[]
                            &2 \ar@{->}[dl] \ar@{<.}@/^/[dl] \ar@{.>}@(u,r)[]
                                \\
                        4 \ar@{->}[r] \ar@{<.}@/_/[r] 
                          \ar@{->}[u] \ar@{<.}@/^/[u] \ar@{->}@(d,l)[]
                            &3 \ar@{->}[u] \ar@{<.}@/_/[u] \ar@{.>}@(r,d)[]
                               \ar@{.>}@/_/[ul] \ar@{<.}@/^/[ul]
                    }}
                }
            \end{matrix}
        \]
    }
\end{frame}

Обоснование данного алгоритма на соответствующем отношению $P$ графе следующее. Первой итерацией внешнего цикла ($k=1$) к исходному графу будут добавлены транзитивные дуги, замыкающие путь через $a_1$ ($P\subseteq A^2, a_i\in A$). $k$-й итерацией будут добавлены транзитивные дуги, замыкающие путь через <<пройденные транзитом>> вершины $a_1,\ldots,a_k$. Последней итерацией $k=n$ будут добавлены дуги, проходящие транзитом через \emph{любую} последовательность из $n$ вершин графа.

\begin{frame}
    \begin{algorithm}[H]
        \caption{Поиск транзитивного замыкания $P$ (алгоритм Уоршалла)}
        \begin{algorithmic}[1]
            \REQUIRE{$[P]_{n\times n}$ --- матрица смежности отношения $P$}
            \ENSURE{$[T]_{n\times n}$ --- матрица транзитивного замыкания отношения $P$}
            \STATE{$[T]\gets [P]$}
            \FOR{$k=1$ to $n$}
                \FOR{$i=1$ to $n$}
                    \FOR{$j=1$ to $n$}
                        \STATE{$[T]_{i,j}\gets\big([T]_{i,j}\lor ([T]_{i,k}\land[T]_{k,j})\big)$}
                    \ENDFOR
                \ENDFOR
            \ENDFOR
            \RETURN{$[T]$}
        \end{algorithmic}
    \end{algorithm}
\end{frame}
    
\begin{frame}
    \frametitle{Близость отношения к свойству}
    
    \begin{definition}
        Близость $\Delta(P,S)$ бинарного отношения $P$ к некоторому свойству $S$ можно оценивать количеством \alert{добавленных} или \alert{удалённых} пар.
    \end{definition}
    \begin{block}{}
        \begin{center}
            \begin{tabular}{cccc}
                {\xymatrix{
                    a \ar@{->}[d] \ar@{->}[r]
                        &d \ar@{->}[d]
                            \\
                    b \ar@{->}[r]
                        &c
                }}
                    &
                    {\xymatrix{
                        a \ar@{->}[d] \ar@{->}[r] \ar@{.>}@(l,u)[]
                            &d \ar@{->}[d] \ar@{.>}@(u,r)[]
                                \\
                        b \ar@{->}[r] \ar@{.>}@(d,l)[]
                            &c \ar@{.>}@(r,d)[]
                    }}
                        &
                        {\xymatrix{
                            a \ar@{->}[d] \ar@{->}[r]
                                &d \ar@{->}[d] \ar@{.>}@/_/[l] 
                                    \\
                            b \ar@{->}[r] \ar@{.>}@/^/[u] 
                                &c \ar@{.>}@/^/[l] \ar@{.>}@/_/[u] 
                        }}
                            &
                            {\xymatrix{
                                a \ar@{->}[d] \ar@{->}[r] \ar@{.>}[dr]
                                    &d \ar@{->}[d]
                                        \\
                                b \ar@{->}[r]
                                    &c
                            }}
                        \\
                        &&&\\
                    Исходное $P$ 
                        & Рефлекс-е($\rho$) 
                            & Симметр-е($\sigma$) 
                                & Транзит-е($\eta$)\\
                        & $\Delta(P,\rho)=4$ 
                            & $\Delta(P,\sigma)=4$ 
                                & $\Delta(P,\eta)=1$ 
            \end{tabular}
        \end{center}
    \end{block}
\end{frame}

   
\section{Специальные бинарные отношения на множестве $A$}


\subsection{Отношение эквивалентности}
 
\begin{frame}
    \frametitle{Отношение эквивалентности}
    
    \begin{definition}
        Бинарное отношение на множестве $A$, обладающее свойствами рефлексивности ($\rho$), симметричности ($\sigma$) и транзитивности ($\eta$), называется отношением \alert{эквивалентности}
    \end{definition}
    
    Обозначается обычно $E$, $\sim$ или $\equiv$.
\end{frame}

\begin{frame}
    \frametitle{Примеры отношения эквивалентности}
    ?
\end{frame}

\begin{frame}
    \frametitle{Класс эквивалентности и фактор-множество}
    
    \begin{definition}
        \alert{Классом эквивалентности} $E(x)$ элемента $x\in A$ называется множество всех элементов $y\in A$, каждый из которых находится в отношении эквивалентности $E$ с элементом $x$:
        \[E(x)=\{y | x\,E\,y, x\in A, y\in A\}.\]
    \end{definition}

    \begin{definition}
        Множество, обозначаемое $A/E$:
        \[
            A/E=\{E(x)|x\in A\}
        \]
        называется \alert{фактор-множеством} множества $A$ по отношению эквивалентности $E$.
    \end{definition}
\end{frame}

\begin{frame}
    \frametitle{Фактор-множество и разбиение}
    
    \begin{theorem}
        Всякое отношение эквивалентности $E$ на множестве $A$ определяет \alert{разбиение}, которым является \alert{фактор-множество} $A/E$. И обратно: всякое разбиение 
        \[
            \mathcal{R}=\{A_i|i\in \mathbb{N}, A_i\subseteq A,(i\neq j)\Rightarrow (A_i\cap A_j=\emptyset)\},A=\bigcup_{i\in\mathbb{N}} A_i
        \]
        множества $A$, не содержащее пустых элементов, определяет отношение эквивалентности $E$ на $A$ по правилу 
        \begin{equation}
            x\,E\,y\Leftrightarrow x,y\in A_i.
            \label{eq:bo:equivByR}
        \end{equation}
    \end{theorem}
\end{frame}

\begin{frame}
    \frametitle{Фактор-множество и разбиение}
    
    \begin{proof}
        Так как $E$ рефлексивно, то $x\in E(x)$ для любого $x\in A$. $\Rightarrow$ $A/E\neq\emptyset$ и $\bigcup_{x\in A}E(x)=A$. Чтобы доказать, что $A/E$ --- разбиение достаточно доказать, что если $E(x)\cap E(y)\neq\emptyset$, то $E(x)=E(y)$.
        
        Покажем, что $E(x)\subseteq E(y)$ и $E(y)\subseteq E(x)$ при $E(x)\cap E(y)\neq\emptyset$. Пусть $z\in E(x)\cap E(y)$. Докажем $E(x)\subseteq E(y)$. Возьмем произвольный $k\in E(x)$, тогда справедливо $k\,E\,z$, $z\,E\,y$ и, следовательно, $k\,E\,y$. Если $k\,E\,y$, то $k\in E(y)$. Стало быть $k\in E(x)\Rightarrow k\in E(y)$, а значит $E(x)\subseteq E(y)$. Аналогично докажем, что $E(y)\subseteq E(x)$.\qed
        
        Теперь докажем обратное утверждение теоремы. Пусть имеется разбиение $\mathcal{R}=\{A_i\}$. Рефлексивность и симметричность $E$, определяемого формулой \eqref{eq:bo:equivByR} очевидны. Докажем транзитивность. $x\,E\,y$ справедливо при $x,y\in A_i$, $y\,E\,z$ --- при $y,z\in A_j$. Но раз $y\in A_i$ и $y\in A_j$, то $A_i=A_j$. Тогда справедливо $x,z\in A_i$ и $x\,E\,z$.
    \end{proof}
\end{frame}

\begin{frame}
    \frametitle{Матрица смежности отношения эквивалентности}
    
    В любом классе $E(x)$ эквивалентности $E$ каждый элемент $y\in E(x)$ связан отношением $E$ с любым $z\in E(x)$. Поэтому, если 
    \[
        A/E=\{\{a^1_1,\ldots,a^1_{m_1}\},\{a^2_1,\ldots,a^2_{m_2}\},\ldots,\{a^n_1,\ldots,a^n_{m_n}\}\}
    \]
    и элементы множества $A$ упорядочены так:
    \[
        a^1_1,\ldots,a^1_{m_1},a^2_1,\ldots,a^2_{m_2},\ldots,a^n_1,\ldots,a^n_{m_n},
    \]
    то матрица смежности отношения $E$ имеет блочно-диагональный вид:
\end{frame}

\begin{frame}
    \frametitle{Матрица смежности отношения эквивалентности}
    
    \[[E]=
        \begin{array}{c|cccc}
            E
                &a^1_1\cdots a^1_{m_1} 
                    & a^2_1\cdots a^2_{m_2} 
                        & \cdots 
                            & a^n_1\cdots a^n_{m_n}
                                \\ 
            \hline
            \begin{matrix}a^1_1\\ \vdots \\ a^1_{m_1}\end{matrix} 
                &\begin{array}{|ccc|}\hline 1&\cdots&1\\ \vdots & \ddots & \\ 1 & & 1\\ \hline\end{array}
                    &
                        &
                            &
                                \\
            \begin{matrix}a^2_1\\ \vdots \\ a^2_{m_2}\end{matrix} 
                &
                    &\begin{array}{|ccc|}\hline 1&\cdots&1\\ \vdots & \ddots & \\ 1 & & 1\\ \hline\end{array}
                        &
                            & 
                                \\
            \vdots
                &
                    &
                        &\ddots
                            & 
                                \\
            \begin{matrix}a^n_1\\ \vdots \\ a^n_{m_n}\end{matrix} 
                &
                    &
                        &
                            &\begin{array}{|ccc|}\hline 1&\cdots&1\\ \vdots & \ddots & \\ 1 & & 1\\ \hline\end{array}
        \end{array},
    \]
\end{frame}

Если множество $A$ таким образом не упорядочено, то соответствующая матрица смежности $[E]$ приводится к блочно-диагональному виду одновременными перестановками строк и столбцов. Элементы $a_i$ и $a_j$ эквивалентны тогда и только тогда, когда $i$-я и $j$-я строки (а также столбцы) матрицы $[E]$ совпадают. Класс эквивалентности $E(a_i)$ состоит из элементов $a_j$, для которых $[E]_{ij}=1$.

\begin{example}
    Задача. Пусть имеется, например, численный рассчет, представленный орграфом на рисунке \ref{fig:bo:calcFlowEx}. Исходным данным соответствует $S$, конечному результату --- $R$. Вершинам графа соответствуют операции $O_i$. Дуге, соединяющей операции $O_i$ и $O_j$ соответствует численный результат $r_i$ полученный на выходе операции $O_i$ и подаваемый на вход операции $O_j$. Так как исходные данные (или их часть) для операции $O_j$ вычисляются операцией $O_i$, то $O_j$ всегда выполняется во времени \emph{позже} $O_i$. Операции, представленные на графе на одной вертикали, в общем случае могут быть выполнены параллельно. Необходимо минимизировать затраты памяти для хранения промежуточных результатов $r_i$, предполагая, что все они занимают одинаковый объем.
\end{example}

\begin{frame}
    \frametitle{Использование отношения эквивалентности на практике}
    \frametitle{Граф вычислений}
    
    \[
        {\xymatrix{
            *{}
                &O_1\ar@{->}[rr]^{r_1}
                    &*{}
                        &O_5 \ar@{->}[r]^{r_5}
                            &O_7 \ar@{->}[dr]^{r_7}
                                &*{}
                                    &*{}
                    \\
            *{S} \ar@{-->}[ur] \ar@{-->}[dr] \ar@{-->}[rr]
                &*{}
                    &O_3 \ar@{->}[r]^{r_3} \ar@{->}[ur]^{r_3}
                        &O_6 \ar@{->}[rr]^{r_6}
                            &*{}
                                &O_8 \ar@{=>}[r]
                                    &*{R}
                    \\
            *{}
                &O_2 \ar@{->}[r]^{r_2} \ar@{->}[ur]^{r_2}
                    &O_4 \ar@{->}[ur]^{r_4} \ar@{->}[urrr]^{r_4}
                        &*{}
                            &*{}
                                &*{}
                                    &*{}
                    \\
        }}
    \]
    
    Дать словесное описание отношения, которое задает граф вычислений.
\end{frame}

Решая задачу \emph{грубой силой и невежеством}\footnote{Сокращенно: ГСН-алгоритм.}, можно для хранения каждого промежуточного результата $r_i$ использовать отдельную ячейку $m_i$. Кроме того, для сохранения исходных данных и окончательного результата нужно еще две ячейки: $m_S$, $m_R$. См. не оптимизированный по памяти вариант программы в таблице \ref{table:bo:calcFlowProgramEx}. Итого 9 ячеек.

\begin{frame}
    \frametitle{Использование отношения эквивалентности на практике}
    \framesubtitle{ГСН программа. Под каждый результат --- отдельная ячейка памяти}
    
    \begin{center}
        \begin{tabular}{l}
            \hline\hline
            До оптимизации --- 9 я.п.      \\
            \hline\hline
            $S \to m_S$                    \\ \hline
            $O_1(m_S)\to r_1 \to m_1$      \\
            $O_2(m_S)\to r_2 \to m_2$      \\ \hline
            $O_3(m_S,m_2)\to r_3 \to m_3$  \\
            $O_4(m_2)\to r_4 \to m_4$      \\ \hline
            $O_5(m_1,m_3)\to r_5 \to m_5$  \\
            $O_6(m_3,m_4)\to r_6 \to m_6$  \\ \hline
            $O_7(m_5)\to r_7 \to m_7$      \\ \hline
            $O_8(m_4,m_6,m_7)\to R \to m_R$\\ \hline
        \end{tabular}    
    \end{center}
\end{frame}

Результат можно сделать лучше, заметив, что времена жизни некоторых результатов не пересекаются во времени! Это значит, что для них можно использовать одну ячейку. 

\begin{frame}
    \frametitle{Использование отношения эквивалентности на практике}
    \framesubtitle{Оптимизация расхода памяти. Отношение $\not\perp$}
    
    Например, использовать для хранения $r_1$ и $r_2$ одну ячейку нельзя, а вот $r_1$ и $r_5$ можно: $r_1$ после вычисления $r_5$ уже не нужен. 
    
    Введем отношение $\not\perp$, означающее: <<времена жизни не пересекаются>>. Это отношение, очевидно:
    \begin{itemize}
        \item симметрично $(r_i\not\perp r_j)\Rightarrow (r_j\not\perp r_i)$; 
        \item рефлексивно $r_i\not\perp r_i$.
    \end{itemize}
    Но не транзитивно: например, $r_1\not\perp r_5$ и $r_5\not\perp r_3$, но $r_1$ и $r_3$ в отношении $\not\perp$ не находятся. 
    
    Тем не менее, указанное отношение может содержать отношение эквивалентности.
\end{frame}

\begin{frame}
    \frametitle{Использование отношения эквивалентности на практике}
    \framesubtitle{Оптимизация расхода памяти. Матрица смежности $\not\perp$}
    
    \[
        [\not\perp] = 
        \begin{array}{c|ccccccccc}
            \not\perp
                & S &r_1&r_2&r_3&r_4&r_5&r_6&r_7& R\\ \hline
            S   & 1 & 0 & 0 & 1 & 1 & 1 & 1 & 1 & 1\\
            r_1 & 0 & 1 & 0 & 0 & 0 & 1 & 1 & 1 & 1\\
            r_2 & 0 & 0 & 1 & 1 & 1 & 1 & 1 & 1 & 1\\
            r_3 & 1 & 0 & 1 & 1 & 0 & 1 & 1 & 1 & 1\\
            r_4 & 1 & 0 & 1 & 0 & 1 & 0 & 0 & 0 & 1\\
            r_5 & 1 & 1 & 1 & 1 & 0 & 1 & 0 & 1 & 1\\
            r_6 & 1 & 1 & 1 & 1 & 0 & 0 & 1 & 0 & 1\\
            r_7 & 1 & 1 & 1 & 1 & 0 & 1 & 0 & 1 & 1\\
            R   & 1 & 1 & 1 & 1 & 1 & 1 & 1 & 1 & 1
        \end{array}
    \]
\end{frame}

\begin{frame}
    \frametitle{Использование отношения эквивалентности на практике}
    \framesubtitle{Оптимизация расхода памяти. Перестановка в матрице смежности $\not\perp$}
    
    \[
        [\not\perp] = 
        \begin{array}{c|ccccccccc}
            \not\perp
                & R &r_1&r_5&r_7&r_3&r_6& S &r_2&r_4\\\hline
            R   & \alert<1>{1} & \alert<1>{1} & \alert<1>{1} & \alert<1>{1} & 1 & 1 & 1 & 1 & 1\\
            r_1 & \alert<1>{1} & \alert<1>{1} & \alert<1>{1} & \alert<1>{1} & 0 & 1 & 0 & 0 & 0\\
            r_5 & \alert<1>{1} & \alert<1>{1} & \alert<1>{1} & \alert<1>{1} & 1 & 0 & 1 & 1 & 0\\
            r_7 & \alert<1>{1} & \alert<1>{1} & \alert<1>{1} & \alert<1>{1} & 1 & 0 & 1 & 1 & 0\\
            r_3 & 1 & 0 & 1 & 1 & \alert<1>{1} & \alert<1>{1} & \alert<1>{1} & 1 & 0\\
            r_6 & 1 & 1 & 0 & 0 & \alert<1>{1} & \alert<1>{1} & \alert<1>{1} & 1 & 0\\
            S   & 1 & 0 & 1 & 1 & \alert<1>{1} & \alert<1>{1} & \alert<1>{1} & 0 & 1\\
            r_2 & 1 & 0 & 1 & 1 & 1 & 1 & 0 & \alert<1>{1} & \alert<1>{1}\\
            r_4 & 1 & 0 & 0 & 0 & 0 & 0 & 1 & \alert<1>{1} & \alert<1>{1}
        \end{array}
    \]
\end{frame}

\begin{frame}
    \frametitle{Использование отношения эквивалентности на практике}
    \framesubtitle{Оптимизация расхода памяти. Удаление лишних 1 в матрице смежности $\not\perp$}
    
    \[
        [\not\perp']=
        \begin{array}{c|ccccccccc}
            \not\perp'
                & R &r_1&r_5&r_7&r_3&r_6& S &r_2&r_4\\\hline
            R   & 1 & 1 & 1 & 1 & 0 & 0 & 0 & 0 & 0\\
            r_1 & 1 & 1 & 1 & 1 & 0 & 0 & 0 & 0 & 0\\
            r_5 & 1 & 1 & 1 & 1 & 0 & 0 & 0 & 0 & 0\\
            r_7 & 1 & 1 & 1 & 1 & 0 & 0 & 0 & 0 & 0\\
            r_3 & 0 & 0 & 0 & 0 & 1 & 1 & 1 & 0 & 0\\
            r_6 & 0 & 0 & 0 & 0 & 1 & 1 & 1 & 0 & 0\\
            S   & 0 & 0 & 0 & 0 & 1 & 1 & 1 & 0 & 0\\
            r_2 & 0 & 0 & 0 & 0 & 0 & 0 & 0 & 1 & 1\\
            r_4 & 0 & 0 & 0 & 0 & 0 & 0 & 0 & 1 & 1
        \end{array}
    \]
    Три класса эквивалентности $\{R,r_1,r_5,r_7\}$, $\{S,r_3,r_6\}$, $\{r_2,r_4\}$
\end{frame}

\begin{frame}
    \frametitle{Использование отношения эквивалентности на практике}
    \framesubtitle{<<Жадная>> программа. Экономия памяти}

    Элементы одного класса можно хранить в одной ячейке памяти:
    \[
        \{R,r_1,r_5,r_7\}\mapsto m_1, \{S,r_3,r_6\}\mapsto m_2, \{r_2,r_4\}\mapsto m_3
    \]
    \begin{center}
        \begin{tabular}{l|l}
            \hline\hline
            До оптимизации --- 9 я.п.      & После оптимизации --- 3 я.п.   \\
            \hline\hline
            $S \to m_S$                    &  $S\to m_2$                    \\ \hline
            $O_1(m_S)\to r_1 \to m_1$      &  $O_1(m_2)\to r_1 \to m_1$     \\
            $O_2(m_S)\to r_2 \to m_2$      &  $O_2(m_2)\to r_2 \to m_3$     \\ \hline
            $O_3(m_S,m_2)\to r_3 \to m_3$  &  $O_3(m_2,m_3)\to r_3 \to m_2$ \\
            $O_4(m_2)\to r_4 \to m_4$      &  $O_4(m_3)\to r_4 \to m_3$     \\ \hline
            $O_5(m_1,m_3)\to r_5 \to m_5$  &  $O_5(m_1,m_2)\to r_5 \to m_1$ \\
            $O_6(m_3,m_4)\to r_6 \to m_6$  &  $O_6(m_2,m_3)\to r_6 \to m_2$ \\ \hline
            $O_7(m_5)\to r_7 \to m_7$      &  $O_7(m_1)\to r_7 \to m_1$     \\ \hline
            $O_8(m_4,m_6,m_7)\to R \to m_R$&  $O_8(m_3,m_2,m_1)\to R\to m_1$\\ \hline
        \end{tabular}
    \end{center}
\end{frame}    


\subsection{Отношение порядка}

\begin{frame}
    \frametitle{Отношение порядка}
    
    \begin{itemize}
        \item Выделяют несколько видов отношений порядка. 
    
        \item В общем случае отношение порядка обозначается \[\prec\] когда неважно, о каком его виде идет речь. 
    
        \item Обратное к отношению порядка отношение также отношение порядка. $\prec^{-1}$ обозначается $\succ$.
        
        \item Если на множестве $A$ задано некоторое отношение порядка $\prec$, то это обозначается
        \[
            \langle A,\prec\rangle
        \]
    \end{itemize}
\end{frame}


\begin{frame}
    \frametitle{Отношение частичного порядка}
    
    \begin{definition}
        Отношение \alert{частичного} порядка рефлексивно, транзитивно и антисимметрично. 
    \end{definition}
    
    Обозначается символом $\leq$, обратное отношение $\geq$. 
    
    \begin{definition}
        Множество $A$ над элементами которого задано отношение частичного порядка, называется \alert{частично упорядоченным множеством}. 
    \end{definition}
    
    Пример?
\end{frame}    

\begin{frame}
    \frametitle{Отношение строгого порядка}
    
    \begin{definition}
        Отношение \alert{строгого} порядка антирефлексивно транзитивно и антисимметрично. 
    \end{definition}
    
    Обозначается символом $<$, обратное отношение символом $>$.
    
    Пример?
\end{frame}    
    
\begin{frame}
    \frametitle{Отношение линейного порядка}
    
    \begin{definition}
        Отношение \alert{линейного} порядка, представляет собой отношение частичного порядка, в котором отсутствуют несравнимые элементы. То есть для любых $x,y$ справедливо, что $x\leq y$ или $y\leq x$. Отношение рефлексивно, транзитивно, антисимметрично и \alert{полно} (линейно). 
    \end{definition}
    
    Множество $A$, над элементами которого задано отношение линейного порядка, называется \alert{линейно упорядоченным множеством}. 
    
    Пример?
\end{frame}    

\begin{frame}
    \frametitle{Примеры отношений порядка}

    \begin{center}
        \begin{tabular}{ccc}
            {\xymatrix@=.7pc{
                *{}
                    &\{1,2\} \ar@{->}@(ul,ur)[]
                        &*{}
                            \\
                \{1\}  \ar@{->}@(dl,ul)[] \ar@{->}[ur]
                    &*{} 
                        &\{2\} \ar@{->}@(ur,dr)[] \ar@{->}[ul]
                            \\
                *{} 
                    &\emptyset \ar@{->}@(dr,dl)[] \ar@{->}[ul] \ar@{->}[ur] \ar@{->}[uu]
                        &*{} 
            }}
                &
                {\xymatrix@=.7pc{
                    *{}
                        &4 
                            &*{}
                                \\
                    2  \ar@{->}[ur] \ar@{->}[rr] 
                        &*{} 
                            &3 \ar@{->}[ul]
                                \\
                    *{} 
                        &1 \ar@{->}[ul] \ar@{->}[ur] \ar@{->}[uu]
                            &*{} 
                }}
                    &
                    {\xymatrix@=.7pc{
                        *{}
                            &4 \ar@{->}@(ul,ur)[]
                                &*{}
                                    \\
                        2  \ar@{->}@(dl,ul)[] \ar@{->}[ur] \ar@{->}[rr] 
                            &*{} 
                                &3 \ar@{->}@(ur,dr)[] \ar@{->}[ul]
                                    \\
                        *{} 
                            &1 \ar@{->}@(dr,dl)[] \ar@{->}[ul] \ar@{->}[ur] \ar@{->}[uu]
                                &*{} 
                    }}
                        \\
            &&\\
            $\subseteq$ на $2^{\{1,2\}}$
                &$<$ на $\{1,2,3,4\}$
                    & $\leq$ на $\{1,2,3,4\}$
                        \\
            Частичный порядок
                &Строгий порядок
                    &Линейный порядок
        \end{tabular}
    \end{center}
\end{frame}    

\begin{frame}    
    \frametitle{Минимальный и максимальный элементы}
    
    \begin{definition}
        Элемент $a\in A$ частично упорядоченного множества $A$ называется \alert{минимальным}, если для всех $x\in A$ из $x\leq a$ следует $a=x$.
    \end{definition}
    
    \begin{definition}
        Элемент $a\in A$ частично упорядоченного множества $A$ называется \alert{максимальным}, если для всех $x\in A$ из $a\leq x$ следует $a=x$. 
    \end{definition}

    Минимальных (максимальных) элементов в частично упорядоченном множестве может быть несколько и их \alert{не может не быть}.
\end{frame}

\begin{frame}
    \frametitle{Наименьший и наибольший элементы}
    
    \begin{definition}
        Элемент $a\in A$ частично упорядоченного множества $A$ называется \alert{наименьшим}, если для всех $x\in A$ справедливо $a\leq x$. 
    \end{definition}
    
    \begin{definition}
        Элемент $a\in A$ частично упорядоченного множества $A$ называется \alert{наибольшим}, если для всех $x\in A$ справедливо $x\leq a$.
    \end{definition}
    
    Наименьший элемент частично упрядоченного множества $A$ обозначается $\min{A}$, а наибольший $\max{A}$.
\end{frame}

    
\begin{frame}
    \frametitle{Наименьший и наибольший элементы}
    
    \begin{theorem}
        Частично упорядоченное множество содержит не более одного наименьшего (наибольшего) элемента.
    \end{theorem}
    \begin{proof}    
        Допустим, что в множестве более одного наименьшего элемента. Допустим, что $a_1$, $a_2$ --- два из этих элементов, тогда справедливо, что $a_1\leq a_2$ и $a_2\leq a_1$, стало быть $a_1=a_2$. Аналогичго можно доказать теорему для наибольшего элемента.
    \end{proof}

    В качестве следствия этой теоремы можно отметить, что наименьшего (наибольшего) элемента может \alert{не быть}, а также то, что наименьший (наибольший) элемент также будет минимальным (максимальным). 
\end{frame}

\begin{frame}
    \frametitle{Задание}
    
    Частично упорядоченное множество $\langle\{a,b,c\},\leq\rangle$
    \[
        {\xymatrix{
            a \ar@{->}@(dl,ul)[] \ar@{->}[r]
                &c \ar@{->}@(ur,dr)[]
                    \\
            b  \ar@{->}@(dl,ul)[] \ar@{->}[ur]
                &*{}                 
        }}
    \]
    Найти минимальный, максимальный, наименьший и наибольший элементы.
    \uncover<2>{
        \begin{itemize}
            \item минимальные: $a$, $b$;
            \item максимальный: $c$;
            \item наименьший: \alert{нет};
            \item наибольший: $c$.
        \end{itemize}
    }
\end{frame}    


\begin{frame}
    \frametitle{Грани}

    \begin{definition}    
        \alert{Нижней} (\alert{верхней}) гранью подмножества $B$ частично упорядоченного множества $A$ ($B\subseteq A$) называется элемент $a\in A$, такой что $a\leq b$ ($b\leq a$) для всех $b\in B$. 
    \end{definition}    

    \begin{definition}    
        Элемент $a\in A$ называется \alert{точной} нижней гранью (инфимумом $\inf{B}$) множества $B\subseteq A$, если $a$ --- \alert{наибольшая} из нижних граней $B$. 
    \end{definition}    

    \begin{definition}    
        Элемент $a\in A$ называется \alert{точной} верхней гранью (супремумом $\sup{B}$) множества $B\subseteq A$, если $a$ --- \alert{наименьшая} из верхних граней $B$. 
    \end{definition}    
    
    Например, для $B=[0,1)$, $B\subset\mathbb{R}$ справедливо $\inf{B}=0,\sup{B}=1$.
\end{frame}

\begin{frame}
    \frametitle{Полный порядок}
    
    \begin{definition}
        Линейный порядок $\leq$ на множестве $A$ назывется \alert{полным}, если каждое непустое подмножество множества $A$ имеет наименьший элемент. В этом случае множество $A$ называется \alert{вполне упорядоченным}.
    \end{definition}
    
    \begin{itemize}
        \item $\langle\mathbb{N},\leq\rangle$ --- вполне упорядоченное
        \item $\langle\mathbb{Z},\leq\rangle$ --- не является вполне упорядоченным
    \end{itemize}
\end{frame}

\begin{frame}
    \frametitle{Диаграмма Хассе}
    
    \begin{definition}
        Элемент $y$ \alert{покрывает} элемент $x$, если $x\leq y$ и не существует $z$, такого, что $x<z<y$. 
    \end{definition}
    
    \begin{definition}[Диаграмма Хассе]
        Конечное частично упорядоченное множество $\langle A,\leq\rangle$ можно представить в виде графа (\alert{диаграммы Хассе}), в котором вершинами являются элементы $A$, и если $y$ покрывает $x$, то вершины $x,y$ соединяют линией, причём вершину $x$ располагают ниже вершины $y$.
    \end{definition}
\end{frame}

Диаграмма Хассе получается из орграфа отношения удалением петель и транзитивно замыкающих дуг ( стрелки превращаются в линии).

\begin{frame}
    \frametitle{Диаграмма Хассе для $\langle2^{\{1,2,3\}},\subseteq\rangle$}
    
    \[
        {\xymatrix{
            *{} 
                &\{1,2,3\}
                    &*{}
                        \\
            \{1,2\} \ar@{-}[ur]
                &\{1,3\} \ar@{-}[u]
                    &\{2,3\} \ar@{-}[ul]
                        \\
            \{1\} \ar@{-}[u] \ar@{-}[ur]
                &\{2\} \ar@{-}[ul] \ar@{-}[ur]
                    &\{3\} \ar@{-}[u] \ar@{-}[ul]
                        \\
            *{}
                &\emptyset \ar@{-}[ul] \ar@{-}[u] \ar@{-}[ur]
                    &*{}
        }}
    \]
\end{frame}    

\begin{frame}
    \frametitle{Построение диаграммы Хассе для $\langle A,\preceq\rangle$}
    
    \begin{example}[$A=\{2,3,10,14,33,42,70,462\}$]
        Построить диаграмму Хассе для отношения $x\preceq y$. Причём $x\preceq y$, если $x$ делит нацело $y$.
    \end{example}
    
    \begin{proof}[Решение]
        \raisebox{\depth}{
            \begin{tabular}{ll}
                \hline\hline
                $x\in A$        & Покрывающие $x$                       \\ \hline\hline
                \alert<2>{$2$}  & \uncover<2->{$\{10,14\}$}             \\
                \alert<2>{$3$}  & \uncover<2->{$\{33,\alert<3>{42}\}$}  \\
                $10$            & \uncover<2->{$\{70\}$}                \\
                $14$            & \uncover<2->{$\{70,\alert<3>{42}\}$}  \\
                $33$            & \uncover<2->{$\{\alert<4>{462}\}$}    \\
                $42$            & \uncover<2->{$\{\alert<4>{462}\}$}    \\
                $70$            & \uncover<2->{$\emptyset$}             \\
                $462$           & \uncover<2->{$\emptyset$}             \\ \hline
            \end{tabular}
        }
        \only<3>{\raisebox{\depth}{        
            {\xymatrix@=.7pc{
                *{~} & 2 & 3 &*{}
            }}
        }}
        \only<4>{\raisebox{\depth}{
            {\xymatrix@=.7pc{
                10  &14 &*{}&33 \\
                *{} 
                    & 2 \ar@{-}[u] \ar@{-}[ul]
                        & 3 \ar@{-}[ur]
                            & *{}
            }}
        }}
        \only<5>{\raisebox{\depth}{
            {\xymatrix@=.7pc{
                *{} &70 &42 &*{} \\
                10 \ar@{-}[ur]
                    &14 \ar@{-}[u] \ar@{-}[ur]
                        &*{}
                            &33 
                                \\
                *{} 
                    & 2 \ar@{-}[u] \ar@{-}[ul]
                        & 3 \ar@{-}[ur] \ar@{-}[uu]
                            & *{}
            }}
        }}
        \only<6>{\raisebox{\depth}{
            {\xymatrix@=.7pc{
                *{} &*{}&462&*{} \\
                *{} 
                    &70 
                        &42 \ar@{-}[u]
                            &*{} 
                                \\
                10 \ar@{-}[ur]
                    &14 \ar@{-}[u] \ar@{-}[ur]
                        &*{}
                            &33  \ar@{-}[uul]
                                \\
                *{} 
                    & 2 \ar@{-}[u] \ar@{-}[ul]
                        & 3 \ar@{-}[ur] \ar@{-}[uu]
                            & *{}
            }}
        }}
    \end{proof}
\end{frame}

\begin{frame}
    \frametitle{Сравнение диаграммы Хассе с исходным отношением}
    
    \begin{example}[$A=\{2,3,10,14,33,42,70,462\}$]
        Построить диаграмму Хассе для отношения $x\preceq y$. Причём $x\preceq y$, если $x$ делит нацело $y$.
    \end{example}
    
    \begin{center}
    \begin{tabular}{c||c}
        Исходный хаос ($x\preceq y$)
            & Ставший порядок (Хассе) 
                \\ \hline
            &   \\
        {\xymatrix@=.7pc{
            10 \ar[d]
                &2 \ar[l] \ar[dl] \ar[dd] \ar[dr] \ar@/_/[ddr]
                    &*{}
                        &33 \ar[dl]
                            \\
            70  
                &*{}
                    &462
                        &*{}
                            \\
            *{}
                &14 \ar[ul] \ar[ur] \ar[r]
                    &42 \ar[u]
                        &3 \ar[uu] \ar[ul] \ar[l]
        }}
            &    
            {\xymatrix@=.7pc{
                *{} &*{}&462&*{} \\
                *{} 
                    &70 
                        &42 \ar@{-}[u]
                            &*{} 
                                \\
                10 \ar@{-}[ur]
                    &14 \ar@{-}[u] \ar@{-}[ur]
                        &*{}
                            &33  \ar@{-}[uul]
                                \\
                *{} 
                    & 2 \ar@{-}[u] \ar@{-}[ul]
                        & 3 \ar@{-}[ur] \ar@{-}[uu]
                            & *{}
            }}
    \end{tabular}
    \end{center}
\end{frame}

\begin{frame}
    \begin{theorem}
        Всякий частичный порядок $\leq$ может быть дополнен до полного $P$
    \end{theorem}
    
    \begin{algorithm}[H]
        \caption{Топологическая сортировка}
        \begin{algorithmic}[1]
            \REQUIRE{$\langle A,\leq\rangle$ --- частично упорядоченное множество}
            \ENSURE{$\langle A,P\rangle$ --- вполне упорядоченное множество}
            \STATE{$M_1\gets A$}
            \STATE{$M_2\gets \emptyset$}
            \STATE{$P\gets \emptyset$}
            \WHILE{$M_1\neq \emptyset$}
                \STATE{$m\gets\textit{ЛюбойМинимальный}(M_1,\leq)$}
                \STATE{$M_2\gets M_2\cup\{m\}$}
                \STATE{$P\gets P\cup\{(a,m)|a\in M_2\}$}
                \STATE{$M_1\gets M_1\backslash\{m\}$}
            \ENDWHILE
            \RETURN{$\langle A,P\rangle$}
        \end{algorithmic}
    \end{algorithm}
\end{frame}

\begin{frame}
    \frametitle{Пример топологической сортировки}
    
    На диаграмме Хассе задан частичный порядок $\preceq$:
    \[
        {\xymatrix@=.7pc{
            *{} 
                &*{}
                    &\uncover<-8>{462}
                        &*{} 
                            \\
            *{} 
                &\uncover<-6>{70} 
                    &\uncover<-7>{42} \ar@{-}[u]
                        &*{} 
                            \\
            \uncover<-3>{10} \ar@{-}[ur]
                &\uncover<-4>{14} \ar@{-}[u] \ar@{-}[ur]
                    &*{}
                        &\uncover<-5>{33}  \ar@{-}[uul]
                            \\
            *{} 
                & \uncover<-1>{2} \ar@{-}[u] \ar@{-}[ul]
                    & \uncover<-2>{3} \ar@{-}[ur] \ar@{-}[uu]
                        & *{}
        }}
    \]
    Найти полный порядок $\leq$:
    \[
        \uncover<2->{2}   \leq
        \uncover<3->{3}   \leq
        \uncover<4->{10}  \leq
        \uncover<5->{14}  \leq
        \uncover<6->{33}  \leq
        \uncover<7->{70}  \leq
        \uncover<8->{42}  \leq
        \uncover<9->{462} 
    \]
\end{frame}

Отношение порядка --- очень важное отношение. Например, вводя отношение порядка, можно значительно ускорить поиск элемента множества, а это весьма распространенная практическая задача. Можно рекомендовать книгу \cite{bib:knuth:artOfProgramming3}, которая целиком посвящена вопросам сортировки и поиска.

\appendix


\begin{frame}
    \frametitle{В заключение}
    
    Приводятся сведения о бинарных отношениях с особыми свойствами. Бинарные отношения активно применяются на практике. Специальные же бинарные отношения достойны особой роли, благодаря тем свойствам, которыми они обладают. Для углубленного изучения рекомендуются: \cite{bib:novic:discrmathprogrammer,bib:sudoplatov:discrmath,bib:shaporev:discretemath}.
\end{frame}


\begin{frame}[allowframebreaks]{Библиография}
    \bibliographystyle{gost780u}
    \bibliography{./../../bibliobase}
\end{frame}

\end{document}
\usepackage{etex} %эта магическая херь избавляет от переполнения регистров TeX а!!!

\mode<article>{\usepackage{fullpage}}
\mode<presentation>{
    \usetheme{Madrid} %%Boadilla,Madrid,AnnArbor,CambridgeUS,Malmoe,Singapore,Berlin
    \useoutertheme{shadow}
} 

\usepackage[utf8]{inputenc}
\usepackage[russian]{babel}
\usepackage{indentfirst}
\usepackage{graphicx}

\usepackage{amsmath}
\usepackage{amsfonts}
\usepackage{amsthm}
\usepackage{algorithm}
\usepackage{algorithmic}

\usepackage[all]{xy}

\date{Лекция по дисциплине <<дискретная математика>>\\(\today)}
\author[М.~М.~Шихов]{Михаил Шихов \\ \texttt{\underline{m.m.shihov@gmail.com}}}

%для рисования графов пакетом xy-pic
\entrymodifiers={++[o][F-]}

%для псевдокода алгоритмов (algorithm,algorithmic)
\renewcommand{\algorithmicrequire}{\textbf{Вход:}}
\renewcommand{\algorithmicensure}{\textbf{Выход:}}
\renewcommand{\algorithmiccomment}[1]{// #1}
\floatname{algorithm}{Псевдокод}

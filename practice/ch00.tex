\chapter{Позиционные системы счисления}
\label{ch:ss}

\emph{Число} является одним из важнейших понятий математики (дискретной в том числе). Числа существует в мире человеческого сознания\footnote{Попробуйте представить себе число \emph{один}, \emph{два}, \emph{три}, \emph{семь}, \emph{тысяча}, \emph{тысяча двести двенадцать}, \ldots Какие образы у вас возникли? Солнце, две вишенки, листок клевера, семья, войско, \rotatebox{15}{\textbf{1212}},\ldots Не так?}. Чтобы использовать числа на практике, нужно уметь \emph{представить} их средствами физического мира. Например:
\begin{itemize}
    \item строкой знаков на бумаге;
    \item комбинацией косточек на счётах;
    \item положением шестерёнок в механической счётной машине;
    \item изменением уровня напряжения в электрическом кабеле;
    \item изменением яркости пучка света в оптоволокне;
    \item значениями ячеек памяти компьютера и т.д.
\end{itemize}

Далее внимание читателя будет сосредоточено на некоторых способах конструирования \emph{представлений чисел} с помощью конечного набора \emph{символов} (знаков). Также будут разобраны преобразования таких представлений, соответсвующие операциям сложения и вычитания.

Подробнее о системах счисления см. \cite{bib:gorbatov:fodm,bib:sudoplatov:discrmath}.


\section{Символьное представление целых и вещественных чисел}

Позиционная система счисления является способом представления\footnote{Запись числа на бумаге является представлением} чисел с помощью конечного набора символов (знаков). Система счисления --- есть способ \emph{кодирования} чисел.

Натуральное\footnote{Далее считаем, что ноль является натуральным числом: $0$, $1$, $2$, $3$,\ldots} $n$-разрядное число в позиционной системе счисления с основанием $K$ представляется (записывается) следующим образом:
\begin{equation}\label{eq:ch:ss:digitsN}
    (a_{n-1}\cdots a_{1}a_{0})_K.
\end{equation}

В такой записи на месте $a_{i}$, где $i=0,1,\ldots, n-1$, находится один из $K$ отличимых друг от друга \emph{символов} --- \emph{цифра}. Каждому такому символу (цифре) взаимооднозначно соответствует натуральное \emph{число} из промежутка $[0,K-1]$. В представлении числа $i$-я цифра $a_i$ называется также $i$-м \emph{разрядом}.

На основе своего представления натуральное число формируется так:
\begin{equation}\label{eq:ch:ss:intPart}
    \sum_{i=0}^{n-1}a_i\cdot K^{i}.
\end{equation}

Как видно из формулы \eqref{eq:ch:ss:intPart}, каждый $i$-й разряд представления \eqref{eq:ch:ss:digitsN} вносит свой, кратный $K^{i}$, вклад в общую сумму, составляющую число. Чем старше разряд, тем величина вкдада больше\footnote{Конечно, при условии, что $a_i\neq 0$}. Причём для $n\geq 0$ справедливо:
\[
    K^n\geq 1+\sum_{i=0}^{n-1}a_i\cdot K^i.
\]

То есть $n$-й разряд вносит в общую сумму вклад больший, чем сумма вкдадов всех предыдущих (младших) разрядов.

Каким бы большим не было натуральное число, рано или поздно все цифры в разрядах старше некоторого $n-1$-го будут нулевыми:
\[
    (\cdots 0000000a_{n-1}\cdots a_{1}a_{0})_K,
\]
где $a_{n-1}\neq 0$. Поэтому бесконечный ряд нулей слева в записи числа опускают.

Позиционная система счисления с основанием $K$ называется \emph{$K$-ичной}. Одна и та же цифра $a$, записанная в разных позициях ($i$ --- номер позиции в формуле (\ref{eq:ch:ss:intPart})), например в $i$-й и $j$-й будет вносить в общую сумму разной величины слагаемые: $a\cdot K^i$ и $a\cdot K^j$ соответственно. От того, в какой позиции (разряде) стоит цифра в представлении числа, зависит всё. Именно поэтому система счисления названа \emph{позиционной}. Значение $K^i$ называется \emph{весом} $i$-го разряда.

\begin{exampl}
    Записи $(78642)_{10}$ (большинство запишет просто \textbf{78642}, так как с детства приучены считать именно в \emph{десятичной} системе счисления) соответствует число
    \[
        7\cdot 10^{4}+
        8\cdot 10^{3}+
        6\cdot 10^{2}+
        4\cdot 10^{1}+
        2\cdot 10^{0}=78642.
    \]
\end{exampl}

\begin{exampl}
    Записи $(10221)_{3}$  соответствует число
    \[
        \begin{array}[c]{c}
            1\cdot 3^{4}+
            0\cdot 3^{3}+
            2\cdot 3^{2}+
            2\cdot 3^{1}+
            1\cdot 3^{0} = \\
            =1\cdot 81+
            0\cdot 27+
            2\cdot 9+
            2\cdot 3+
            1\cdot 1 = \\
            = 106.
        \end{array}
    \]
\end{exampl}

\begin{exampl}
    Записи $(10101)_{2}$  соответствует число
    \[
        \begin{array}[c]{c}
            1\cdot 2^{4}+
            0\cdot 2^{3}+
            1\cdot 2^{2}+
            0\cdot 2^{1}+
            1\cdot 2^{0} = \\
            =1\cdot 16+
            0\cdot 8+
            1\cdot 4+
            0\cdot 2+
            1\cdot 1 = \\
            = 21.
        \end{array}
    \]
\end{exampl}

\begin{exampl}
    В ручной записи \emph{числа} каждой \emph{цифре} соответствует \emph{символ} определенного начертания. Поэтому, если оговорено, например, что $K=3$ и цифре $\alpha$ соответствует ноль, $\beta$ --- один, $\gamma$ --- два, то записи 
    \(
        (\beta\alpha\gamma\gamma\beta)_3
    \)
    соответствует число $106$. 
\end{exampl}

В вычислительных системах каждой цифре будет соответствовать определенное устойчивое состояние физической среды. Например в ЭВМ, производящей вычисления в двоичной системе счисления, цифрам $0$ и $1$ соответствуют низкий и высокий (иногда наоборот) уровни электрического сигнала.

Дробная часть положительного вещественного (действительного) числа, изменяющаяся, как известно в пределах $[0,1)$ будет представлена в позиционной системе счисления с основанием $K$ так:
\[(.a_{-1}a_{-2}\cdots a_{-m}\cdots)_K,\]
где $m>0$ --- количество значащих разрядов дробной части.

Дробная часть при этом на основе своего представления формируется так:
\begin{equation}\label{eq:ch:ss:floatPart}
        \sum_{i=-m}^{-1}a_{i}\cdot K^{i}
\end{equation}
или
\begin{equation}\label{eq:ch:ss:floatPartAlt}
        \sum_{i=1}^{m}\frac{a_{-i}}{K^{i}}.
\end{equation}

Для записи иррациональных чисел, например таких, как число 
\[\pi=3.141592653589793238462643\cdots\] 
понадобится \emph{бесконечное} количество цифр для представления дробной части в позиционной системе счисления с \emph{любым} целым основанием. Но, как видно из примера, записи $\pi$, на практике ничего другого не остается как отбросить, начиная с некоторого $i<-m$, все цифры справа, теряя в точности представления. Для дробной части также справедливо, что $-n$-й разряд вносит вклад больший, чем суммарный вклад младших разрядов
\[
    K^{-n}>\sum_{i=n+1}^{\infty}\frac{a_{-i}}{k^i}.
\]

\begin{exampl}
    Записи дробной части $(.10111)_{2}$  соответствует число:
    \[
        \begin{array}[c]{c}
            1\cdot 2^{-1}+
            0\cdot 2^{-2}+
            1\cdot 2^{-3}+
            1\cdot 2^{-4}+
            1\cdot 2^{-5} = \\
            =
            1\cdot 0.5+
            0\cdot 0.25+
            1\cdot 0.125+
            1\cdot 0.0625+
            1\cdot 0.03125 = \\
            = 0.71875
        \end{array}
    \]
\end{exampl}

Объединяя вышесказанное относительно целой ($A_{int}$) и дробной ($A_{frac}$) части (формулы (\ref{eq:ch:ss:intPart}) и (\ref{eq:ch:ss:floatPart})) можем сказать, что положительное вещественное число 
\[
    A=
    \underbrace{\left(\sum_{j=0}^{n-1}a_j\cdot K^{j}\right)}_{A_{int}} + 
    \underbrace{\left(\sum_{i=-m}^{-1}a_i\cdot K^{i}\right)}_{A_{frac}} =
    \sum_{i=-m}^{n-1}a_i\cdot K^{i}.
\]
можно представить:
\[
    A\equiv(\underbrace{a_{n-1}\cdots a_{0}}_{A_{int}}\underbrace{.a_{-1}\cdots a_{-m}}_{A_{frac}})_K.
\]

Перед старшей цифрой представления отрицательного числа ставят знак минус\footnote{Желая подчеркнуть, что число положительное, перед старшей цифрой ставят необязательный плюс}. В целом:
\[
    A\equiv(\pm a_{n-1}\cdots a_{0}.a_{-1}\cdots a_{-m})_K.
\]

Число --- понятие универсальное. Одно и то же число совершенно по разному будет представлено (т.е. будет выглядеть) в системах счисления с разными основаниями. 

Операции над числами (например, сложение, вычитание и пр.) реализуются на практике соответсвующими преобразованиями представлений чисел. Поэтому очень важно, какая используется система счисления. Люди привыкли использовать десятичную систему счисления, и вряд ли у читателя получится так же быстро сложить числа $10$ и $20$, если они представлены в двоичной системе счисления. Здесь и далее по тексту, если явно не оговорено, как представлено число --- используется десятичная система счисления.


\section{Перевод чисел из одной системы счисления в другую}

Пусть необходимо представить число $A$ в $K$-ичной системе счисления. При этом само число представлено в $L$-ичной системе, а вычислитель способен выполнять операции в $M$-ичной системе счисления.

\begin{itemize}
    \item $A\equiv(\pm\cdots a_1a_0.a_{-1}a_{-2}\cdots)_L$;
    \item $A\to B$, $A=B$;
    \item $A\equiv(\pm\cdots b_1b_0.b_{-1}b_{-2}\cdots)_K$;
    \item Вычислитель считает в $M$-ичной СС.
\end{itemize}

Например, люди привыкли считать в десятичной системе счисления и, если число $A$ записано не в десятичной системе, то ничего другого не остается, как это число в ней представить.

Знак у переводимого в $M$-ичную систему счисления числа останется прежним, поэтому вначале рассмотрим перевод положительного числа:
\[
    A\equiv(a_{n}\cdots a_{0}.a_{-1}\cdots a_{-m})_L;
\]
\begin{equation}\label{f:srcLBase}
    A=\sum_{i=-m}^{n}a_i\cdot L^{i}.
\end{equation}

Каждой цифре $a_{i}$, а также основанию $L$ ставятся в соответствие \emph{числа} в системе счисления с основанием $M$ (это не представляет сложности). Далее вычислитель по формуле (\ref{f:srcLBase}) проводит расчет и получает число в $M$-ичной системе счисления.

\begin{exampl}[Задача]
    Дано число $A\equiv(-7AFC.4)_{16}$ в шестнадцатиричной системе счисления. Необходимо представить его в доступной для вычислителя системе счисления.
\end{exampl}
\begin{proof}[Решение]
    Данный текст был написан вычислителем, которому удобно считать в десятичной системе. В шестнадцатиричной системе счисления цифры обозначены следующим образом: цифрам от нуля до девяти соответствуют цифры десятичной системы, а далее используются латинские буквы от $A$ до $F$ в алфавитном порядке, которым соответствуют числа от $10$ до $15$ соответственно.
    \[
        \begin{array}[c]{c}
            A=-(
            7\cdot 16^{3}+
            A\cdot 16^{2}+
            F\cdot 16^{1}+
            C\cdot 16^{0}+
            4\cdot 16^{-1})=\\
            =-(
            7\cdot 16^{3}+
            10\cdot 16^{2}+
            15\cdot 16^{1}+
            12\cdot 16^{0}+
            4\cdot 16^{-1})=\\
            =
            -(7\cdot 4096+
            10\cdot 256+
            15\cdot 16+
            12\cdot 1+
            4\cdot 0.0625)=\\
            = -31484.25
        \end{array}
    \]
    
    После пересчёта получен результат в $10$ СС.
\end{proof}

После того, как число $A$ представлено в системе счисления ($M=10$), в которой умеет считать вычислитель, его можно перевести в целевую $K$-ичную. Допустим, что число $A$ уже представлено в $K$-ичной системе. 

\[
    \begin{matrix}
        A\equiv(\underbrace{b_{n'}\cdots b_0}_{A_{int}}\underbrace{.b_{-1}\cdots b_{-m'}}_{A_{frac}})_K\\
        A=
            \underbrace{\left(\sum_{i=0}^{n'}b_i\cdot K^i\right)}_{A_{int}} + 
            \underbrace{\left(\sum_{i=-m'}^{-1}b_i\cdot K^i\right)}_{A_{frac}}\\
    \end{matrix}
\]

Рассмотрим что будет, если разделить его \emph{целую} часть $A_{int}$ на основание $K$:
\[
    A_{int}=K\cdot\underbrace{\left(\sum_{i=1}^{n}b_i\cdot K^{i-1}\right)}_\text{частное $A_{int}^{(1)}$}+b_{0} = K\cdot A_{int}^{(1)} + b_{0},
\]
где остаток $a_{0}\in[0,K-1]$ есть искомое значение нулевого разряда, которое однозначно переводится в цифру $K$-ичной системы. Если поделить частное $A_{int}^{(1)}$ на $K$, то найдем значение первого разряда и так далее: 
\begin{align*}
    \displaystyle
    A_{int}^{(1)}=&K\cdot\left(\sum_{i=2}^{n'}b_i\cdot K^{i-2}\right)+b_{1} = K\cdot A_{int}^{(2)}+b_{1},\\
    A_{int}^{(2)}=&K\cdot A_{int}^{(3)}+b_{2},\\
    A_{int}^{(3)}=&K\cdot A_{int}^{(4)}+b_{3},\\
    \cdots \\
    A_{int}^{(n')}=&K\cdot 0 + b_{n'}.
\end{align*}

Каждый следующий разряд числа получается как остаток от частного, полученного на предыдущем шаге. Процесс завершается, когда на шаге $n'$ получается нулевое частное.

Теперь представим, что дробная часть $A_{frac}$ числа $A$ также представлена в $K$-ичной системе. Умножим дробную часть на $K$:
\[
    A_{frac}\cdot K = 
        \left(\sum_{i=-m'}^{-1}b_i\cdot K^{i} \right)\cdot K=
            b_{-1} + 
            \underbrace{\left(\sum_{i=-m'}^{-2}b_i\cdot K^{i+1}\right)}_{A_{frac}^{(1)}}
            = b_{-1} + A_{frac}^{(1)},
\]
где\footnote{$\lfloor x\rfloor$ обозначает наибольшее целое, меньшее, либо равное $x$. Например, $\lfloor 4.7\rfloor = 4$} целая часть $b_{-1}=\lfloor Y\cdot K\rfloor$ есть искомое значение $1$-го разряда дробной части $b_{-1}\in[0,K-1]$, которое однозначно переводится в цифру $K$-ичной системы.

Оставшаяся дробная часть $A_{frac}^{(1)}$ умножается на $K$. Таким образом находится $b_{-2}$ и так далее:
\begin{align*}
    \displaystyle
    A_{frac}^{(1)}\cdot K &= \left(\sum_{i=-m}^{-2}b_i\cdot K^{i+1} \right)\cdot K = b_{-2} + A_{frac}^{(2)},\\
    A_{frac}^{(2)}\cdot K &= b_{-3} + A_{frac}^{(3)},\\
    A_{frac}^{(3)}\cdot K &= b_{-4} + A_{frac}^{(4)},\\
    \cdots \\
    A_{frac}^{(-m'+1)}\cdot K &= \left(\sum_{i=-m}^{-m}b_i\cdot K^{i+m-1} \right)\cdot K = b_{-m'} + 0.
\end{align*}

Процесс поиска завершается, когда очередная дробная часть числа $A_{frac}^{(m')}$ станет равна нулю или значения дробных частей не начнут периодически повторяться\footnote{Тогда получается представление периодической дроби}.

\begin{exampl}[Задача]
    Дано число $A\equiv-31484.25$ в десятичной системе. Необходимо его представить в системе счисления с основанием $3$.
\end{exampl}
\begin{proof}[Решение]
    Переводится целая часть:
    \[
        \begin{array}[c]{l}
            31484=10494\cdot 3 + 2,\Rightarrow b_{0}=2, \\
            10494=3498\cdot 3 + 0,\Rightarrow b_{1}=0, \\
            3498=1166\cdot 3 + 0,\Rightarrow b_{2}=0, \\
            1166=388\cdot 3 + 2,\Rightarrow b_{3}=2, \\
            388=129\cdot 3 + 1,\Rightarrow b_{4}=1, \\
            129=43\cdot 3 + 0,\Rightarrow b_{5}=0, \\
            43=14\cdot 3 + 1,\Rightarrow b_{6}=1, \\
            14=4\cdot 3 + 2,\Rightarrow b_{7}=2, \\
            4=1\cdot 3 + 1,\Rightarrow b_{8}=1, \\
            1=0\cdot 3 + 1,\Rightarrow b_{9}=1
        \end{array}
    \]

    Переводится дробная часть (получается периодическая дробь):
    \[
    \begin{array}[c]{l}
        0.25\cdot 3=0.75,\Rightarrow a_{-1}=0, \\
        0.75\cdot 3=2.25,\Rightarrow a_{-2}=2, \\
        0.25\cdot 3=0.75,\Rightarrow a_{-1}=0, \\
        0.75\cdot 3=2.25,\Rightarrow a_{-2}=2, \\
        \cdots
    \end{array}
    \]

    Результат: $A\equiv(-7AFC.4)_{16}\equiv -31484.25 \equiv (-1121012002.(02))_{3}$.
\end{proof}

Напоследок остается заметить, что иногда цифре $K$-ичной системы не всегда соответствуют числа из промежутка $[0,K-1]$. Вполне допустимо сдвинуть весь промежуток влево, в сторону отрицательных чисел: 
\([0-m,K-1-m],\)
где  $m\in[1,\ldots,K-2]$.

Как осуществлять перевод чисел в этом случае рассматривается при обсуждении троичной симметричной системы (см. \ref{s:ch:ss:triplet} на странице \pageref{s:ch:ss:triplet}). Далее рассматривается несколько популярных позиционных систем счисления.



\section{Двоичная система счисления}
\label{s:ch:ss:binaryNS}

Система счисления с основанием $K=2$. Цифры всего две: $0$ и $1$. Цифру двоичной системы счисления называют битом (bit --- binary digit).

Рассмотрим, как производится сложение чисел $A$ и $B$ в двоичной системе. Представления чисел 
\begin{align*}
    A\equiv&(a_n\cdots a_0.a_{-1}\cdots a_{-m})_2\\
    B\equiv&(b_{n'}\cdots b_0.b_{-1}\cdots b_{-m'})_2
\end{align*}
выравниваются по разделителю целой и дробной части числа (точке). Сложение выполняется поразрядно, начиная с младших разрядов (справа-налево). Значение $i$-го разряда результата получается сложением трех величин: 
\[
    a_i+b_i+c_{i-1},
\]
где $a_i, b_i$ --- значения разрядов операндов, а $c_{i-1}$ --- значение переноса из младшего $(i-1)$-го разряда результата. Правила сложения значений разрядов представлены в следующей таблице:

\begin{equation}\label{f:addBinary}
    \begin{array}[c]{c|c|c|}
        + & 0 & 1 \\
        \hline
        0 & \xleftarrow{0}0 & \xleftarrow{0}1\\
        \hline
        1 & \xleftarrow{0}1 & \xleftarrow{1}0 \\
        \hline
    \end{array}
\end{equation}
	
В записи $\xleftarrow{c_i}r_i$ внутри ячейки, значение $c_i$ над стрелкой есть перенос в старший $(i+1)$-й разряд, а $r_i$ --- результат сложения. Ненулевой перенос возникает в случае, когда сумма цифр <<не умещается>> в одном разряде, например, $1+1=2$, а $2\equiv (10)_2$ --- возникает единица переноса: $\xleftarrow{1}0$.

\begin{exampl}[Задача]
    Сложить двоичные числа:
    $A\equiv(101.1101)_2$ и
    $B\equiv(11.010111)_2$.
\end{exampl}
\begin{proof}[Решение]
    \[
        {\entrymodifiers={}
            {\xymatrix@=1pc{
                A&\equiv
                    & &1&0&1&.&1&1&0&1& & \\
                B&\equiv
                    &   &   &1  &1  &.  &0  &1  &0  &1  &1  &1\\
                \xleftarrow{c}             
                &  
                    &\xleftarrow{1}
                        &\xleftarrow{1}
                            &\xleftarrow{1}
                                &\xleftarrow{1}
                                    & 
                                        &\xleftarrow{1}
                                            & 
                                                &\xleftarrow{1}
                                                    & 
                                                        & 
                                                            & \\                    
                A+B
                &\equiv
                    &1
                        &0\ar[ul]|-{6}
                            &0\ar[ul]|-{5}
                                &1\ar[ul]|-{4}
                                    &.
                                        &0\ar[ull]|-{3}
                                            &0\ar[ul]|-{2}
                                                &1
                                                    &0\ar[ul]|-{1}
                                                        &1
                                                            &1
            }}
        }
    \]
    
    Нулевые переносы, не меняющие результат сложения, на рисунке не показаны. В качестве упражнения предлагается перевести слагаемые и результат в десятичную систему и перепроверить результат.
\end{proof}

\emph{Сложение} и \emph{вычитание} --- базовые операции, на которых основаны операции \emph{умножения} и \emph{деления}. На практике используются приёмы, позволяющие заменить вычитание сложением. Эти приёмы основаны на потере переноса при сложении чисел в конечной разрядной сетке\footnote{В процессорах используется фиксированное количество разрядов для представления чисел}. Рассмотрим, как можно перейти от вычитания к сложению, используя \emph{дополнительный} и \emph{обратный} коды.

Допустим, что для представления двоичного числа используется $m$ разрядов. В такой разрядной сетке можно представить целые положительные числа от нуля до $(2^{m}-1)$. Например, если к числу $(2^{m}-1)$, имеющему двоичное представление
\[
    (\underbrace{11\cdots 11}_m)_2,
\]
прибавить единицу, то возникнет перенос в отсутствующий в разрядной сетке $m$-й разряд\footnote{Будем считать крайний справа (младший) разряд разрядной сетки \emph{нулевым}}. Этот перенос будет потерян, и результом будет \emph{ноль}. 

В $m$-разрядной сетке можно представить не только целые числа, но и вещественные. Для этого достаточно жестко <<зафиксировать>> точку (разделитель целой и дробной части) между определенными разрядами. Такой формат представления вещественных чисел называется форматом с \emph{фиксированной точкой} (\emph{запятой})\footnote{Есть и формат с <<плавающей>> точкой, в котором сохраняется информация о том, между какими разрядами находится точка}.

Известно, что
\[A-B=A+(-B).\]

Как представляется отрицательное число $(-B)$?

В дополнительном коде это делается так: в $n$-разрядной сетке старший, $(n-1)$-й, разряд считается \emph{знаковым}. Как будет видно далее, он действительно хранит знак числа: <<$0$>> соответствует знаку <<$+$>>, а <<$1$>> --- знаку <<$-$>>. Знаковый разряд будем выделять следующим образом: $\Sign{0}$ или $\Sign{1}$. В остальных разрядах находятся цифры двоичного представления числа. Разрядная сетка \emph{на время вычислений} дополняется еще одним разрядом, значение которого дублирует значение знакового разряда. Этот разряд вводится для того, чтобы отличить правильный результат вычислений от результата, который в $n$-разрядной сетке представить невозможно\footnote{Эта ситуация называется <<переполнением разрядной сетки>>}. Знаковый разряд с <<дублёром>> далее будем выделять следующим образом: $\Signs{0}{0}$ или $\Signs{1}{1}$. В итоге получается $m$-разрядная сетка: $m=n+1$. 

Как ранее уже отмечалось, $2^m$ в такой сетке соответствует нулю. Тогда отрицательное число $-B$ можно представить как $2^m-B$.
\[
-B=(2^m-1)-B+1 = (\underbrace{\Signs{1}{1}1\cdots11}_{m})_{2}-B+1=\overline{B}+1,
\]
где $\overline{B}$--- инверсия\footnote{Т.е. $1$ переходит в $0$, а $0$ в $1$} всех разрядов числа $B$.

$\overline{B}+1$ называется дополнительным кодом\footnote{Можно заметить, что инвертируются все разряды числа $B$, кроме самого младшего единичного} числа $-B$. При этом в знаковом\footnote{Т.е. в разряде с номером $n-1$} разряде будет $\Sign{1}$. Дополнительный код положительного числа совпадает с представлением модуля $B$ в $n$-разрядной сетке\footnote{При этом в знаковом разряде всегда $\Sign{0}$}. Модули (положительные части) чисел при этом должны быть представимы в $(n-1)$ разрядной сетке, иначе в знаковых разрядах результата будут комбинации, отличные от $\Signs{0}{0}$ и $\Signs{1}{1}$. В $n$-разрядной сетке диапазон изменения представимого в дополнительном коде числа $X$:

\[-2^{n-1}\leq X \leq 2^{n-1}-1.\]

Дополнительный код числа $X$ будем обозначать $\DC{X}$.
\[
    \DC{X}=
    \begin{cases}
        \overline{|X|}+1, &\text{если\,} X<0, \\
        |X|,              &\text{если\,} X\geq 0.
    \end{cases}
\]

При этом обратный перевод (получение двоичного представления модуля числа)
\footnote{
    Для отрицательного $X$: 
    \[\DC{X}=(2^m-1)-|X|+1 \Rightarrow |X|=(2^m-1)-\DC{X}+1=\overline{\DC{X}}+1\]
}:
\[
    |X|=
    \begin{cases}
        \overline{\DC{X}}+1, &\text{если в знаковых разрядах \Signs{1}{1}}, \\
        \DC{X},              &\text{если в знаковых разрядах \Signs{0}{0}}.
    \end{cases}
\]

Всегда справедливо \[\DC{X}+\DC{Y}=\DC{X+Y}.\]

Следует отметить, что знаковые разряды участвуют в сложении наравне со всеми остальными. Т.е. компьютер, на самом деле складывая дополнительные коды, складывает их как обычные натуральные числа и о существовании знаковых разрядов <<даже не догадывается>>. 

\begin{exampl}[Задача]
    Используя дополнительный код, сложить числа $235$ и $-156$ в двоичной системе счисления. Разрядная сетка $n=9$.
\end{exampl}
\begin{proof}[Решение]
    \[
    \begin{array}[c]{c}
        235=(11101011)_2,\\
        -156=(-10011100)_2,\\
        \DC{235}=(\Sign{0}11101011)_2,\\
        \DC{-156}=(\Sign{1}01100100)_2,\\
        \DC{235}+\DC{-156}=(\Signs{0}{0}11101011)_2+(\Signs{1}{1}01100100)_2=(\Signs{0}{0}01001111)_2,\\
        \DC{235-156}=(\Sign{0}01001111)_2,\\
        235-156=(1001111)_2=79.
    \end{array}
    \]
\end{proof}

\begin{exampl}[Задача]
    Используя дополнительный код, сложить числа $-235$ и $156$ в двоичной системе счисления. Разрядная сетка $n=9$.
\end{exampl}
\begin{proof}[Решение]
    \[
        \begin{array}[c]{c}
            -235=(-11101011)_2,\\
            156=(10011100)_2,\\
            \DC{-235}=(\Sign{1}00010101)_2,\\
            \DC{156}=(\Sign{0}10011100)_2,\\
            \DC{-235}+\DC{156}=(\Signs{1}{1}00010101)_2+(\Signs{0}{0}10011100)_2=(\Signs{1}{1}10110001)_2,\\
            \DC{-235+156}=(\Sign{1}10110001)_2,\\
            |-235+156|=(\Sign{0}01001111)_2=79,\\
            -235+156=(-1001111)_2=-79.
        \end{array}
    \]
\end{proof}

Приведем напоследок пример сложения, дающего ошибочный результат (переполнение разрядной сетки).
\begin{exampl}[Задача]
    Используя дополнительный код, сложить числа $-235$ и $-156$ в двоичной системе счисления. Разрядная сетка $n=9$.
\end{exampl}
\begin{proof}[Решение]
    \[
        \begin{array}[c]{c}
            -235=(-11101011)_2,\\
            -156=(-10011100)_2,\\
            \DC{-235}=(\Sign{1}00010101)_2,\\
            \DC{-156}=(\Sign{1}01100100)_2,\\
            \DC{-235}+\DC{-156}=(\Signs{1}{1}00010101)_2+(\Signs{1}{1}01100100)_2=(\Signs{1}{0}01111001)_2,\\
            \text{Признак ошибки: $\Signs{1}{0}$},\\
            \DC{-235-156}=(\Sign{0}01111001)_2,\\
            \text{ОШИБКА:} {-391}\neq 121
        \end{array}
    \]
\end{proof}

Рассмотрим еще один способ перехода от вычитания к сложению --- \emph{обратный} код. Как и ранее, в $n$ разрядном числе $(n-1)$-й разряд считается знаковым и $\Sign{0}$ соответствует знак плюс, а $\Sign{1}$ соответствует минус. На время вычислений знаковый разряд также дублируется с целью обнаружения переполнения разрядной сетки. Справедливо, что
\[
    \begin{array}[c]{c}
        -B=(2^m-1)-B+1 = (\underbrace{\Signs{1}{1}1\cdots11}_{m})_{2}-B+1=\overline{B}+1\\
        \overline{B}=-B-1.
    \end{array}
\]

$\overline{B}$ --- обратный код числа $-B$. Обратный код отрицательного числа --- это инверсия всех разрядов его модуля. Обратный код $X$ будем обозначать $\OC{X}$. 
\[
    \OC{X}=
    \begin{cases}
        \overline{|X|}, & \text{если $X<0$},\\
        |X|,            & \text{если $X\geq 0$}.
    \end{cases}
\]

Получить модуль числа из его представления в обратном коде очено просто:
\[
    |X|=
    \begin{cases}
        \overline{\OC{X}}, & \text{если в знаковом разряде $\Sign{1}$},\\
        \OC{X},            & \text{если в знаковом разряде $\Sign{0}$}.
    \end{cases}
\]

В $n$-разрядной сетке диапазон изменения представимого $X$ следующий\footnote{Диапазон представления меньше, чем в дополнительном коде потому что в обратном коде по определению получается, что $\OC{0}\neq \OC{-0}$}:

\[-2^{n-1}+1\leq X\leq 2^{n-1}-1.\]

Обратный код прост в получении\footnote{По крайней мере по сравнению с дополнительным. В стародавние времена эта простота сильно уменьшала аппаратные затраты на создание вычислительного устройства и экономила потребляемую им энергию}, но в ряде случаев результат сложения обратных кодов чисел будет получен не в обратном коде. Т.е. не всегда справедливо
\[
    \OC{X}+\OC{Y}=\OC{X+Y}.
\]

Могут возникнуть следующие случаи, требующие поправок ($A,B$ --- положительные).

\begin{enumerate}
    \item $A+B$
    \[\OC{A}+\OC{B}=A+B.\]
    
    В этом случае результат представлен в обратном коде верно. Переноса из знаковых разрядов нет.

    \item $A-B$
    \[\OC{A}+\OC{-B}=A+\bar{B}=A-B-1.\]
    В этом случае возможны два варианта.
    
    \begin{enumerate}
    \item $A-B<0$, результат верно представлен в обратном коде $(A-B)-1$. Признаком тому может служить отсутствие переноса из знаковых разрядов.

    \item $A-B>0$, результат неверен (должен быть $A-B$). Признаком тому может служить перенос из знаковых разрядов. Чтобы получить верный результат в обратном коде, к полученному числу нужно прибавить единицу.
    \end{enumerate}

    \item $-A-B$
    \[\OC{-A}+\OC{-B}=-A-1-B-1.\]
    Результат получен неверно (должен быть $(-A-B)-1$). Признаком тому может служить тот же перенос из знаковых разрядов. Чтобы получить верный результат в обратном коде к полученному числу нужно прибавить единицу.
\end{enumerate}

К полученному в результате сложения обратных кодов числу нужно прибавить перенос из знаковых разрядов. Так же как и в дополнительном коде, признаком ошибки переполнения служит комбинаця $\Signs{1}{0}$ или $\Signs{0}{1}$ в знаковых разрядах.

\begin{exampl}[Задача]
    Используя обратный код, сложить числа $235$ и $-156$ в двоичной системе счисления. Разрядная сетка $n=9$.
\end{exampl}
\begin{proof}[Решение]
    \[
        \begin{array}[c]{c}
            235=(11101011)_2,\\
            -156=(-10011100)_2,\\
            \OC{235}=(\Sign{0}11101011)_2,\\
            \OC{-156}=(\Sign{1}01100011)_2,\\
            \OC{235}+\OC{-156}=
                (\Signs{0}{0}11101011)_2+
                (\Signs{1}{1}01100011)_2=
                (\xleftarrow{1}\Signs{0}{0}01001110)_2,\\
            \OC{235-156}=(\Sign{0}01001110)_2+1=(\Sign{0}01001111)_2,\\
            235-156=(\Sign{0}01001111)_2=79.
        \end{array}
    \]
\end{proof}

\begin{exampl}[Задача]
    Используя обратный код, сложить числа $-235$ и $156$ в двоичной системе счисления. Разрядная сетка $n=9$.
\end{exampl}
\begin{proof}[Решение]
    \[
        \begin{array}[c]{c}
            -235=(-11101011)_2,\\
            156=(10011100)_2,\\
            \OC{-235}=(\Sign{1}00010100)_2,\\
            \OC{156}=(\Sign{0}10011100)_2,\\
            \OC{-235}+\OC{156}=(\Signs{1}{1}00010100)_2+(\Signs{0}{0}10011100)_2=(\Signs{1}{1}10110000)_2,\\
            \OC{-235+156}=(\Sign{1}10110000)_2,\\
            |-235+156|=\left(\overline{\Sign{1}10110000}\right)_2=(\Sign{0}01001111)_2,\\
            -235+156=(-1001111)_2=-79.
        \end{array}
    \]
\end{proof}

Напоследок пример сложения, дающего ошибочный результат (переполнение разрядной сетки).
\begin{exampl}[Задача]
    Используя обратный код, сложить числа $-235$ и $-156$ в двоичной системе счисления. Разрядная сетка $n=9$.
\end{exampl}
\begin{proof}[Решение]
    \[
        \begin{array}[c]{c}
            -235=(-11101011)_2,\\
            -156=(-10011100)_2,\\
            \OC{-235}=(\Sign{1}00010100)_2,\\
            \OC{-156}=(\Sign{1}01100011)_2,\\
            \OC{-235}+\OC{-156}=
                (\Signs{1}{1}00010100)_2+
                (\Signs{1}{1}01100011)_2=
                (\xleftarrow{1}\Signs{1}{0}01110111)_2,\\
            \text{Признак ошибки: $\Signs{1}{0}$},\\
            \OC{-235-156}=(\Sign{0}01110111)_2+1=(\Sign{0}01111000)_2,\\
            \text{ОШИБКА:} {-391}\neq 120.
        \end{array}
    \]
\end{proof}

Особенностью двоичной системы являетя то, что цифры (0, 1) можно трактовать как биты информации (истина, ложь). Преобразования двоичных чисел тесно связаны с логическими операциями. Это, а также то, что двоичная техника надежна (нужно различать лишь два устойчивых состояния среды) и обусловило то, что двоичная система счисления используется в подавляющем большинстве вычислительных устройств.


\section{Восьмиричная и шестнадцатиричная системы счисления}

Системы, основание которых есть степень двух: 
\[
    8=2^3,\, 16=2^4.
\]

Эти системы активно используются на практике, так как облегчают работу с двоичными числами. Перевод числа в представление в этих системах счисления не имеет особенностей.

Рассмотрим преставление числа $X$ в двоичной системе, разбив его, начиная от точки, на группы по три цифры:
\[
    X=
        \begin{array}[c]{c}
            \ldots+\\+
            a_{8}\cdot 2^{8} +
            a_{7}\cdot 2^{7} +
            a_{6}\cdot 2^{6} +\\+

            a_{5}\cdot 2^{5} +
            a_{4}\cdot 2^{4} +
            a_{3}\cdot 2^{3} +\\+

            a_{2}\cdot 2^{2} +
            a_{1}\cdot 2^{1} +
            a_{0}\cdot 2^{0} +\\+

            \frac {a_{-1}}{2^{1}} +
            \frac {a_{-2}}{2^{2}} +
            \frac {a_{-3}}{2^{3}} +\\+

            \frac {a_{-4}}{2^{4}} +
            \frac {a_{-5}}{2^{5}} +
            \frac {a_{-6}}{2^{6}} +\\

            \frac {a_{-7}}{2^{7}} +
            \frac {a_{-8}}{2^{8}} +
            \frac {a_{-9}}{2^{9}} +\\

            +\ldots
    \end{array}
\]

Вынесем в каждой группе $8=2^3$ за скобки:
\[
    \begin{array}[c]{c}
        \ldots+\\+
        \left(
        a_{8}\cdot 2^{2} +
        a_{7}\cdot 2^{1} +
        a_{6}\cdot 2^{0}\right)\cdot 8^{2} +\\+

        \left(
        a_{5}\cdot 2^{2} +
        a_{4}\cdot 2^{1} +
        a_{3}\cdot 2^{0}\right)\cdot 8^{1} +\\+

        \left(
        a_{2}\cdot 2^{2} +
        a_{1}\cdot 2^{1} +
        a_{0}\cdot 2^{0}\right)\cdot 8^{0} +\\+


        \frac{\displaystyle
        a_{-1}\cdot 2^{2} +
        a_{-2}\cdot 2^{1} +
        a_{-3}\cdot 2^{0}
        }{8^{1}} +\\+


        \frac{\displaystyle
        a_{-4}\cdot 2^{2} +
        a_{-5}\cdot 2^{1} +
        a_{-6}\cdot 2^{0}
        }{8^{2}} +\\+

        \frac{\displaystyle
        a_{-7}\cdot 2^{1} +
        a_{-8}\cdot 2^{2} +
        a_{-9}\cdot 2^{3}
        }{8^{3}} +\\

        +\ldots
    \end{array}
\]

Получили запись числа:
\[
    X=\sum_{i=-m'}^{n'}b_{i}\cdot 8^{i},
\]
где 
\[
    b_{i}=\sum_{j=0}^{2}a_{3\cdot i + j}\cdot 2^{j}; a_{3\cdot i + j}\in[0,1]
\]

$b_i\in [0,2^{3}-1]$. Поставить трехразрядным двоичным числам цифры восьмиричной системы не представляет труда:
\[
    \begin{array}[c]{c|c|c|c|c|c|c|c|c|}
        \text{2-ичное число:}&000&001&010&011&100&101&110&111\\
        \hline
        \text{8-ичная цифра:}&0&1&2&3&4&5&6&7
    \end{array}
\]

Для представления восьмиричных чисел в некоторых языках программирования используются следующие соглашения:
\begin{itemize}
    \item С++, java, C\#, и т.д.: если справа перед числом записан ноль, то число в восьмиричной системе. 015720 - восьмиричное число (равное 7120). 0189 - ошибка: недопустимы цифры 8 и 9. Без ведущего нуля число считается десятичным;
    
    \item В ассемблере после цифр восьмиричного числа пишется латинская буква <<o>> (octal). 15720o. Ну и 189o --- надо ли говорить\ldots
\end{itemize}

Совершенно аналогично происходит перевод в шестнадцатиричную систему. При этом исходное двоичное число разбивается от точки на группы по четыре двоичные цифры. Числа от нуля до десяти обозначаются теми же символами, что и в десятичной системе, а оставшиеся числа от десяти до пятнадцати обозначаются латинскими буквами: $A$, $B$, $C$, $D$, $E$, $F$.

Для облегчения перевода тетрад (четырехразрядных двоичных чисел) в шестнадцатеричные цифры можно использовать следующую таблицу:
\[
    \begin{tabular}{lll|lll}
        \hline\hline
        16\text{-я CC} 
            &10\text{-я CC} 
                    & 2\text{-я CC}
                        & 16\text{-я CC} 
                            & 10\text{-я CC} 
                                & 2\text{-я CC}\\
        \hline\hline
        0   &0  &0000   &8  &8  &1000\\
        1   &1  &0001   &9  &9  &1001\\
        2   &2  &0010   &A  &10 &1010\\
        3   &3  &0011   &B  &11 &1011\\
        4   &4  &0100   &C  &12 &1100\\
        5   &5  &0101   &D  &13 &1101\\
        6   &6  &0110   &E  &14 &1110\\
        7   &7  &0111   &F  &15 &1111\\
        \hline
    \end{tabular}
\]

Для представления шестнадцатиричных чисел в некоторых языках программирования используются следующие соглашения.
\begin{itemize}
    \item С++, java, C\#, и т.д.: если слева от цифр цисла есть префикс <<0x>>, то число в шестнадцатеричной системе. 0xAF - шестнадцатеричное число (равное 175). 0x1h - ошибка: недопустима цифра h.
    
    \item В некоторых ассемблерах после цифр шестнадцатиричного числа пишется латинская буква <<h>> (hexadecimal): afh, AFh, AFH.
\end{itemize}

Так как перевод перевод в двоичную систему тривиален, то эти системы счисления (восьмиричная и шестнадцатиричная) активно применяются в языках программирования для повышения компактности записи программ. 
\begin{exampl} Компактность уменьшает количество ошибок:
    \[
        \begin{array}[c]{c}
            (11111110101000000001011111001101)_2=\\
            =(37650013715)_8=\\
            =(FEA017CD)_{16}
        \end{array}
    \]
    
    В короткой записи числа ошибиться сложнее.
\end{exampl}

\begin{exampl}[Задача]
    Дано двоичное число $(1110011.0101101)_2$. Перевести его в системы счисления с основанием $8$ и $16$.
\end{exampl}

\begin{proof}[Решение]
    \[
        \begin{array}[c]{c}
            (
            \underbrace{001}_{1}
            \underbrace{110}_{6}
            \underbrace{011}_{3}.
            \underbrace{010}_{2}
            \underbrace{110}_{6}
            \underbrace{100}_{4})_2=(163.264)_8=\\
            =(
            \underbrace{0111}_{7}
            \underbrace{0011}_{3}.
            \underbrace{0101}_{5}
            \underbrace{1010}_{A}
            )_2=(73.5A)_{16}
        \end{array}
    \]
\end{proof}

\begin{exampl}[Задача]
    Дано восьмиричное число $(673245.471)_8$. Перевести его в систему счисления c основанием $16$.
\end{exampl}

\begin{proof}[Решение]
    Переводим в двоичную систему, а из двоичной в шестнадцатеричную:
    \[
        \begin{array}[c]{c}
            (110\ 111\ 011\ 010\ 100\ 101.100\ 111\ 001)_2=\\
            =(0011\ 0111\ 0110\ 1010\ 0101.1001\ 1100\ 1000)_2=\\
            =(376A5.9C8)_{16}
        \end{array}
    \]
\end{proof}

\begin{exampl}[Задача]
    Дано десятичное число $65045.875$. Перевести его в счистему счисления c основанием $2$.
\end{exampl}
\begin{proof}[Решение]
    Переводим в шестнадцатеричную систему целую часть:
    \[
        \begin{array}[c]{l}
            65045=4065\cdot 16 + 5,\Rightarrow a_{0}=5, \\
            4065=254\cdot 16 + 1,\Rightarrow a_{1}=1, \\
            254=15\cdot 16 + 14,\Rightarrow a_{2}=E, \\
            15=0\cdot 16 + 15,\Rightarrow a_{3}=F.
        \end{array}
    \]

    Дробную часть:
    \[
        0.875\cdot 16=14.0,\Rightarrow a_{-1}=E.
    \]

    В двоичной системе:
    \[(FE15.E)_{16}=(1111111000010101.1110)_2.\]
\end{proof}

Данные системы счисления облегчают работу с двоичной системой для человека и представлены практически во всех языках, подходящих для системного программирования.


\section{Троичная симметричная система счисления}
\label{s:ch:ss:triplet}

Троичная система счисления, в которой цифрам соответствуют числа из сдвинутого на единицу влево диапазона $[0,2]$, т.е. из диапазона $[-1,1]$. Для удобства вводятся следующие обозначения цифр: цифре <<$n$>>(negative-отрицательное) соответствует число $-1$, цифре <<$0$>> соответствует число $0$, а цифре <<$p$>>(positive-положительное) соответствует число $1$.

\begin{exampl}[Положительное число] 
    \label{ex:ch:ss:positiveDigit}
    \[
        \begin{array}[c]{c}
            pnp0p=
            1\cdot 3^{4} +
            (-1)\cdot 3^{3} +
            1\cdot 3^{2} +
            0\cdot 3^{1} +
            1\cdot 3^{0} = \\
            = 81-27+9+0+1= \\
            = 64
        \end{array}
    \]
\end{exampl}

Если старшая значащая (ненулевая) цифа числа равна $p$, то число положительное, если $n$ --- отрицательное. Так как каждый разряд такого числа знаковый, то для того, чтобы сменить знак числа на противоположный нужно <<проинвертировать>> цифры числа, то есть заменить $n$ на $p$, а $p$ на $n$. Ноль остается нолём. 

\begin{exampl}[Отрицательное число из примера \ref{ex:ch:ss:positiveDigit}]
    \[
        \begin{array}[c]{c}
            npn0n=
            (-1)\cdot 3^{4} +
            1\cdot 3^{3} +
            (-1)\cdot 3^{2} +
            0\cdot 3^{1} +
            (-1) \cdot 3^{0} = \\
            = -81+27-9+0-1= \\
            = -64
        \end{array}
    \]
    Вычитание эквивалентно сложению с <<инвертированным>> вычитаемым.
\end{exampl}

Следует обратить внимание, что положительное вещественное число, меньшее единицы, может содержать $p$ в нулевом разряде:
\[
    \begin{array}[c]{c}
        p.npn=
        1\cdot 3^{0} +
        (-1)\cdot 3^{-1} +
        1\cdot 3^{-2} +
        (-1)\cdot 3^{-3} = \\
        = 1-\frac{1}{3}+\frac{1}{9}-\frac{1}{27} 
        = \frac{27-9+3-1}{27}
        = \frac{20}{27}=\\
        =0.(740),
    \end{array}
\]

\begin{exampl}[Несколько целых чисел в троичной симметричной системе]
    \[
        \begin{array}[c]{c|c|c|c|c|c|c|c|c|c|c}
            \cdots & -4 & -3 & -2 & -1 & 0  & +1 & +2 & +3 & +4  & \cdots\\
            \hline
            \cdots & nn & n0 & np & 0n & 00 & 0p & pn & p0 & pp & \cdots\\
        \end{array}
    \]
\end{exampl}

В приведенных ниже формулах \eqref{eq:ch:ss:addTriplet0}, \eqref{eq:ch:ss:addTripletn} и \eqref{eq:ch:ss:addTripletp} справа от стрелки $\Leftrightarrow$ приводится поясняющая таблица сложения.

Таблица сложения цифр в отсутствие переноса из предыдущего разряда (или при переносе равном $0$):
\begin{equation}
    \label{eq:ch:ss:addTriplet0}
    \begin{array}[c]{c|c|c|c|}
        \xleftarrow{0} & n & 0 & p \\
        \hline
        n & \xleftarrow{n}p & \xleftarrow{0}n & \xleftarrow{0}0 \\
        \hline
        0 & \xleftarrow{0}n & \xleftarrow{0}0 & \xleftarrow{0}p \\
        \hline
        p & \xleftarrow{0}0 & \xleftarrow{0}p & \xleftarrow{p}n \\
        \hline
    \end{array}
    \Leftrightarrow
    \begin{array}[c]{c|c|c|c|}
        \xleftarrow{0}  & -1 & 0 & +1 \\
        \hline
        -1 & -2 & -1 & 0 \\
        \hline
        0 & -1 & 0 & +1 \\
        \hline
        +1 & 0 & +1 & +2 \\
        \hline
    \end{array}
\end{equation}

Сложение при переносе равном $n$:
\begin{equation}
    \label{eq:ch:ss:addTripletn}
    \begin{array}[c]{c|c|c|c|}
        \xleftarrow{n} & n  & 0 & p \\
        \hline
        n & \xleftarrow{n}0 & \xleftarrow{n}p & \xleftarrow{0}n \\
        \hline
        0 & \xleftarrow{n}p & \xleftarrow{0}n & \xleftarrow{0}0 \\
        \hline
        p & \xleftarrow{0}n & \xleftarrow{0}0 & \xleftarrow{0}p\\
        \hline
    \end{array}
    \Leftrightarrow
    \begin{array}[c]{c|c|c|c|}
        \xleftarrow{-1}  & -1 & 0 & +1 \\
        \hline
        -1 & -3 & -2 & -1 \\
        \hline
        0 & -2 & -1 & 0 \\
        \hline
        +1 & -1 & 0 & +1\\
        \hline
    \end{array}
\end{equation}

Сложение при переносе равном $p$:
\begin{equation}
    \label{eq:ch:ss:addTripletp}
    \begin{array}[c]{c|c|c|c|}
        \xleftarrow{p}  & n & 0 & p \\
        \hline
        n & \xleftarrow{0}n & \xleftarrow{0}0 & \xleftarrow{0}p \\
        \hline
        0 & \xleftarrow{0}0 & \xleftarrow{0}p & \xleftarrow{p}n \\
        \hline
        p & \xleftarrow{0}p & \xleftarrow{p}n & \xleftarrow{p}0 \\
        \hline
    \end{array}
    \Leftrightarrow
    \begin{array}[c]{c|c|c|c|}
        \xleftarrow{+1} & -1 &  0 & +1 \\
        \hline
        -1 & -1 &  0 & +1 \\
        \hline
        0  & 0  & +1 & +2 \\
        \hline
        +1 & +1 & +2 & +3 \\
        \hline
    \end{array}
\end{equation}

Сложение чисел не имеет особенностей: числа выравниваются по точке и соответствующие цифры последовательно суммируются, начиная с младших разрядов, при этом учитывается перенос из предыдущего разряда.

\begin{exampl}[Задача]
    Сложить числа $A\equiv pn0np.00p$ и $B\equiv nn0nn.ppp$
\end{exampl}
\begin{proof}[Решение]
    Пользуясь таблицами сложения  \eqref{eq:ch:ss:addTriplet0}, \eqref{eq:ch:ss:addTripletn} и \eqref{eq:ch:ss:addTripletp}, находим результат:
    \[
        {\entrymodifiers={}
            {\xymatrix@=1pc{
                A&\equiv
                    &p&n&0&n&p&.&0&0&p\\
                B&\equiv
                    &n&n&0&n&n&.&p&p&p\\
                \xleftarrow{c}
                 &  &\xleftarrow{n}
                        & 
                            &\xleftarrow{n}
                                & 
                                    &\xleftarrow{p}
                                        &.
                                            &\xleftarrow{p}
                                                &\xleftarrow{p}
                                                    &\\
                A+B&\equiv
                    &n
                      &p\ar[ul]|-{5}
                        &n
                          &p\ar[ul]|-{4}
                            &p
                              &.
                                &n\ar[ull]|-{3}
                                  &n\ar[ul]|-{2}
                                    &n\ar[ul]|-{1}
            }}
        }
    \]
    Нулевые переносы не обозначены.
\end{proof}

Алгоритм перевода числа $A$ в троичную симметричную систему счисления из десятичной следующий. Пусть в десятичной системе счисления
\[
    A \equiv (\underbrace{a_{n}\cdots a_0}_{A_{int}}.\underbrace{a_{-1}\cdots a_{-m}}_{A_{frac}})_{10},
\]
а в троичной симметричной:
\begin{align*}
    A_{int}  &\equiv (b_{n'}\cdots b_0)_{\pm 3},\\
    A_{frac} &\equiv (b_0.b_{-1}\cdots b_{-m'})_{\pm 3}.
\end{align*}

\begin{enumerate}
    \item\label{en:ch:ss:tripletConvSign} Запоминается и отбрасывается знак десятичного числа.

    \item\label{en:ch:ss:tripletConvIntStart} Переводится целая часть десятичного числа.
    \begin{enumerate}
        \item $j=0, A_{int}^{(0)}=A_{int}$.

        \item \label{en:ch:ss:tripletConvDiv}
        Выполняется разложение $A_{int}^{(j)}$.
        \[
            A_{int}^{(j)}=3\cdot A_{int}^{(j+1)} + r_{j},
        \]
        где остаток $r_{j}\in[-1,1]$, а представление частного в троичной симметричной системе становится короче на один разряд (см. формулу \eqref{eq:ch:ss:intPart}):
        \[
            A_{int}^{(j+1)} = \sum_{i=j+1}^{n}b_{i}\cdot 3^{i-(j+1)}
        \]

        \item Фиксируется значение $j$-го разряда в представлении троичной симметричной системой счисления:
        \[
            b_j=
            \begin{cases}
                n, &\text{если $r_j=-1$},\\
                0, &\text{если $r_j=0$},\\
                p, &\text{если $r_j=1$}.
            \end{cases}
        \]
        
        \item $j=j+1$.
        Если $A_{int}^{(j)}\neq 0$, то перейти к шагу \ref{en:ch:ss:tripletConvDiv}.

        \item Найдены все цифры $b_{i}$ в представлении целой части.
    \end{enumerate}

    \item\label{en:ch:ss:tripletConvFloat} Переводится дробная часть числа $A_{frac}$. При этом предполагается, что количество цифр $m'$ в дробной части явно задано.
    \begin{enumerate}
        \item $j=1,A_{frac}^{(1)}=A_{frac}$. Пусть $D$ --- представление дробной части $A_{frac}$ в троичной симметричной системе. $D=0$

        \item\label{en:ch:ss:tripletConvFloatDiv}
        Умножая $A_{frac}^{(j)}$ на $3$ получаем $j$-й разряд обычной троичной системы в целой части: 
        \[
            3\cdot A_{frac}^{(j)}= A_{frac}^{(j+1)} + r_{-j},
        \]
        где целая часть результата $r_{-j}\in[0,2]$, а представление дробной части в троичной симметричной системе счисления становится кроче на один разряд (см. формулу \eqref{eq:ch:ss:floatPartAlt}):
        \[
            A_{frac}^{(j+1)} = \sum_{i=j+1}^{m}\frac{b_{-i}}{3^{i-j}}.
        \]

        Находится $(b_1b_0)_{\pm 3}$ как соответствующее представление в троичной симметричной системе для $r_{-j}$: 
        \[
            (b_1b_0)_{\pm 3}=
            \begin{cases}
                00, &\text{если $r_{-j}=0$},\\
                0p, &\text{если $r_{-j}=1$},\\
                pn, &\text{если $r_{-j}=2$}.
            \end{cases}
        \]
        
        \item К числу $D$ по правилам сложения троичных симметричных чисел прибавляетя $(b_1b_0)_{\pm 3}\cdot 3^{-j}$:
        \[
            D=D+(b_1b_0)_{\pm 3}\cdot 3^{-j}=D+\underbrace{0.0\cdots b_1b_0}_{\text{$j+1$ цифр}}.
        \]
        
        Умножению на $3^{-j}$ соответствует сдвиг представления на $j$ разрядов вправо (разряды слева заполняются нулями). 
        
        \item $j=j+1$. Если $j<m'$, то перейти к шагу \ref{en:ch:ss:tripletConvFloatDiv}.

        \item Найдено представление $D$ в троичной симметричной системе счисления дробной части десятичного числа $A_{frac}$.
    \end{enumerate}

    \item Представления целой и дробной частей (полученные в п. \ref{en:ch:ss:tripletConvIntStart} и \ref{en:ch:ss:tripletConvFloat} соответственно) складываются по правилам сложения троичных симметричных чисел.
    
    \item Результат получен. Если отброшенный в п. \ref{en:ch:ss:tripletConvSign} знак был знаком минус, то цифры результата следует <<проинвертировать>>.
\end{enumerate}

\begin{exampl}[Задача]
    Перевести число $-542.731$ в троичную симметричную систему счисления.
\end{exampl}

\begin{proof}[Решение]
    Перевод целой части $542$:
    \[
    \begin{array}[c]{llll}
    A_{int}^{(0)}=542\\
    542=181\cdot 3 - 1, &\Rightarrow r_{0}=-1,  &A_{int}^{(1)}=181, &b_{0}'=n,\\
    181=60\cdot 3 + 1,  &\Rightarrow r_{1}=1,   &A_{int}^{(2)}=60,  &b_{1}'=p,\\
    60=20\cdot 3 + 0,   &\Rightarrow r_{2}=0,   &A_{int}^{(3)}=20,  &b_{2}'=0,\\
    20=7\cdot 3 - 1,    &\Rightarrow r_{3}=-1,  &A_{int}^{(4)}=7,   &b_{3}'=n,\\
    7=2\cdot 3 + 1,     &\Rightarrow r_{4}=1,   &A_{int}^{(5)}=2,   &b_{4}'=p,\\
    2=1\cdot 3 - 1,     &\Rightarrow r_{5}=2,   &A_{int}^{(6)}=1,   &b_{5}'=n,\\
    1=0\cdot 3 + 1,     &\Rightarrow r_{6}=1,   &A_{int}^{(7)}=0,   &b_{6}'=p
    \end{array}
    \]

    $542 \equiv pnpn0pn$.

    Перевод дробной части $0.731$ (предполагая точность $m=11$):
    \[
        \begin{array}[c]{l}
            \hline
            A_{frac}^{(1)}=0.731, \\
            D=0,\\ 
            \hline
        \end{array}
    \]
    \[
        \begin{array}[c]{l}
            \hline
            A_{frac}^{(1)}\cdot 3=0.731\cdot 3=2.193, \Rightarrow a_{-1}=2, Y^{(2)}=0.193, \\ 
            D=0+p.n=p.n,\\ 
            \hline
        \end{array}
    \]
    \[
        \begin{array}[c]{l}
            \hline
            A_{frac}^{(2)}\cdot 3=0.193\cdot 3=0.579,\Rightarrow a_{-2}=0, Y^{(3)}=0.579, \\
            D=p.n+0=p.n,\\
            \hline
        \end{array}
    \]
    \[
        \begin{array}[c]{l}
            \hline
            A_{frac}^{(3)}\cdot 3=0.579\cdot 3=1.737,\Rightarrow a_{-3}=1, Y^{(4)}=0.737, \\
            D=p.n0+0.00p=p.n0p,\\
            \hline
        \end{array}
    \]
    \[
        \begin{array}[c]{l}
            \hline
            A_{frac}^{(4)}\cdot 3=0.737\cdot 3=2.211,\Rightarrow a_{-4}=2, Y^{(5)}=0.211, \\
            D=p.n0p+0.00pn=p.npnn,\\
            \hline
        \end{array}
    \]
    \[
        \begin{array}[c]{l}
            \hline
            A_{frac}^{(5)}\cdot 3=0.211\cdot 3=0.633,\Rightarrow a_{-5}=0, Y^{(6)}=0.633, \\
            D=p.npnn+0=p.npnn,\\
            \hline
        \end{array}
    \]
    \[
        \begin{array}[c]{l}
            \hline
            A_{frac}^{(6)}\cdot 3=0.633\cdot 3=1.899,\Rightarrow a_{-6}=1, Y^{(7)}=0.899, \\
            D=p.npnn+0.00000p=p.npnn0p,\\
            \hline
        \end{array}
    \]
    \[
        \begin{array}[c]{l}
            \hline
            A_{frac}^{(7)}\cdot 3=0.899\cdot 3=2.697,\Rightarrow a_{-7}=2, Y^{(8)}=0.697, \\
            D=p.npnn0p+0.00000pn=p.npnnpnn,\\
            \hline
        \end{array}
    \]
    \[
        \begin{array}[c]{l}
            \hline
            A_{frac}^{(8)}\cdot 3=0.697\cdot 3=2.091,\Rightarrow a_{-8}=2, Y^{(9)}=0.091, \\
            D=p.npnnpnn+0.000000pn=p.npnnpn0n,\\
            \hline
        \end{array}
    \]
    \[
        \begin{array}[c]{l}
            \hline
            A_{frac}^{(9)}\cdot 3=0.091\cdot 3=0.273,\Rightarrow a_{-9}=0, Y^{(10)}=0.273, \\
            D=p.npnnpn0n+0=p.npnnpn0n,\\
            \hline
        \end{array}
    \]
    \[
        \begin{array}[c]{l}
            \hline
            A_{frac}^{(10)}\cdot 3=0.273\cdot 3=0.819,\Rightarrow a_{-10}=0, Y^{(11)}=0.819, \\
            D=p.npnnpn0n+0=p.npnnpn0n,\\
            \hline
        \end{array}
    \]
    \[
        \begin{array}[c]{l}
            \hline
            A_{frac}^{(11)}\cdot 3=0.819\cdot 3=2.457,\Rightarrow a_{-11}=2, Y^{(12)}=0.457, \\
            D=p.npnnpn0n+0.000000000pn=p.npnnpn0n0pn,\\
            \hline \\
            \cdots
        \end{array}
    \]

    Результат для дробной части получился не точным (поиск цифр мог продолжаться и далее): 
    \[p.npnnpn0n0pn \approx 0.7309974202.\] 
    
    Впрочем, погрешность достаточно мала.

    Для получения окончательного результата: 
    \[pnpn0pn+p.npnnpn0n0pn=pnpn0p0.npnnpn0n0pn,\]
    и так как знак исходного операнда отрицательный, то
    \[-542.731\approx npnp0n0.pnppnp0p0np.\]
\end{proof}

В заключение следует отметить, что троичная система счисления является оптимальной для представления чисел с точки зрения компактрности записи. Предпринимались попытки создания вычислительных систем не её основе\footnote{Например, русским ученым Николаем Петровичем Бруснецовым. Были спроектированы и созданы машины <<Сетунь>>, <<Сетунь-70>>. На машинах <<Сетунь>> решались задачи математического моделирования в физике и химии, оптимизации управления производством, краткосрочных прогнозов погоды, конструкторских расчетов, компьютерного обучения, обработки экспериментальных данных и т. д. К сожалению проект был закрыт, не выдержав конкуренции с двоичными вычислителями}.


\section*{Задания}
\addcontentsline{toc}{section}{Задания}


\paragraph{Задания базового уровня}

\begin{enumerate}
    \item Перевести в десятичную систему счисления числа:
    \[
        \begin{array}[c]{c|l}
            \hline\hline
            \text{Вариант}  &\text{Число}   \\
            \hline\hline
            1               &(10011.101)_2  \\
            2               &(1021.12)_3    \\
            3               &np0n.pp        \\
            4               &(231.42)_5     \\
            5               &(357.34)_8     \\
            6               &(37A.FD)_{16}  \\
            \hline
        \end{array}
    \]
    \ProofAnswer{
        \[
            \begin{array}[c]{c|l|l}
                \hline\hline
                \text{Вариант}  &\text{Число}   &\text{Ответ}   \\
                \hline\hline
                1               &(10011.101)_2  &19.625         \\
                2               &(1021.12)_3    &34.(5)         \\
                3               &np0n.pp        &-19.(4)        \\
                4               &(231.42)_5     &66.88          \\
                5               &(357.34)_8     &1863.4375      \\
                6               &(37A.FD)_{16}  &14090.9375     \\
                \hline
            \end{array}
        \]
    }
    
    \item Перевести из десятичной системы счисления числа в систему счисления с указанным основанием:
    \[
        \begin{array}[c]{|c|c|c|}
            \hline\hline
            \text{Вариант}&\text{Число}&\text{Основание} \\
            \hline\hline
            1&174.375&2\\
            \hline
            2&241.33&2\\
            \hline
            3&8713.31&3\\
            \hline
            4&9715.13&5\\
            \hline
            5&11579.13&7\\
            \hline
            6&65891.31&8\\
            \hline
            7&6791501.55&16\\
            \hline
            8&6791501.55 &17\\
            \hline
        \end{array}
    \]

    \item Перевести число из системы счисления с одним основанием в систему счисления с другим основанием:
    \[
        \begin{array}[c]{|c|c|c|}
            \hline\hline
            \text{Вариант}&\text{Число}&\text{Основание} \\
            \hline\hline
            1&(201011.121)_{3}&2\\
            \hline
            2&(402013.413)_{5}&3\\
            \hline
            3&(1100111010.1010111)_{2}&8\\
            \hline
            4&(1101100111010.101011001)_{2}&16\\
            \hline
            5&(AFE01.C7)_{16}&8\\
            \hline
            6&(72354.0137)_{8}&16\\
            \hline
            7&(1102112.02102)_{3}&9\\
            \hline
            8&(13587.158)_{9}&3\\
            \hline
        \end{array}
    \]

    \item Сравнить двоичные числа $x_1$ и $x_2$:
    \[
        \begin{array}[c]{|c|c|c|}
            \hline\hline
            \text{Вариант}  &x_1            &x_2\\
            \hline\hline
            1               &(101100)_2     &(101010)_2\\
            \hline
            2               &(0.1011101)_2  &(0.101111)_2\\
            \hline
            3               &(110.1011)_2   &(110.1010111)_2\\
            \hline
        \end{array}
    \]
    
    \item Сложить числа $A$ и $B$ в их исходных представлениях. Выполнить проверку в 10-й системе счисления.
    \[
        \begin{array}[c]{|c|c|c|}
            \hline\hline
            \text{Вариант}&A&B\\
            \hline\hline
            1&(10100111.110111)_{2}&(100101.101011)_{2}\\
            \hline
            3&(102211.2201)_{3}&(210220.12121)_{3}\\
            \hline
            2&(310342.3401)_{5}&(343143.3124)_{5}\\
            \hline
%            2&(.)_{7}&(.)_{7}\\
%            \hline
            3&(3BE0A.7EA3)_{15}&(72DDC.8BA8.)_{15}\\
            \hline
        \end{array}
    \]
    
    
    \item Сложить числа $A$ и $B$ в двоичной системе счисления (размер разрядной сетки выбрать самостоятельно). Использовать дополнительный или обратный код. Выполните проверку в десятичной системе счисления.
    \[
        \begin{array}[c]{|c|c|c|c|}
            \hline\hline
            \text{Вариант}&A&B&\text{Код}\\
            \hline\hline
            1&(10100111.110111)_2&(100101.101011)_2&\text{ДК}\\
            \hline
            2&(-11100111.110111)_2&(100101.10011)_2&\text{ОК}\\
            \hline
            3&(10100111.110111)_2&(-100101.10101)_2&\text{ДК}\\
            \hline
            4&(-10100111.110111)_2&(-100101.101011)_2&\text{ДК},\text{ОК}\\
            \hline
            5&(-1100111.010111)_2&(100101.00111)_2&\text{ОК}\\
            \hline
            6&(10100111.110111)_2&(110010.100011)_2&\text{ДК}\\
            \hline
        \end{array}
    \]

    \item Восстановить число в десятичной системе счисления если известно, что оно представлено в двоичной системе счисления в $8$-разрядной сетке в\ldots
    \[
        \begin{array}[c]{c|c|l}
            \hline\hline
            \text{Вариант}&\text{Число}&\text{\ldots задание}\\
            \hline\hline
            1 &\Sign{0}1010011 &\text{\ldots дополнительном коде.\LabeledAnswer{$83$}}\\
            2 &\Sign{0}0110111 &\text{\ldots обратном коде.\LabeledAnswer{$55$}}\\
            3 &\Sign{1}1101011 &\text{\ldots дополнительном коде.\LabeledAnswer{$21$}}\\
            4 &\Sign{1}1101011 &\text{\ldots обратном коде.\LabeledAnswer{$20$}}\\
            5 &\Sign{1}0110100 &\text{\ldots дополнительном коде.\LabeledAnswer{$76$}}\\
            4 &\Sign{1}0110011 &\text{\ldots обратном коде.\LabeledAnswer{$76$}}\\
            \hline
        \end{array}
    \]
    
    \item Сложить числа $A$ и $B$, представив их в двоичной системе, в $8$-ми разрядной сетке, в дополнительном коде. Выполнить сложение и проверить правильность результата вычисления (в десятичной системе счисления).
    \[
        \begin{array}[c]{c|c|c}
            \hline\hline
            \text{Вариант}&A&B\\
            \hline\hline
            1 & 92  & 45    \\
            2 & -26 & -97   \\
            3 & -25 & -106  \\
            4 & -72 & 110   \\
            5 & -85 & 75    \\
            \hline
        \end{array}
    \]
    \ProofAnswer{
        \[
            \begin{array}[c]{c|l}
                \hline\hline
                \text{Вар-т}  &\text{Результат} \\
                \hline\hline
                1               &\Signs{0}{0}1011100 + \Signs{0}{0}0101101 = \Signs{0}{1}0001001\text{ (92+45=ПРС!)}\\
                2               &\Signs{1}{1}1100110 + \Signs{1}{1}0011111 = \Signs{1}{1}0000101\text{ (-26-97=-123)}\\
                3               &\Signs{1}{1}1100111 + \Signs{1}{1}0010110 = \Signs{1}{0}1111101\text{ (-25-106=ПРС!)}\\
                4               &\Signs{1}{1}0111000 + \Signs{0}{0}1101110 = \Signs{0}{0}0100110\text{ (-72+110=38)}\\
                5               &\Signs{1}{1}0101011 + \Signs{0}{0}1001011 = \Signs{1}{1}1110110\text{ (-85+75=10)}\\
                \hline
            \end{array}
        \]
    }
    
    
    \item Сложить числа $A$ и $B$, представив их в $8$-ми разрядной сетке, в обратном коде. Выполнить сложение и проверить правильность результата вычисления (в десятичной системе счисления).
    \[
        \begin{array}[c]{c|c|c}
            \hline\hline
            \text{Вариант}  &A              &B              \\
            \hline\hline
            1               & (1110010)_2   & (101111)_2    \\
            2               & (-1010100)_2  & (-1100010)_2  \\
            3               & (100011)_2    & (-101010)_2   \\
            4               & (101011)_2    & (-100010)_2   \\
            5               & (1011010)_2   & (-1000)_2     \\
            \hline
        \end{array}
    \]
    \ProofAnswer{
        \[
            \begin{array}[c]{c|l}
                \hline\hline
                \text{Вар-т}    &\text{Результат} \\
                \hline\hline
                1               &\Signs{0}{0}1110010 + \Signs{0}{0}0101111 = \Signs{0}{1}0100001\text{ (114+47=ПРС!)}\\
                2               &\Signs{1}{1}0101011 + \Signs{1}{1}0011101 = \Signs{1}{0}1001000\text{ (84+98=ПРС!)}\\
                3               &\Signs{0}{0}0100011 + \Signs{1}{1}1010101 = \Signs{1}{1}1111000\text{ (35-42=-7)}\\
                4               &\Signs{0}{0}0101011 + \Signs{1}{1}1011101 = \Signs{0}{0}0001000+1\text{ (43-34=9)}\\
                5               &\Signs{1}{1}0100101 + \Signs{1}{1}1110111 = \Signs{1}{1}0011100+1\text{ (-90-8=-98)}\\
                \hline
            \end{array}
        \]
    }
    
\end{enumerate}

\paragraph{Задания повышенной сложности}
\begin{enumerate}
    \item Найти кратчайшей длины (т.е. состоящее из наименьшего количества цифр) двоичное число, принадлежащее интервалу:
    \[
        \begin{array}[c]{|c|c|}
            \hline\hline
            \text{Вариант}  &\text{Интервал}\\
            \hline\hline
            1               &(0.66006, 0.668065)\\
            \hline
            2               &(0.32021, 0.33604)\\
            \hline
            3               &(0.76542, 0.79708)\\
            \hline
        \end{array}
    \]

    \item Сложить в троичной симметричной системе счисления числа $A$ и $B$:
    \[
        \begin{array}[c]{|c|c|c|}
            \hline\hline
            \text{Вариант}&\text{A}&\text{B} \\
            \hline\hline
            1&npnpn0.npn&ppppnn.n0p\\
            \hline
            2&pnp0pn0.npp&p0p0pnn.00p\\
            \hline
            3&n0p0n0.ppn&pp0pnn.nnn\\
            \hline
        \end{array}
    \]

    \item Перевести из десятичной системы в троичную симметричную числа:
    \[
        \begin{array}[c]{|c|c|}
            \hline\hline
            \text{Вариант}&\text{Число}\\
            \hline\hline
            1&7000\frac{37}{81}\\
            \hline
            2&-\left(6000\frac{4}{243}\right) \\
            \hline
            3&-\left(6703\frac{7}{243}\right) \\
            \hline
        \end{array}
    \]
    
    \item Перевести из троичной в троичную симметричную систему числа:
    \[
        \begin{array}[c]{|c|c|}
            \hline\hline
            \text{Вариант}&\text{Число}\\
            \hline\hline
            1 &(2012.121)_3\\
            \hline
            2 &(1221.2201)_3\\
            \hline
            3 &(2212.0122)_3\\
            \hline
        \end{array}
    \]
    
    \item Разобраться с особенностями работы с числами на границах диапазона представления в дополнительном (обратном) коде. Например, отследить корректность результата, если:
    \begin{itemize}    
        \item один из операндов (или оба) находится на границе диапазона представления;
        \item результат сложения чисел попадает на границу диапазона.
    \end{itemize}
    Нужны ли дополнительные проверки на правильность результата, или способа с разрядом, дублирующим знаковый, достаточно?
    
    \item Разработайте правила работы в дополнительном коде для десятичной системы счисления.
    
    \item Разработайте правила работы в обратном коде для десятичной системы счисления.
    
    \item Разработать алгоритм перевода числа (целой и дробной части) в позиционную систему с \emph{отрицательным} основанием $K$. При этом цифры по прежнему принадлежат диапазону $[0, |K|-1]$.
\end{enumerate}

\paragraph{Философия}
\begin{enumerate}
    \item Разработать способ перевода чисел в систему счисления с основанием $K$, причем цифрам соответствуют числа из диапазона 
    \[[0-m,K-1-m]; 1\leq m \leq K-2.\] 

    Привести примеры.
    
    \item На основе своего представления, 
    \[
        (a_n\cdots a_1)_{k!}
    \]
    целое число в \emph{факториальной} системе счисления формируется так:
    \[
        X = \sum_{k=1}^{n}a_k\cdot{k!},
    \]
    где $0\leq a_k\leq k$, а $k!=1\cdot 2\cdot \ldots \cdot k$.
    
    Разработать алгоритм перевода числа в факториальное представление. Выполнить перевод числа $201$ в факториальную систему счисления. Рассчитать значение $(764321)_{k!}$.
    
    \item Возможно построение систем счисления, в которых цифры для $i$-го и для $j$-го ($i\neq j$) изменяются в различных пределах. То есть для каждого $i$-го разряда \emph{свое} жестко заданное основание $K_i$. Целая часть получается так:
    \[X=\sum_{i=1}^{n}a_{i}\cdot \left(\prod_{j=0}^{i-1} K_{j} \right)+a_0,\]
    где $a_i\in[0,K_{i}-1]$.

    Дробная часть получается так:
    \[X=\sum_{i=1}^{m}\frac{a_{-i}}{\displaystyle \prod_{j=1}^{i} K_{j}},\]
    где $a_{-i}\in[0,K_{i}-1]$.

    Например, если указано, что число представлено в системе счисления с основаниями <<2-4-7-5.2-10-3>> и дано число: 1354.191, то ему соответствует:
    \[
    \begin{array}[c]{c}
    X=1\cdot(4\cdot 7\cdot 5) + 3\cdot (7\cdot 5) + 5\cdot (5) + 4 + \frac{1}{2}+\frac{9}{2\cdot 10} + \frac{1}{2\cdot 10\cdot 3} =\\
    = 140 + 105 + 25 + 4 + 0.5 + 0.45 + 0,01(6) = \\
    = 274,9(6).
    \end{array}
    \]

    Разработайте обобщенный алгоритм перевода в такую систему счисления. Приведите примеры.

\end{enumerate}


\chapter{Бинарные отношения на множестве $A$}
\label{ch:bo}
%bo: label prefix
Приводятся сведения о бинарных отношениях с особыми свойствами. Бинарные отношения активно применяются на практике. Специальные же бинарные отношения достойны особой роли, благодаря тем свойствам, которыми они обладают. Для углубленного изучения рекомендуются: \cite{bib:novic:discrmathprogrammer,bib:sudoplatov:discrmath,bib:shaporev:discretemath}.


\section{Свойства бинарных отношений на множестве $A$}

Напомним, что бинарным отношением $P$ на множестве $A$ называется $P\subseteq A^2$. 

Отношение $P$ на множестве $A$ называется.
\begin{enumerate}
    \item \emph{Рефлексивным ($\rho$)}, если для всех $x\in A$ справедливо $(x,x)\in P$.

    \item \emph{Антарефлексивным}, если для всех $x\in A$ справедливо $(x,x)\not\in P$.

    \item \emph{Симметричным ($\sigma$)}, если для любых $x,y\in A$ из $(x,y)\in P$ следует $(y,x)\in P$. То есть $P^{-1}=P$.

    \item \emph{Антисимметричным}, если для любых $x,y\in A$ из $(x,y)\in P\land(y,x)\in P$ следует $x=y$. То есть $P\cap P^{-1}\subseteq I_A$.    
    
    \item \emph{Транзитивным ($\eta$)}, если для любых $x,y\in A$ из $(x,y)\in P\land (y,z)\in P$ следует $(x,z)\in P$. То есть $P\cdot P\subseteq P$.
    
    \item \emph{Полным} или \emph{линейным}, если для любых $x,y\in A$ из $x\neq y$ следует $(x,y)\in P\lor (y,x)\in P$. То есть $U_A=I_A\cup P\cup P^{-1}$.
\end{enumerate}
 
Если бинарное отношение $P$ на множестве $A$ не обладает тем или иным свойством, то его можно дополнить упорядоченными парами из $A^2$ до отношения $P^*$, которое нужным свойством обладает. Если полученное $P^*$:
\begin{enumerate}
    \item обладает нужным свойством $S$ (обозначим $S(P^*)$);
    \item содержит $P$ ($P^*\supset P$);
    \item является подмножеством любого другого отношения $P'$, содержащего $P$ и обладающего свойством $S$ ($S(P')\land (P'\supset P)\Rightarrow P^*\subset P'$),
\end{enumerate}
то оно называется \emph{замыканием отношения $P$ относительно свойства $S$}.

Замыкания находят массу применений на практике. Особенную практическую ценность имеют замыкания относительно транзитивности. Например:
\begin{itemize}
    \item если задан граф коммуникационной сети, то найденное замыкание относительно транзитивности даст ответ на вопрос: существует ли возможность передать сообщение из одного узла в другой. 
    
    \item если на группе людей задано бинарное отношение <<родитель>>, то замыкание относительно транзитивности даст отношение <<предок>>.
\end{itemize}
    
\begin{exampl}
    На множестве $\{1,2,3\}$ задано бинарное отношение 
    \[P=\{(1,1),(1,2),(1,3),(3,1),(2,3)\},\]
    которое не обладает свойствами рефлексивности, симметричности и транзитивности. Необходимо найти соответствующие замыкания.
    
    До обладания свойством рефлексивности нужно добавить 2 пары: \[\{(2,2),(3,3)\}.\]
    
    До обладания свойством симметричности нужно добавить также 2 пары: \[\{(2,1),(3,2)\}.\]
    
    Процесс поиска замыкания относительно транзивности состоит из нескольких шагов.
    \begin{enumerate}
        \item Для пар $(3,1)$ и $(1,2)$ нужна пара $(3,2)$.
        \item Для пар $(2,3)$ и $(3,1)$ нужна пара $(2,1)$.\label{enumer:transClosureEx1}
        \item Для пар $(3,1)$ и $(1,3)$ нужна пара $(3,3)$.
        \item Для пар $(2,1)$ и $(1,2)$ нужна пара $(2,2)$. Внимание:пара $(2,1)$ добавилась на шаге \ref{enumer:transClosureEx1}.
    \end{enumerate}
    \qed
\end{exampl}

Близость $\Delta(P,S)$ бинарного отношения $P$ к некоторому свойству $S$ можно оценивать количеством \emph{добавленных} или \emph{удаленных} пар (см. рис. \ref{fig:nearnessOfBinRelations}).

\begin{figure}
    \centering
    \begin{tabular}{cccc}
        {\xymatrix{
            a \ar@{->}[d] \ar@{->}[r]
                &d \ar@{->}[d]
                    \\
            b \ar@{->}[r]
                &c
        }}
            &
            {\xymatrix{
                a \ar@{->}[d] \ar@{->}[r] \ar@{.>}@(l,u)[]
                    &d \ar@{->}[d] \ar@{.>}@(u,r)[]
                        \\
                b \ar@{->}[r] \ar@{.>}@(d,l)[]
                    &c \ar@{.>}@(r,d)[]
            }}
                &
                {\xymatrix{
                    a \ar@{->}[d] \ar@{->}[r]
                        &d \ar@{->}[d] \ar@{.>}@/_/[l] 
                            \\
                    b \ar@{->}[r] \ar@{.>}@/^/[u] 
                        &c \ar@{.>}@/^/[l] \ar@{.>}@/_/[u] 
                }}
                    &
                    {\xymatrix{
                        a \ar@{->}[d] \ar@{->}[r] \ar@{.>}[dr]
                            &d \ar@{->}[d]
                                \\
                        b \ar@{->}[r]
                            &c
                    }}
                \\
                &&&\\
            Исходное $P$ 
                & Рефлексивное($\rho$) 
                    & Симметричное($\sigma$) 
                        & Транзитивное($\eta$)\\
                & $\Delta(P,\rho)=4$ 
                    & $\Delta(P,\sigma)=4$ 
                        & $\Delta(P,\eta)=1$ 
    \end{tabular}
    \caption{Близость бинарных отношений к свойствам}
    \label{fig:nearnessOfBinRelations}
\end{figure}

Поиск транзитивных замыканий отношений представляет наибольшую сложность. Человек может найти транзитивное замыкание, глядя на орграф отношения, что неприемлемо для машины. Приведенный псевдокод для алгоритма Уоршалла (см. псевдокод \ref{alg:bo:warshall}) позволяет найти транзитивное замыкание отношения, представленного матрицей смежности. 

Обоснование данного алгоритма на соответствующем отношению $P$ графе следующее. Первой итерацией внешнего цикла ($k=1$) к исходному графу будут добавлены транзитивные дуги, замыкающие путь через $a_1$ ($P\subseteq A^2, a_i\in A$). $k$-й итерацией будут добавлены транзитивные дуги, замыкающие путь через <<пройденные транзитом>> вершины $a_1,\ldots,a_k$. Последней итерацией $k=n$ будут добавлены дуги, проходящие транзитом через \emph{любую} последовательность из $n$ вершин графа.

\begin{algorithm}
    \caption{Поиск транзитивного замыкания $P$ (алгоритм Уоршалла)}\label{alg:bo:warshall}
    \begin{algorithmic}[1]
        \REQUIRE{$[P]_{n\times n}$ --- матрица смежности отношения $P$}
        \ENSURE{$[T]_{n\times n}$ --- матрица транзитивного замыкания отношения $P$}
        \STATE{$[T]\gets [P]$}
        \FOR{$k=1$ to $n$}
            \FOR{$i=1$ to $n$}
                \FOR{$j=1$ to $n$}
                    \STATE{$[T]_{i,j}\gets\big([T]_{i,j}\lor ([T]_{i,k}\land[T]_{k,j})\big)$}
                \ENDFOR
            \ENDFOR
        \ENDFOR
    \end{algorithmic}
\end{algorithm}
    
    
\section{Специальные бинарные отношения на множестве $A$}

Некоторые бинарные отношения имеют большое практическое значение. Следует изучить их подробнее.


\subsection{Отношение эквивалентности}
 

Бинарное отношение на множестве $A$, обладающее свойствами рефлексивности ($\rho$), симметричности ($\sigma$) и транзитивности ($\eta$), называется отношением \emph{эквивалентности}\footnote{Это отношение уже упоминалось при обсуждении темы мощности множеств} и обозначается $\sim$ или $\equiv$. Эквивалентность является  обобщением равенства\footnote{Равенство обладает ярко выраженной рефлексивностью --- это тождественное отношение $I_A$. $(x,x)\in =$. Но видно, что и в симметричности с транзитивностью ему не откажешь\ldots Равенство --- частный случай эквивалентности}.

\emph{Классом эквивалентности} $E(x)$ элемента $x\in A$ называется множество всех элементов $y\in A$, каждый из которых находится в отношении эквивалентности $E$ с элементом $x$:
\[E(x)=\{y|x\,E\,y, x\in A, y\in A\}\]

Множество, обозначаемое $A/E$:
\[
    A/E=\{E(x)|x\in A\},
\]
называется \emph{фактор-множеством} множества $A$ по отношению эквивалентности $E$.

\begin{Theor}
    Всякое отношение эквивалентности $E$ на множестве $A$ определяет \emph{разбиение}, которым явлеется \emph{фактор-множество} $A/E$. И обратно: всякое разбиение 
    \[
        \mathcal{R}=\{A_i|i\in \mathbb{N}, A_i,\subseteq A,(i\neq j)\Rightarrow (A_i\cap A_j=\emptyset)\},A=\bigcup_{i\in\mathbb{N}} A_i
    \]
    множества $A$, не содержащее пустых элементов, определяет отношение эквивалентности $E$ на $A$ по правилу \begin{equation}
        x\,E\,y\Leftrightarrow x,y\in A_i.
        \label{eq:bo:equivByR}
    \end{equation}
\end{Theor}
\begin{proof}
    Так как $E$ рефлексивно, то $x\in E(x)$ для любого $x\in A$. Отсюда следует, что каждое множество из $A/E$ непусто и $\bigcup_{x\in A}E(x)=A$. Чтобы доказать, что $A/E$ является разбиением достаточно доказать, что если $E(x)\cap E(y)\neq\emptyset$, то $E(x)=E(y)$.
    
    Покажем, что $E(x)\subseteq E(y)$ и $E(y)\subseteq E(x)$ при $E(x)\cap E(y)\neq\emptyset$. Пусть $z\in E(x)\cap E(y)$. Докажем $E(x)\subseteq E(y)$. Возьмем $k\in E(x)$, тогда справедливо $k\,E\,z$, $z\,E\,y$ и, следовательно, $k\,E\,y$. Если $k\,E\,y$, то $k\in E(y)$. Стало быть $k\in E(x)\Rightarrow k\in E(y)$, а значит $E(x)\subseteq E(y)$. Аналогично докажем, что $E(y)\subseteq E(x)$.\qed
    
    Теперь докажем обратное утверждение теоремы. Пусть имеется разбиение $\mathcal{R}=\{A_i\}$. Рефлексивность и симметричность $E$, определяемого формулой \eqref{eq:bo:equivByR} очевидны. Докажем транзитивность. $x\,E\,y$ справедливо при $x,y\in A_i$, $y\,E\,z$ --- при $y,z\in A_j$. Но раз $y\in A_i$ и $y\in A_j$, то $A_i=A_j$. Тогда справедливо $x,z\in A_i$ и $x\,E\,z$.
\end{proof}

В любом классе $E(x)$ эквивалентности $E$ каждый элемент $y\in E(x)$ связан отношением $E$ с любым $z\in E(x)$. Поэтому, если 
\[
    A/E=\{\{a^1_1,\ldots,a^1_{m_1}\},\{a^2_1,\ldots,a^2_{m_2}\},\ldots,\{a^n_1,\ldots,a^n_{m_n}\}\}
\]
и элементы множества $A$ упорядочены так:
\[
    a^1_1,\ldots,a^1_{m_1},a^2_1,\ldots,a^2_{m_2},\ldots,a^n_1,\ldots,a^n_{m_n},
\]
то матрица смежности отношения $E$ имеет блочно - диагональный вид:
\[[E]=
    \begin{array}{c|cccc}
        E
            &a^1_1\cdots a^1_{m_1} 
                & a^2_1\cdots a^2_{m_2} 
                    & \cdots 
                        & a^n_1\cdots a^n_{m_n}
                            \\ 
        \hline
        \begin{matrix}a^1_1\\ \vdots \\ a^1_{m_1}\end{matrix} 
            &\begin{array}{|ccc|}\hline 1&\cdots&1\\ \vdots & \ddots & \\ 1 & & 1\\ \hline\end{array}
                &
                    &
                        &
                            \\
        \begin{matrix}a^2_1\\ \vdots \\ a^2_{m_2}\end{matrix} 
            &
                &\begin{array}{|ccc|}\hline 1&\cdots&1\\ \vdots & \ddots & \\ 1 & & 1\\ \hline\end{array}
                    &
                        & 
                            \\
        \vdots
            &
                &
                    &\ddots
                        & 
                            \\
        \begin{matrix}a^n_1\\ \vdots \\ a^n_{m_n}\end{matrix} 
            &
                &
                    &
                        &\begin{array}{|ccc|}\hline 1&\cdots&1\\ \vdots & \ddots & \\ 1 & & 1\\ \hline\end{array}
    \end{array},
\]
где квадратные непересекающиеся блоки на главной диагонали состоят из единиц, а остальные элементы равны нулю.

Если множество $A$ таким образом не упорядочено, то соответствующая матрица смежности $[E]$ приводится к блочно-диагональному виду одновременными перестановками строк и столбцов. Элементы $a_i$ и $a_j$ эквивалентны тогда и только тогда, когда $i$-я и $j$-я строки (а также столбцы) матрицы $[E]$ совпадают. Класс эквивалентности $E(a_i)$ состоит из элементов $a_j$, для которых $[E]_{ij}=1$.

\begin{exampl}
    Задача. Пусть имеется, например, численный рассчет, представленный орграфом на рисунке \ref{fig:bo:calcFlowEx}. Исходным данным соответствует $S$, конечному результату --- $R$. Вершинам графа соответствуют операции $O_i$. Дуге, соединяющей операции $O_i$ и $O_j$ соответствует численный результат $r_i$ полученный на выходе операции $O_i$ и подаваемый на вход операции $O_j$. Так как исходные данные (или их часть) для операции $O_j$ вычисляются операцией $O_i$, то $O_j$ всегда выполняется во времени \emph{позже} $O_i$. Операции, представленные на графе на одной вертикали, в общем случае могут быть выполнены параллельно. Необходимо минимизировать затраты памяти для хранения промежуточных результатов $r_i$, предполагая, что все они занимают одинаковый объем.
\end{exampl}

\begin{figure}
    \centering
    \[
    {\xymatrix{
        *{}
            &O_1\ar@{->}[rr]^{r_1}
                &*{}
                    &O_5 \ar@{->}[r]^{r_5}
                        &O_7 \ar@{->}[dr]^{r_7}
                            &*{}
                                &*{}
                \\
        *{S} \ar@{-->}[ur] \ar@{-->}[dr] \ar@{-->}[rr]
            &*{}
                &O_3 \ar@{->}[r]^{r_3} \ar@{->}[ur]^{r_3}
                    &O_6 \ar@{->}[rr]^{r_6}
                        &*{}
                            &O_8 \ar@{=>}[r]
                                &*{R}
                \\
        *{}
            &O_2 \ar@{->}[r]^{r_2} \ar@{->}[ur]^{r_2}
                &O_4 \ar@{->}[ur]^{r_4} \ar@{->}[urrr]^{r_4}
                    &*{}
                        &*{}
                            &*{}
                                &*{}
                \\
    }}
    \]
    \caption{Граф потока вычислений}
    \label{fig:bo:calcFlowEx}
\end{figure}

\begin{proof}[Решение]
    Решая задачу \emph{грубой силой и невежеством}\footnote{Сокращенно: ГСН-алгоритм.}, можно для хранения каждого промежуточного результата $r_i$ использовать отдельную ячейку $m_i$. Кроме того, для сохранения исходных данных и окончательного результата нужно еще две ячейки: $m_S$, $m_R$. См. не оптимизированный по памяти вариант программы в таблице \ref{table:bo:calcFlowProgramEx}. Итого 9 ячеек.
    
    Результат можно сделать лучше, заметив, что времена жизни некоторых результатов не пересекаются во времени! Это значит, что для них можно использовать одну ячейку. Например, использовать для хранения $r_1$ и $r_2$ одну ячейку нельзя, а вот $r_1$ и $r_5$ можно: $r_1$ после вычисления $r_5$ уже не нужен. Введем отношение $\not\perp$, означающее: <<времена жизни не пересекаются>>. Это отношение, очевидно:
    \begin{itemize}
        \item симметрично $(r_i\not\perp r_j)\Rightarrow (r_j\not\perp r_i)$; 
        \item рефлексивно $r_i\not\perp r_i$.
    \end{itemize}
    Увы, оно не транзитивно: например, $r_1\not\perp r_5$ и $r_5\not\perp r_3$, но $r_1$ и $r_3$ в отношении $\not\perp$ не находятся. Тем не менее, так как указанное отношение рефлексивно, то оно содержит по крайней мере тождественное отношение (которое есть отношение эквивалентности), а может быть и какое-то другое отношение эквивалентности.
    
    Построим матрицу смежности для отношения $\not\perp$:
    \[
        \begin{array}{c|ccccccccc}
            \not\perp
                & S &r_1&r_2&r_3&r_4&r_5&r_6&r_7& R\\ \hline
            S   & 1 & 0 & 0 & 1 & 1 & 1 & 1 & 1 & 1\\
            r_1 & 0 & 1 & 0 & 0 & 0 & 1 & 1 & 1 & 1\\
            r_2 & 0 & 0 & 1 & 1 & 1 & 1 & 1 & 1 & 1\\
            r_3 & 1 & 0 & 1 & 1 & 0 & 1 & 1 & 1 & 1\\
            r_4 & 1 & 0 & 1 & 0 & 1 & 0 & 0 & 0 & 1\\
            r_5 & 1 & 1 & 1 & 1 & 0 & 1 & 0 & 1 & 1\\
            r_6 & 1 & 1 & 1 & 1 & 0 & 0 & 1 & 0 & 1\\
            r_7 & 1 & 1 & 1 & 1 & 0 & 1 & 0 & 1 & 1\\
            R   & 1 & 1 & 1 & 1 & 1 & 1 & 1 & 1 & 1
        \end{array}
    \]
    
    Перестановкой строк и столбцов можно получить такую матрицу смежности:
    \[
        \begin{array}{c|ccccccccc}
            \not\perp
                & R &r_1&r_5&r_7&r_3&r_6& S &r_2&r_4\\\hline
            R   & 1 & 1 & 1 & 1 & 1 & 1 & 1 & 1 & 1\\
            r_1 & 1 & 1 & 1 & 1 & 0 & 1 & 0 & 0 & 0\\
            r_5 & 1 & 1 & 1 & 1 & 1 & 0 & 1 & 1 & 0\\
            r_7 & 1 & 1 & 1 & 1 & 1 & 0 & 1 & 1 & 0\\
            r_3 & 1 & 0 & 1 & 1 & 1 & 1 & 1 & 1 & 0\\
            r_6 & 1 & 1 & 0 & 0 & 1 & 1 & 1 & 1 & 0\\
            S   & 1 & 0 & 1 & 1 & 1 & 1 & 1 & 0 & 1\\
            r_2 & 1 & 0 & 1 & 1 & 1 & 1 & 0 & 1 & 1\\
            r_4 & 1 & 0 & 0 & 0 & 0 & 0 & 1 & 1 & 1
        \end{array}
    \]
    
    Видно, что легко выделяется (удалением пар) содержащееся в нем отношение эквивалентности с тремя классами $\{R,r_1,r_5,r_7\}$, $\{S,r_3,r_6\}$ и $\{r_2,r_4\}$:
    \[
        \begin{array}{c|ccccccccc}
            \not\perp
                & R &r_1&r_5&r_7&r_3&r_6& S &r_2&r_4\\\hline
            R   & 1 & 1 & 1 & 1 & 0 & 0 & 0 & 0 & 0\\
            r_1 & 1 & 1 & 1 & 1 & 0 & 0 & 0 & 0 & 0\\
            r_5 & 1 & 1 & 1 & 1 & 0 & 0 & 0 & 0 & 0\\
            r_7 & 1 & 1 & 1 & 1 & 0 & 0 & 0 & 0 & 0\\
            r_3 & 0 & 0 & 0 & 0 & 1 & 1 & 1 & 0 & 0\\
            r_6 & 0 & 0 & 0 & 0 & 1 & 1 & 1 & 0 & 0\\
            S   & 0 & 0 & 0 & 0 & 1 & 1 & 1 & 0 & 0\\
            r_2 & 0 & 0 & 0 & 0 & 0 & 0 & 0 & 1 & 1\\
            r_4 & 0 & 0 & 0 & 0 & 0 & 0 & 0 & 1 & 1
        \end{array}
    \]
    
    Данные результатов одного класса эквивалентности можно хранить в одной ячейке. Оказывается, можно использовать три ячейки: в $m_1$ хранить данные класса $\{R,r_1,r_5,r_7\}$, в $m_2$ --- $\{S,r_3,r_6\}$ и в $m_3$ --- $\{r_2,r_4\}$. Пример программы до и после оптимизации представлен в таблице \ref{table:bo:calcFlowProgramEx}. Полученные две программы, очевидно, имеют отличия, но они дают одинаковые результаты. Такие программы называют \emph{эквивалентными}. Конечно, эквивалентные программы могут отличаться не только затратами памяти, но и, например, скоростью получения результата.
    
    \begin{table}
        \centering
        \begin{tabular}{l||l}
            \hline\hline
            До оптимизации --- 9 я.п.      & После оптимизации --- 3 я.п.   \\
            \hline\hline
            $S \to m_S$                    &  $S\to m_2$                    \\ \hline
            $O_1(m_S)\to r_1 \to m_1$      &  $O_1(m_2)\to r_1 \to m_1$     \\
            $O_2(m_S)\to r_2 \to m_2$      &  $O_2(m_2)\to r_2 \to m_3$     \\ \hline
            $O_3(m_S,m_2)\to r_3 \to m_3$  &  $O_3(m_2,m_3)\to r_3 \to m_2$ \\
            $O_4(m_2)\to r_4 \to m_4$      &  $O_4(m_3)\to r_4 \to m_3$     \\ \hline
            $O_5(m_1,m_3)\to r_5 \to m_5$  &  $O_5(m_1,m_2)\to r_5 \to m_1$ \\
            $O_6(m_3,m_4)\to r_6 \to m_6$  &  $O_6(m_2,m_3)\to r_6 \to m_2$ \\ \hline
            $O_7(m_5)\to r_7 \to m_7$      &  $O_7(m_1)\to r_7 \to m_1$     \\ \hline
            $O_8(m_4,m_6,m_7)\to R \to m_R$&  $O_8(m_3,m_2,m_1)\to R\to m_1$\\ \hline
        \end{tabular}
        \caption{Эквивалентные программы с разными затратами памяти}
        \label{table:bo:calcFlowProgramEx}
    \end{table}    
\end{proof}


\subsection{Отношение порядка}

Отношение \emph{порядка} является обобщением отношения $\leq$, например, на натуральных числах. Выделяют несколько видов отношений порядка. В общем случае отношение порядка обозначается $\prec$, когда неважно, о каком его виде идет речь. Свойства всех отношений порядка таковы, что обратное отношение также является отношением порядка. Обратное отношение $\prec^{-1}$ обозначается $\succ$. Если на множестве $A$ задано некоторое отношение порядка $\prec$, то это обозначается $\langle A,\prec\rangle$.

\begin{enumerate}
    \item Отношение \emph{частичного} порядка рефлексивно, транзитивно и антисимметрично. Обозначается символом $\leq$, обратное отношение $\geq$. Множество $A$ над элементами которого задано отношение частичного порядка, называется \emph{частично упорядоченным множеством}. Пример частичного порядка --- отношение $\subseteq$ на булеане множества $M$ (см. рис. \ref{fig:bo:ordersEx})
    
    \item Отношение \emph{строгого} порядка транзитивно, антисимметрично и антирефлексивно. Обозначается символом $<$, а обратное символом $>$. То есть
    \[(x<y)\Leftrightarrow (x\leq y)\land(x\neq y).\]
    Примером строгого порядка явлеется отношение $<$ на $\mathbb{N}$ (см. рис. \ref{fig:bo:ordersEx}).
    
    \item Отношение \emph{линейного} порядка, представляет собой отношение частичного порядка, в котором отсутствуют несравнимые элементы. То есть для любых $x,y$ справедливо, что $x\leq y$ или $y\leq x$. То есть отношение рефлексивно, транзитивно, антисимметрично и полно. Множество $A$, над элементами которого задано отношение линейного порядка, называется \emph{линейно упорядоченным множеством}. Примером линейного порядка явлеется отношение $\leq$ на $\mathbb{N}$ (см. рис. \ref{fig:bo:ordersEx}).
\end{enumerate}

\begin{figure}
    \centering
    \begin{tabular}{ccc}
        {\xymatrix{
            *{}
                &\{1,2\} \ar@{->}@(ul,ur)[]
                    &*{}
                        \\
            \{1\}  \ar@{->}@(dl,ul)[] \ar@{->}[ur]
                &*{} 
                    &\{2\} \ar@{->}@(ur,dr)[] \ar@{->}[ul]
                        \\
            *{} 
                &\emptyset \ar@{->}@(dr,dl)[] \ar@{->}[ul] \ar@{->}[ur] \ar@{->}[uu]
                    &*{} 
        }}
            &
            {\xymatrix{
                *{}
                    &4 
                        &*{}
                            \\
                2  \ar@{->}[ur] \ar@{->}[rr] 
                    &*{} 
                        &3 \ar@{->}[ul]
                            \\
                *{} 
                    &1 \ar@{->}[ul] \ar@{->}[ur] \ar@{->}[uu]
                        &*{} 
            }}
                &
                {\xymatrix{
                    *{}
                        &4 \ar@{->}@(ul,ur)[]
                            &*{}
                                \\
                    2  \ar@{->}@(dl,ul)[] \ar@{->}[ur] \ar@{->}[rr] 
                        &*{} 
                            &3 \ar@{->}@(ur,dr)[] \ar@{->}[ul]
                                \\
                    *{} 
                        &1 \ar@{->}@(dr,dl)[] \ar@{->}[ul] \ar@{->}[ur] \ar@{->}[uu]
                            &*{} 
                }}
                    \\
        &&\\
        $\subseteq$ на $2^{\{1,2\}}$
            &$<$ на $\{1,2,3,4\}$
                & $\leq$ на $\{1,2,3,4\}$
                    \\
        Частичный порядок
            &Строгий порядок
                &Линейный порядок
    \end{tabular}
    \caption{Отношения порядка}
    \label{fig:bo:ordersEx}
\end{figure}    
    
Элемент $a\in A$ частично упорядоченного множества $A$ называется \emph{минимальным}, если для всех $x\in A$ из $x\leq a$ следует $a=x$. Элемент $a\in A$ частично упорядоченного множества $A$ называется \emph{максимальным}, если для всех $x\in A$ из $a\leq x$ следует $a=x$. Минимальных (максимальных) элементов в частично упорядоченном множестве может быть несколько и их не может не быть.

Элемент $a\in A$ частично упорядоченного множества $A$ называется \emph{наименьшим}, если для всех $x\in A$ справедливо $a\leq x$. Элемент $a\in A$ частично упорядоченного множества $A$ называется \emph{наибольшим}, если для всех $x\in A$ справедливо $x\leq a$. 
\begin{Theor}
    Частично упорядоченное множество содержит не более одного наименьшего (наибольшего) элемента.
\end{Theor}
\begin{proof}    
    Допустим, что в множестве более одного наименьшего элемента. Допустим, что $a_1$, $a_2$ --- два из этих элементов, тогда справедливо, что $a_1\leq a_2$ и $a_2\leq a_1$, стало быть $a_1=a_2$. Аналогичго можно доказать теорему для наибольшего элемента.
\end{proof}

В качестве следствия этой теоремы можно отметить, что наименьшего (наибольшего) элемента может и не быть, а также то, что наименьший (наибольший) элемент также будет минимальным (максимальным). См., например, рисунок \ref{fig:bo:minMaxEx}: наибольший (он же максимальный) элемент --- $c$, минимальные элементы --- $a,b$, наименьшего элемента нет.

Наименьший элемент частично упрядоченного множества $A$ обозначается $\min{A}$, а наибольший $\max{A}$.
\begin{figure}
    \centering
    \[
    {\xymatrix{
        a \ar@{->}@(dl,ul)[] \ar@{->}[r]
            &c \ar@{->}@(ur,dr)[]
                \\
        b  \ar@{->}@(dl,ul)[] \ar@{->}[ur]
            &*{}                 
    }}
    \]
    \caption{Частично упорядоченное множество $\langle\{a,b,c\},\leq\rangle$}
    \label{fig:bo:minMaxEx}
\end{figure}    

\emph{Нижней} (\emph{верхней}) гранью подмножества $B$ частично упорядоченного множества $A$ ($B\subseteq A$) называется элемент $a\in A$, такой что $a\leq b$ ($b\leq a$) для всех $b\in B$. Элемент $a\in A$ называется \emph{точной} нижней гранью (инфимумом $\inf{B}$) множества $B\subseteq A$, если $a$ --- \emph{наибольшая} из нижних граней множества $B$. Элемент $a\in A$ называется \emph{точной} верхней гранью (супремумом $\sup{B}$) множества $B\subseteq A$, если $a$ --- \emph{наименьшая} из верхних граней множества $B$. Например, для $B=[0,1)$, $B\subset\mathbb{R}$ справедливо $\inf{B}=0,\sup{B}=1$.

Линейный порядок $\leq$ на множестве $A$ назывется полным, если каждое непустое подмножество множества $A$ имеет наименьший элемент. В этом случае множество $A$ называется \emph{вполне упорядоченным}.

Говорят, что элемент $y$ \emph{покрывает} элемент $x$, если $x\leq y$ и не существует $z$, такого, что $x<z<y$. Конечное частично упорядоченное множество $\langle A,\leq\rangle$ можно представить в виде графа, в котором вершинами являются элементы $A$, и если $y$ покрывает $x$, то вершины $x,y$ соединяют линией, причем вершину $x$ располагают ниже вершины $y$. Такие схемы называются \emph{диаграммами Хассе}. Диаграмма Хассе получается из орграфа отношения удалением петель и транзитивно замыкающих дуг (при этом стрелки превращаются в линии).

Пример диаграммы Хассе для отношения $\subseteq$ на $2^{\{1,2,3\}}$ представлена на рисунке \ref{fig:bo:hasseOnBoolean}.

\begin{figure}
    \centering
    \[
        {\xymatrix{
            *{} 
                &\{1,2,3\}
                    &*{}
                        \\
            \{1,2\} \ar@{-}[ur]
                &\{1,3\} \ar@{-}[u]
                    &\{2,3\} \ar@{-}[ul]
                        \\
            \{1\} \ar@{-}[u] \ar@{-}[ur]
                &\{2\} \ar@{-}[ul] \ar@{-}[ur]
                    &\{3\} \ar@{-}[u] \ar@{-}[ul]
                        \\
            *{}
                &\emptyset \ar@{-}[ul] \ar@{-}[u] \ar@{-}[ur]
                    &*{}
            b  
                &*{}                 
        }}
    \]
    \caption{Диаграмма Хассе для отношения $\subseteq$ на $2^{\{1,2,3\}}$}
    \label{fig:bo:hasseOnBoolean}
\end{figure}    

\begin{exampl}
    \label{exampl:bo:hasseDivisible}
    Пусть \[A=\{0,1,2,3,4,5,6,7,8,9,10,11,12\}\]
    постройте диаграмму Хассе для отношения $x\preceq y$. Причем $x\preceq y$, если $x$ делит нацело $y$.
\end{exampl}
\begin{proof}[Решение]
    Найдем для каждого элемента множества элементов его покрывающих (см. таблицу \ref{table:bo:hasseOverhead}). Далее найдем минимальные элементы: $\{1\}$. Записываем их на одном уровне. Второй уровень образуют элементы, покрывающие элементы первого уровня, но не покрывающие элементов, которые не вошли в нижние уровни. И так далее. Результат представлен на рисунке \ref{fig:bo:hasseOnDiv}.
\end{proof}

\begin{table}
    \centering
    \begin{tabular}{c|c||c|c}
        \hline\hline
        Элемент $a\in A$ & Покрывающие $a$ & Элемент $a\in A$ & Покрывающие $a$ \\
        \hline\hline
        $0$ & $\emptyset$ 
                                & $7$ & $\{0\}$ \\
        $1$ & $\{2,3,5,7,11\}$ 
                                & $8$ & $\{0\}$ \\
        $2$ & $\{4,6,10\}$ 
                                & $9$ & $\{0\}$ \\
        $3$ & $\{6,9\}$ 
                                & $10$ & $\{0\}$ \\
        $4$ & $\{8,12\}$ 
                                & $11$ & $\{0\}$ \\
        $5$ & $\{10\}$ 
                                & $12$ & $\{0\}$ \\
        $6$ & $\{12\}$ 
                                &&\\
        \hline
    \end{tabular}
    \caption{Покрывающие элементы для примера \ref{exampl:bo:hasseDivisible}}
    \label{table:bo:hasseOverhead}
\end{table}

\begin{figure}
    \centering
    \[
        {\xymatrix{
            *{} 
                &*{}
                    &*{}
                        &0
                            &*{}
                                &*{}
                                    \\
            8 \ar@{-}[urrr]
                &12 \ar@{-}[urr]
                    &*{}
                        &*{}
                            &*{}
                                &*{}
                                    \\
            4 \ar@{-}[u]\ar@{-}[ur]
                &6 \ar@{-}[u]
                    &9 \ar@{-}[uur]
                        &10 \ar@{-}[uu]
                            &*{}
                                &*{}
                                    \\
            2 \ar@{-}[u] \ar@{-}[ur] \ar@{-}[urrr]
                &3 \ar@{-}[u] \ar@{-}[ur]
                    &*{}
                        &5 \ar@{-}[u]
                            &7 \ar@{-}[uuul]
                                &11 \ar@{-}[uuull]
                                    \\
            *{} 
                &*{}
                    &1 \ar@{-}[ull] \ar@{-}[ul] \ar@{-}[ur] \ar@{-}[urr] \ar@{-}[urrr]
                        &*{}
                            &*{}
                                &*{}
        }}
    \]
    \caption{Диаграмма Хассе для примера \ref{exampl:bo:hasseDivisible}}
    \label{fig:bo:hasseOnDiv}
\end{figure}    

Отношение порядка --- очень важное отношение. Например, вводя отношение порядка, можно значительно ускорить поиск элемента множества, а это весьма распространенная практическая задача (подробнее см. псевдокод \ref{alg:rec:binSearchIter}). Можно рекомендовать книгу \cite{bib:knuth:artOfProgramming3}, которая целиком посвящена вопросам сортировки и поиска.


%todo лексикографический порядок
%todo топологическая сортировка

\section*{Задания}
\addcontentsline{toc}{section}{Задания}

\begin{enumerate}

    \item Определите обладают ли следующие бинарные отношения свойствами рефлексивности ($\rho$), симметричности ($\sigma$) и транзитивности ($\eta$).
    \begin{enumerate}
        \item <<$x$ делит $y$ нацело>> на $\mathbb{N}$.
        \item <<$x\neq y$>> на $\mathbb{Z}$.
        \item <<$x+y$ нечётно>> на $\mathbb{Z}$.
        \item <<$x+y$ чётно>> на $\mathbb{Z}$.
        \item <<$x\cdot y$ нечётно>> на $\mathbb{Z}$.
        \item <<$x+x\cdot y$ чётно>> на $\mathbb{Z}$.
    \end{enumerate}

    \item Постройте бинарное отношение на множестве $\{a,b,c,d,e\}$:
    \begin{enumerate}
        \item полное, транзитивное и антисимметричное;
        \item рефлексивное, транзитивное и симметричное;
        \item рефлексивное, антисимметричное и не транзитивное;
        \item не рефлексивное, антисимметричное и транзитивное.
    \end{enumerate}

    \item Постройте транзитивные замыкания отношений $P,Q,R,S$:
    
    \begin{tabular}{cccc}
        {\xymatrix{
            a 
                &b \ar@{->}[l]
                    \\
            d \ar@{->}[r]
                &c \ar@{->}[u]
        }}
            &
            {\xymatrix{
                a \ar@{->}@/^/[r]
                    &b \ar@{->}@/^/[l]
                        \\
                d \ar@{->}[r]
                    &c \ar@{->}[u]
            }}
                &
                {\xymatrix{
                    a \ar@{->}[dr]
                        &b \ar@{->}[l]
                            \\
                    d \ar@{->}[r]
                        &c \ar@{->}[u]
                }}
                    &
                    {\xymatrix{
                        a \ar@{->}[d]
                            &b \ar@{->}[l]
                                \\
                        d \ar@{->}[r]
                            &c \ar@{->}[u]
                    }}
                \\
                &&&\\
            $P$ 
                & $Q$
                    & $R$
                        & $S$
    \end{tabular}
    
    \item Матрицами смежности заданы соответствующие бинарные отношения эквивалентности $P,Q,R$. Необходимо привести мартицы к блочно-диагональному виду и построить соответствующие графы.
    \[
        \begin{split}
            [P]=\begin{pmatrix}
                1&0&1&0&1\\
                0&1&0&1&0\\
                1&0&1&0&1\\
                0&1&0&1&0\\
                1&0&1&0&1
            \end{pmatrix},\\
            [Q]=\begin{pmatrix}
                1&0&1&1&0&0&1\\
                0&1&0&0&1&1&0\\
                1&0&1&1&0&0&1\\
                1&0&1&1&0&0&1\\
                0&1&0&0&1&1&0\\
                0&1&0&0&1&1&0\\
                1&0&1&1&0&0&1
            \end{pmatrix},
            [R]=\begin{pmatrix}
                1&1&0&0&0&1&0\\
                1&1&0&0&0&1&0\\
                0&0&1&0&1&0&0\\
                0&0&0&1&0&0&1\\
                0&0&1&0&1&0&0\\
                1&1&0&0&0&1&0\\
                0&0&0&1&0&0&1
            \end{pmatrix}.     
        \end{split}
    \]
    
    \item На множестве $A\subset\mathbb{N}$ задано отношение частичного порядка $\leq$ по правилу $x\leq y$, если $x$ делит $y$ нацело. Постройте диаграмму Хассе для множеств:
    \begin{enumerate}
        \item $A=\{2,3,4,9,27,36,72,108\}$;
        \item $A=\{2,3,6,12,18,36,72\}$;
        \item $A=\{2,3,5,4,6,10,15,16,24,30,32\}$.
    \end{enumerate}
    
    \item На множестве слов $A\subset T^*$ из букв русского (латинского) алфавита $T$ задано отношение частичного порядка $\leq$ по правилу: $x\leq y$, если слово $y$ содержит слово $x$. Например, слово <<радуга>> содержит слова <<дуга>>\footnote{<<Ра>> --- языческое:свет, солнце. Радуга --- дуга света.}. Постройте диаграмму Хассе для множества слов:
    \begin{enumerate}
        \item родина, родник, природа, род, народ, сродник, народник, уродина, урод.
        \item ограда, грань, дуга, град, ра, рада, рань, радуга, награда;
        \item ириска, рис, ирис, иск, риск, риска.
        \item ab, abba, ba, abb, a, bba.
    \end{enumerate}
    
    \item На множестве отрезков вещественных чисел задано отношение частичного порядка $\leq$ по правилу: $[x_1,y_1[\leq[x_2,y_2[$, если $(x_2\leq x_1)\land(y_1\leq y_2)$. Например, $[2,3]\leq[1,4]$. Постройте диаграмму Хассе для множества пар:
    \begin{enumerate}
        \item $A=\{[0,4]$, $[1,3]$, $[2,4]$, $[1,5]$, $[0,5]$, $[1,4]\}$;
        \item $A=\{[1,2]$, $[1,4]$, $[0,4]$, $[0,2]$, $[1,3]\}$;
        \item $A=\{[2,5]$, $[0,4]$, $[2,3]$, $[0,3]$, $[1,4]$, $[1,5]\}$.
    \end{enumerate}
    
    \item Даны следующие диаграммы Хассе для отношений частичного порядка $P$ и $Q$:
    
    \begin{tabular}{cc}
        {\xymatrix{
            *{}
                &*{}
                    &
                        &*{}
                            &*{}
                                \\
            \ar@{-}[urr]
                &\ar@{-}[ur]
                    &*{}
                        &\ar@{-}[ul]
                            &\ar@{-}[ull]
                                \\
            \ar@{-}[u]
                &*{}
                    &\ar@{-}[ul]\ar@{-}[ur]
                        &*{}
                            &\ar@{-}[u]\ar@{-}[ul]
        }}
            &
            {\xymatrix{
                {}
                    &{}
                        &*{}
                            \\
                \ar@{-}[u]\ar@{-}[ur]
                    &\ar@{-}[u]
                        &\ar@{-}[ul]
                            \\
                *{}
                    &\ar@{-}[ul]\ar@{-}[u]\ar@{-}[ur]
                        &\ar@{-}[u]
            }}
                \\
            & \\
        $P$ & $Q$
    \end{tabular}
    
    Подберите и расставьте в узлах элементы $x$ такие, что
    \begin{enumerate}
        \item $x,y\in\mathbb{N}$. $x\leq y$, если $x$ делит нацело $y$;
        %унарное кодирование единицами, ограниченными нулями слева и справа!!! (1,2,3)=0101101110
        \item $x,y\in\{0,1\}^*$. $x\leq y$, если $x$ содержится в $y$, т.е. $y=\omega_1x\omega_2$;
        \item $x,y\in\mathbb{N}^2$. $x\leq y$, если $x=(a_x,b_x)$, $y=(a_y,b_y)$ и справедливо $(a_x\geq a_y)\land(b_x\leq b_y)$.
    \end{enumerate}
    
    \item Отношение частичного порядка <<предок>> задано матрицей (см. таблицу \ref{table:bo:ancestor}). Постройте диаграмму Хассе для данного отношения.
    
    \begin{table}
        \centering
        \begin{tabular}{l|ccccccc}
            Предок&
                    \rotatebox{90}{Гомер}&
                      \rotatebox{90}{Мардж}&
                        \rotatebox{90}{Абрахам}&
                          \rotatebox{90}{Жаклин}&
                            \rotatebox{90}{Барт}&
                              \rotatebox{90}{Лиза}&
                                \rotatebox{90}{Мэгги} \\
            \hline
            Гомер   &0&0&1&0&1&1&1\\
            Мардж   &0&0&0&1&1&1&1\\
            Абрахам &0&0&0&0&1&1&1\\
            Жаклин  &0&0&0&0&1&1&1\\
            Барт    &0&0&0&0&0&0&0\\
            Лиза    &0&0&0&0&0&0&0\\
            Мэгги   &0&0&0&0&0&0&0
        \end{tabular}
        \caption{Отношение <<предок>>}
        \label{table:bo:ancestor}
    \end{table}
    
    \item Отношения частичного порядка $P$ и $Q$ заданы матрицами смежности. Постройте соответствующие диаграммы Хассе.
    \begin{enumerate}
        \item 
        \(
            [P]=
            \begin{array}{c|ccccccc}
                 &a&b&c&d&e&f&g\\ \hline
                a&0&1&0&0&1&0&1\\
                b&0&0&0&0&0&0&0\\
                c&1&1&0&0&1&0&1\\
                d&1&1&1&0&1&0&1\\
                e&0&0&0&0&0&0&1\\
                f&1&1&1&0&1&0&1\\
                g&0&0&0&0&0&0&0
            \end{array},
        \)
        \(
            [Q]=
            \begin{array}{c|ccccccc}
                 &a&b&c&d&e&f&g\\ \hline
                a&0&0&1&0&1&0&0\\
                b&1&0&1&0&1&0&1\\
                c&0&0&0&0&0&0&0\\
                d&1&1&1&0&1&0&1\\
                e&0&0&0&0&0&0&0\\
                f&1&1&1&0&1&0&1\\
                g&1&0&1&0&1&0&0
            \end{array};
        \)
        
        \item 
        \(
            [P]=
            \begin{array}{c|ccccccc}
                 &a&b&c&d&e&f&g\\ \hline
                a&0&0&0&1&0&0&0\\
                b&0&0&0&1&0&0&1\\
                c&1&1&0&1&0&0&1\\
                d&0&0&0&0&0&0&0\\
                e&1&1&1&1&0&0&1\\
                f&1&1&1&1&0&0&1\\
                g&0&0&0&0&0&0&0
            \end{array},
        \)
        \(
            [Q]=
            \begin{array}{c|ccccccc}
                 &a&b&c&d&e&f&g\\ \hline
                a&0&1&0&0&0&0&0\\
                b&0&0&0&0&0&0&0\\
                c&1&1&0&1&0&0&1\\
                d&0&1&0&0&0&0&1\\
                e&0&1&0&1&0&1&1\\
                f&0&0&0&0&0&0&1\\
                g&0&0&0&0&0&0&0
            \end{array};
        \)
        
        \item 
        \(
            [P]=
            \begin{array}{c|ccccccc}
                 &a&b&c&d&e&f&g\\ \hline
                a&0&0&1&0&0&1&1\\
                b&0&0&0&0&1&1&1\\
                c&0&0&0&0&0&1&0\\
                d&1&1&1&0&1&1&1\\
                e&0&0&0&0&0&1&0\\
                f&0&0&0&0&0&0&0\\
                g&0&0&0&0&0&1&0
            \end{array},
        \)
        \(
            [Q]=
            \begin{array}{c|ccccccc}
                 &a&b&c&d&e&f&g\\ \hline
                a&0&0&1&0&1&1&0\\
                b&1&0&1&0&1&1&0\\
                c&0&0&0&0&1&1&0\\
                d&1&1&1&0&1&1&0\\
                e&0&0&0&0&0&1&0\\
                f&0&0&0&0&0&0&0\\
                g&1&1&1&1&1&1&0
            \end{array}.
        \)
    \end{enumerate}

    \item Постройте матрицы смежности для отношений порядка $P$ и $Q$ по приведенным диаграммам Хассе.
    
    \begin{tabular}{cc}
        {\xymatrix{
            *{}
                &*{}
                    &g
                        &*{}
                            &*{}
                                \\
            *{}
                &*{}
                    &f\ar@{-}[u]
                        &*{}
                            &*{}
                                \\
            b\ar@{-}[uurr]
                &c\ar@{-}[ur]
                    &*{}
                        &d\ar@{-}[ul]
                            &e\ar@{-}[uull]
                                \\
            *{}
                &*{}
                    &a\ar@{-}[ull]\ar@{-}[ul]\ar@{-}[ur]\ar@{-}[urr]
                        &*{}
                            &*{}
        }}
            &
            {\xymatrix{
                g
                    &*{}
                        &h
                            &*{}
                                \\
                *{}
                    &e\ar@{-}[ul]\ar@{-}[ur]
                        &*{}
                            &f\ar@{-}[ul]
                                \\
                c\ar@{-}[ur]
                    &*{}
                        &d\ar@{-}[ul]\ar@{-}[ur]
                            &*{}
                                \\
                *{}
                    &a\ar@{-}[ur]\ar@{-}[ul]
                        &*{}
                            &b\ar@{-}[ul]
            }}
                \\
            & \\
        $P$ & $Q$
    \end{tabular}
    
    \item Для вычисления множества $(\overline{A}\backslash(\overline{A}\cap B))\cup(\overline{A}\cap B)$ кодер (не владеющий алгеброй множеств) написал следующую программу на ассемблере для спецэвм:
\begin{verbatim}    
mov A,        [MA];
mov B,        [MB];
not [MA],     [M1];
cap [M1],[MB],[M2];
sub [M1],[M2],[M3];
cup [M2],[M3],[MR]; //результат в ячейке MR
//обозначения:
//   X  - константа
//  [X] - ячейка памяти X
//команды спецэвм:
//  mov X,[Y]   - константу X в ячейку Y
//  not [X],[Y] - дополнение значения в X в ячейку Y
//  cup [X],[Y],[Z] - объединение X и Y в ячейку Z
//  cap [X],[Y],[Z] - пересечение X и Y в ячейку Z
//  sub [X],[Y],[Z] - разность (X\Y) в ячейку Z
\end{verbatim}

    Необходимо:
    \begin{enumerate}
        \item проверить корректность исходной программы;
        \item построить граф потока вычислений (спэцэвм не допускает параллельное исполнение команд);
        \item оптимизировать расход памяти, построив отношение <<времена жизни не пересекаются>> и выделив в нем классы эквивалентности.
    \end{enumerate}
    Команды программы и их последовательность оставьте прежними: программа работает и производительность всех устраивает. Никто не хочет появления новых логических ошибок в алгоритме --- требуется лишь уменьшить расход памяти.

    \item Графы потоков вычислений представлены на рисунке \ref{fig:bo:calcFlowTask1}.
    \begin{figure}
        \centering
        \begin{tabular}{||c||}
            \hline\hline
            {\xymatrix{
                *{X} \ar@{-->}[r]
                    &O_1 \ar@{->}[r]^{r_1} \ar@{->}@/^3pc/[rr]^{r_1}
                        &O_2 \ar@{->}[r]^{r_2} \ar@{->}@/_3pc/[rr]^{r_2}
                            &O_3 \ar@{->}[r]^{r_3}
                                &O_4 \ar@{=>}[r]^{r_4}
                                    &*{R}
                        \\
                *{Y} \ar@{-->}@/_/[urr]
                    &*{}
                        &*{}
                            &*{}    
                                &*{}    
                                    &*{}
            }}
            \\ \hline
            {\xymatrix{
                *{X} \ar@{-->}[r]
                    &O_1 \ar@{->}[r]^{r_1} \ar@{->}@/^3pc/[rr]^{r_1}
                        &O_3 \ar@{->}[r]^{r_3}
                            &O_4 \ar@{=>}[r]^{r_4}
                                &*{R}
                        \\
                *{Y} \ar@{-->}[r]
                    &O_2 \ar@{->}[ur]^{r_2} \ar@{->}@/_2pc/[urr]^{r_2}
                        &*{}
                            &*{}    
                                &*{}    
            }}
            \\ \hline\hline
            
        \end{tabular}

        \caption{Графы потоков вычислений}
        \label{fig:bo:calcFlowTask1}
    \end{figure}
    
    Оптимизируйте затраты памяти, построив отношение <<времена жизни не пересекаются>> и выделив в нем классы эквивалентности.

    \item В таблице \ref{table:bo:excel} приведен фрагмент поля табличного процесора\footnote{Подобного MS Excel или Open Office Calc}. Задав отношение между ячейками поля отношение порядка $x\leq y$ (т.е. <<ячейка $x$ должна быт вычислена раньше, чем $y$>>), постройте диаграмму Хассе, которая, очевидно, и будет задавать порядок вычисления значений в ячейках. Проведите вычисления.
    
    \begin{table}
        \centering
        \begin{tabular}{|l||l|l|l|}
            \hline
                &a      &b      &c     \\
            \hline\hline
            1   &=b1+1  &=b2+1  &=b1+1 \\ \hline
            2   &=a1+a3 &1      &=c1+b2\\ \hline
            3   &=c3+1  &=b2+1  &=c2+b3\\ \hline
        \end{tabular}
        \caption{Фрагмент листа табличного процессора}
        \label{table:bo:excel}
    \end{table}

\end{enumerate}


\paragraph{Программирование}

\begin{enumerate}
    \item Пусть на некотором множестве объектов $A^2$ программистом заданы отношения <<равно>> (\verb"x==y") и <<меньше>> (\verb"y<x"). То есть программист реализовал соответствующие функции двух аргументов, возвращающие истину или ложь:
\begin{verbatim}    
bool operator== (A x, A y) {return /* сложная проверка x==y */;}
bool operator<  (A x, A y) {return /* сложная проверка x<y  */;}
\end{verbatim}    
    С помощью логических функций выразить отношения <<больше>> (\verb">"), <<больше или равно>>  (\verb">="), <<меньше или равно>>  (\verb"<=") и <<различны>>  (\verb"!="). 
    
    При этом доступны функции основного логического базиса: \emph{И} (\verb"||"), \emph{И} (\verb"&&") и \emph{НЕ} (\verb"!"). Логическая функция выполняется за такт машинного времени.
    
    Очевидно, пожалуй:
\begin{verbatim}    
bool operator!= (A x, A y) {return !(x == y);}
\end{verbatim}    

    Необходимо, чтобы остальные проверки занимали минимум времени.
\end{enumerate}

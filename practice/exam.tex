\chapter{Экзамен}


\section{Вопросы к экзамену}

\begin{enumerate}
    \item Множества. Способы задания. Подмножества. Возможные отношения между множествами. Операции над множествами. Диаграммы Эйлера-Венна. Стандартные обозначения для важнейших множеств.
    
    \item Множества. Алгебра множеств. Свойства операций алгебры множеств. Представления множеств в ЭВМ.
    
    \item Множества. Декартово (прямое) произведение множеств. Степень множества. Булеан. Мощность конечных множеств.
    
    \item Регулярные множества и регулярные выражения. Формальное определение. Использование на практике.
    
    \item $n$-местные отношения. Виды отношений. Бинарные отношения. Способы задания бинарных отношений. Композиция бинарных отношений.
    
    \item $n$-местные отношения. Бинарные отношения. Универсальное и тождественное бинарные отношения. Операции на матрицах смежности бинарных отношений.
    
    \item Функции. Виды функций.
    
    \item Мощность множества. Бесконечные множества. Кардинальные числа. Свойства кардиналов конечных и бесконечных множеств. Диагональная процедура Кантора.
    
    \item Мощность множества. Бесконечные множества. Теоремы Кантора и Кантора-Бернштейна.
    
    \item Бинарные отношения. Свойства бинарных отношений. Замыкания относительно свойств.
    
    \item Отношение эквивалентности. Классы эквивалентности.
    
    \item Отношение порядка. Виды отношений порядка. Минимальные, максимальные, наибольшие и наименьшие элементы. Инфимум и супремум. Диаграмма Хассе.
    
    \item Реляционная алгебра. Реляционные базы данных. Операции реляционной алгебры на примере оператора select языка SQL.
    
    \item Нечеткие множества. Операции над нечеткими множествами.
    
    \item Нечеткие множества. Формы функций принадлежности. $t$-норма и $t$-конорма.
    
    \item Формальное определение кодирования. Назначение кодирования. Определение количества информации по Шеннону. Единицы измерения информации. 
    
    \item Оптимальное кодирование. Информативность источника событий. Информативность источники информации. Алгоритм Хаффмана. Алгоритм Фано.
    
    \item Сжатие информации. Словарные методы. Сжатие методом Лемпела-Зива.
    
    \item Кодирование с целью защиты от ошибок. Стратегии. Код Хемминга.
    
    \item Теория графов. Основные определения и способы задания графов.
    
    \item Связность в графах. Основные определения. Алгоритм выделения компонент связности.
    
    \item Маршруты в графах. Нахождение кратчайших маршрутов. Алгоритм Дейкстры.
    
    \item Деревья в графах. Алгоритмы построения остовных деревьев.
    
    \item Изоморфизм в графах.
    
    \item Планарные графы.
    
    \item Циклы в графах.
    
    \item Индукция. Базовая и общая форма доказательств по индукции.
    
    \item Реккурентные математические формулы. Формулы первого и второго рода.
    
    \item Рекурсивные структуры данных и вычисления. Инфиксная, префиксная и постфиксная записи формул.
    
    \item Представление рекурсивных структур в ЭВМ.
    
    \item Рекурсивные алгоритмы.
    
    \item Анализ алгоритмов. Скорости роста и классы входных данных.
    
    \item Анализ алгоритмов. $P$ и $NP$.
    
    \item Анализ алгоритмов. Вычислимое и невычислимое. Машина Тьюринга. Неразрешимость проблемы останова.
    
\end{enumerate}


\section{Экзаменационные задачи}

\begin{enumerate}
    \item Задание множеств.
    \item Аналитические преобразования множеств.
    \item Построение регулярных выражений.
    \item Поиск композиции отношений.
    \item Оценка мощности конечных множеств.
    \item Выделение классов эквивалентности.
    \item Построение диаграммы Хассе.
    \item Операции над отношениями (SQL, select).
    \item Оптимальное кодирование источника (Хаффман или Фано).
    \item Сжатие. Алгоритм Лемпела-Зива.
    \item Защита от ошибок. Код Хемминга.
    \item Доказательства по индукции.
    \item Перевод формулы в постфиксную форму.
    \item Выделение компонент связности в графах.
    \item Выделение остовного дерева.
    \item Поиск кратчайших маршрутов.
    \item Анализ программ.
    \item Проектирование машины Тьюринга.
\end{enumerate}
